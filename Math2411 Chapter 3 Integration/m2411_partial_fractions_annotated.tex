\documentclass[11pt]{article}
\usepackage[suffix=Solutions]{teaching-header}

\def\classnum{2411}
\def\classtitle{Calculus II}
\def\classtitleshort{Calc 2}
\def\classsec{H01}
\def\instructor{Dr. Rostermundt}
\def\classterm{Spring 2025}


%%%%%%%%%%%%%%%%%%%%%%%%%%%%%%%%%%%%%%%%%%%%%%%%%%%%%%%%%%%%%%%%%%%%%%%%%%%%
%%%%%%%%%%%%%%%%%%%%%%%%%%%%%%%%%%%%%%%%%%%%%%%%%%%%%%%%%%%%%%%%%%%%%%%%%%%%

%This is defined in the teaching-header style file
%\ifnum\printsol=0 (when no solutions printed)
%Do something
%	\else  (when solutions are printed)
%Do something else
%\fi


% Package and setting included in teachin-header style file
%\RequirePackage{amsmath,amsfonts,amssymb,amsthm,graphicx, pgfplots, tcolorbox, xcolor,latexsym,color,verbatim,float,xcolor,setspace}
%%tikzsymbols
%
%\RequirePackage{enumerate}
%\RequirePackage{multicol}
%\RequirePackage{tikz}
%\RequirePackage{cancel}
%\usetikzlibrary{shapes.geometric}
%\usetikzlibrary{calc, positioning, arrows}
%\RequirePackage[margin=1in,letterpaper]{geometry}
%\RequirePackage[colorlinks=true,allcolors=blue]{hyperref}
%\usepackage[final]{pdfpages}
%%\usepackage{capt-of}
%
%
%\setlength{\textheight}{9in}
%\setlength{\textwidth}{6.5in}
%\addtolength{\topmargin}{0cm}
%%\addtolength{\oddsidemargin}{0cm}
%\parindent=0in
%\parskip=.35em
%\singlespacing
%%\pagestyle{empty}  % remove page numbers

%Add captions without being in figure environment
%\captioof{figure}{\text}\label[fig:]
\usepackage{capt-of}
\usepackage{mathtools}

\vfuzz2pt % Don't report over-full v-boxes if over-edge is small
\hfuzz2pt % Don't report over-full h-boxes if over-edge is small


%%%%%%%%%%%%%%%%%%%%%%%%%%%%%%%%%%%%%%%%%%%%%%%%%%%%%%
%%%%%%%%%%%%%%%%%%%%%%%%%%%%%%%%%%%%%%%%%%%%%%%%%%%%%%

\pagestyle{myheadings}

%%%%%%%%%%%%%%%%%%%%%%%%%%%%%%%%%%%%%%%%%%%%%%%%%%%%%%
%%%%%%%%%%%%%%%%%%%%%%%%%%%%%%%%%%%%%%%%%%%%%%%%%%%%%%


%%%%%%%%%%%%%%%%%%%%%%%%%%%%%%%%%%%%%%%%%%%%%%%%%%%%%%
%%%%%%%%%%%%%%%%%%%%%%%%%   Document Body   %%%%%%%%%%
%%%%%%%%%%%%%%%%%%%%%%%%%%%%%%%%%%%%%%%%%%%%%%%%%%%%%%

%Information from classinfo.tex file
%\def\classnum{2411}
%\def\classtitle{Calculus II}
%\def\classtitleshort{Calc 2}
%\def\classsec{001}
%\def\instructor{Rostermundt}
%\def\classterm{Fall 2024}
\def\topic{Partial Fractions}
\def\topicshort{Partial Fractions}

	\title{\vspace{-1in}Math\classnum\;-\;\classtitle\\
	%Section\;\classsec\;-\;\classterm\\
	Guided Lecture Notes\\
	\topic}
	\author{University of Colorado Denver / College of Liberal Arts and Sciences}
	\date{Department of Mathematics}

	\markright{Math\classnum\;-\;\classtitleshort, University of Colorado Denver,\;\topicshort}


%%%%%%%%%%%%%%%%%%%%%%%%%%%%%%%%%%%%%%%%%%%%%%%%%%%%%%
\begin{document}\maketitle\thispagestyle{empty}
%%%%%%%%%%%%%%%%%%%%%%%%%%%%%%%%%%%%%%%%%%%%%%%%%%%%%%

\hrule

\section*{\topic\, Introduction:}

Our objective is to integrate rational functions where $u$-sub or algebraic manipulations are not helpful or clear. Let's look at some examples. 
\vskip 5mm
\[\int\ds\frac{x^2+4x-1}{x^3+6x^2-3x+7}\,dx\qquad\int\ds\frac{2}{x^2+4x+8}\,dx\]
The first integral can be solved using $u$-substitution letting $u=x^3+6x^2-3x+7$ so that $du=\left(3x^2+12x-3\right)\,dx$. We then have
\beq
\int\ds\frac{x^2+4x-1}{x^3+6x^2-3x+7}\,dx&=&\ds\frac{1}{3}\int\ds\frac{3\big(x^2+4x-1\big)}{x^3+6x^2-3x+7}\,dx\\
&=&\ds\frac{1}{3}\int\ds\frac{3x^2+12x-3}{x^3+6x^2-3x+7}\,dx\\
&=&\ds\frac{1}{3}\ds\int\ds\frac{du}{u}\\
&=&\ds\frac{1}{3}\ln|u|+C\\
&=&\ds\frac{1}{3}\ln\left|x^3+6x^2-3x+7\right|+C
\eeq

The second integral can be solved by algebraic manipulations, a $u$-sub, and a known antiderivative.
\beq
\int\ds\frac{2}{x^2+4x+8}\,dx&=&\int\ds\frac{2}{x^2+4x+4+4}\,dx\\
&=&\int\ds\frac{2}{(x+2)^2+4}\,dx\\
&=&\ds\int\ds\frac{2}{u^2+4}\,du\qquad\text{(Letting }u=x+2)\\
&=&\tan^{-1}(u/2)+C\\
&=&\tan^{-1}\left(\ds\frac{x+2}{2}\right)+C
\eeq

But some integrals of rational expressions resist any $u$-sub or algebraic manipulations. For example,
\[\ds\int\ds\frac{2x+5}{x^2+4x-5}\,dx.\]

\vfill\eject

We can not turn this into a log-form by $u$-substitution since with $u=x^2+4x-5$ we have $du=\left(2x+4\right)\,dx$ and the numerator is not a scalar multiple of the denominator.

\vskip 5mm

So we split up the integral.
\[\ds\int\ds\frac{2x+5}{x^2+4x-5}\,dx=\ds\int\ds\frac{2x+4}{x^2+4x-5}\,dx+\ds\int\ds\frac{1}{x^2+4x-5}\,dx\]
The first integral is a simple $u$-substitution. But what about the second integral. We could try to complete the square as follows.
\beq
\ds\int\ds\frac{1}{x^2+4x-5}\,dx&=&\ds\int\ds\frac{1}{x^2+4x-4-1}\,dx\\
&=&\ds\int\ds\frac{1}{(x-2)^2-1}\,dx\\
&=&\ds\int\ds\frac{1}{u^2-1}\,du\qquad\text{(Letting }u=x-2)
\eeq
But the new integral is not in a familiar form. So it looks like we need a new strategy. We will use the following algebraic fact. It is a good exercise to check this equality.
\[\ds\frac{2x+5}{x^2+4x-5}=\ds\frac{7}{6(x-1)}+\ds\frac{5}{6(x+5)}\]
Now we have the following.
\beq
\ds\int\ds\frac{2x+5}{x^2+4x-5}\,dx&=&\ds\int\ds\frac{7}{6(x-1)}+\ds\frac{5}{6(x+5)}\,dx\\
&=&\ds\frac{7}{6}\ds\int\ds\frac{1}{x-1}\,dx+\ds\frac{5}{6}\ds\int\ds\frac{1}{x+5}\,dx\\
&=&\ds\frac{7}{6}\ln|x-1|+\ds\frac{5}{6}\ln|x+5|+C\\
\eeq
So if we can break apart a complicated rational expression into a sum of simpler rational expressions, often the new expressions will be easier to handle as far as integration.

\section*{Partial Fractions Method}

Let's see how the previous fraction was decomposed into simpler fractions. 
\vskip 5mm

\noindent{\bf\emph{Step \#1}:} Factor the denominator.
\[x^2+4x-5=(x-1)(x+5)\]
Do you notice anything about these factors?
\vskip 1.5cm

\noindent{\bf\emph{Step \#2}:} Try to break up the fraction so that the parts correspond to the factors.
\[\ds\frac{2x+5}{x^2+4x-5}=\ds\frac{2x+5}{(x-1)(x+5)}=\ds\frac{A}{x-1}+\ds\frac{B}{x+5}\]

\vskip 5mm

\noindent{\bf\emph{Step \#3}:} Solve for $A$ and $B$. That is, find values of $A$ and $B$ that satisfy the previous equality. This is the crux of the process. We start by combining terms on the RHS and then seeting numerators equal.
\[\ds\frac{2x+5}{(x-1)(x+5)}=\ds\frac{A(x+5)+B(x-1)}{(x-1)(x+5)}\quad\xRightarrow[\stackrel{\text{\textcolor{red}{Set Numerators}}}{\text{\tiny\textcolor{red}{Equal}}}]{}\quad 2x+5=A(x+5)+B(x-1)\]
\vskip 5mm
Methods to solve for $A$ and $B$:
\begin{itemize}
\item[$\bullet$] {\bf\emph{\underline{Method of Strategic Substitution}:}} Set $x=1$ and substitute to get
\[7=6A\quad\Longrightarrow\quad A=7/6.\]
Next set $x=-5$ and substitute to get
\[-5=-6B\quad\Longrightarrow\quad B=5/6.\]
So we have
\[\ds\frac{2x+5}{x^2+4x-5}=\ds\frac{A}{x-1}+\ds\frac{B}{x+5}=\ds\frac{7}{6(x-1)}+\ds\frac{5}{6(x+5)}.\]

\vskip 1cm

\item[$\bullet$] {\bf\emph{\underline{Method of Equating Coefficients}:}} Expand both sides and equate coefficients.
\[2x+5=(A+B)x+(5B-A)\]
We see that
\beq
A+B&=&2\\
5A-B&=&1\\
\eeq
Solving this system of equations gives us $A=7/6$ and $B=5/6$.
So we have
\[\ds\frac{2x+5}{x^2+4x-5}=\ds\frac{A}{x-1}+\ds\frac{B}{x+5}=\ds\frac{7}{6(x-1)}+\ds\frac{5}{6(x+5)}.\]

\end{itemize}

The partial fractions decomposition is an algebraic technique to simplify an integral. There is no calculus involved in the partial fractions.
\vskip 5mm

Let's work an example on your own.

\section*{Partial Fractions Examples}

%%%%%%%%%%%%%%%%%%%%%%%%%%%%%%%%%%%%%%%%%%%%%%%%%%%%%%%%%
%%%%%%%%%%%%%%%%%%%%%%%%%%%%%%%%%%%%%%%%%%%%%%%%%%%%%%%%%


\begin{example} Evaluate $\ds\int\ds\frac{x+3}{x^3-x^2-2x}\,dx$
\vskip 5mm

\noindent{\bf\emph{\underline{Workspace}:}} 

\vfill\eject

\ifnum\longform=1
	\begin{boxsolution}
\vspace*{5mm}
We start by factoring the denominator and breaking apart to get
{\small
\[\ds\frac{x+3}{x^3-x^2-2x}=\ds\frac{x+3}{x(x-2)(x+1)}=\ds\frac{A}{x}+\ds\frac{B}{x-2}+\ds\frac{C}{x+1}=\ds\frac{A(x+1)(x-2)+Bx(x+1)+Cx(x-2)}{x(x-2)(x+1)}\]
}
After putting the pieces back together we get
\[\xRightarrow[\stackrel{\text{\textcolor{red}{Set Numerators}}}{\text{\tiny\textcolor{red}{Equal}}}]{}\quad x+3=A(x+1)(x-2)+Bx(x+1)+Cx(x-2)\quad\text{\textcolor{red}{(Eqn. 3.8)}}\]
We now solve equation 3.8 for $A$, $B$, and $C$ using one of our described methods.
\vskip 1cm
\begin{minipage}[]{6.5in}
\begin{flushleft}
\qquad\includegraphics[trim= 0cm 1cm 0cm 0cm, clip=true, scale=0.65]{partial_fractions_gr01.jpg}
%\captionof{figure}{}
\label{fig:}
\end{flushleft}
\end{minipage}
\vskip 1cm
\begin{minipage}[]{6.5in}
\begin{flushleft}
\qquad\includegraphics[scale=0.65]{partial_fractions_gr02.jpg}
%\captionof{figure}{}
\label{fig:}
\end{flushleft}
\end{minipage}

\vspace*{5mm}
	\end{boxsolution}

\vfill\eject

	\begin{boxsolutioncont}
\[\vdots\]
Now we're ready for the integral.
\vskip 2mm
\beq
\ds\int\ds\frac{x+3}{x^3-x^2-2x}\,dx&=&-\ds\int\ds\frac{1}{x}\,dx+\ds\frac{4}{3}\ds\int\ds\frac{1}{x-2}\,dx-\ds\frac{1}{3}\ds\int\ds\frac{1}{x+1}\,dx\\
\\
&=&-\ln|x|+\ds\frac{4}{3}\ln|x-2|-\ds\frac{1}{3}\ln|x+1|+C
\eeq
\vspace*{5mm}
	\end{boxsolutioncont}
\vskip 5mm

\fi 

\end{example}

Let's work another example together.
\vskip 5mm


%%%%%%%%%%%%%%%%%%%%%%%%%%%%%%%%%%%%%%%%%%%%%%%%%%%%%%%%%
%%%%%%%%%%%%%%%%%%%%%%%%%%%%%%%%%%%%%%%%%%%%%%%%%%%%%%%%%


\begin{example} Evaluate $\ds\int\ds\frac{x-2}{(2x-1)^2(x-1)}\,dx$
\vskip 5mm

\noindent{\bf\emph{\underline{Workspace}:}} 

\vfill\eject

\ifnum\longform=1
	\begin{boxsolution}
\vspace*{5mm}
The denominator is already factored. Then we notice that the factor $(2x-1)$ is a repeated factor. In fact it's repeated twice and so this factor gives us two pieces, one for each power, as follows.
\[\ds\frac{x-2}{(2x-1)^2(x-1)}=\underbrace{\ds\frac{A}{2x-1}+\ds\frac{B}{(2x-1)^2}}_{\stackrel{\text{\textcolor{red}{Repeated factors require}}}{\text{\tiny\textcolor{red}{multiple components}}}}+\ds\frac{C}{x-1}=\ds\frac{A(2x-1)(x-1)+B(x-1)+C(2x-1)^2}{(2x-1)^2(x-1)}\]
After putting the pieces back together we can solve an equation.
\[\xRightarrow[\stackrel{\text{\textcolor{red}{Set Numerators}}}{\text{\tiny\textcolor{red}{Equal}}}]{}\quad x-2=A(2x-1)(x-1)+B(x-1)+C(2x-1)^2\]
We now solve the equation for $A$, $B$, and $C$ using one of our described methods. Equating coefficients we get $A=2$, $B=3$, $C=-1$. Then we have
\beq
\ds\int\ds\frac{x-2}{(2x-1)^2(x-1)}\,dx&=&2\ds\int\ds\frac{1}{2x-1}\,dx+3\ds\int\ds\frac{1}{(2x-1)^2}\,dx-\ds\int\ds\frac{1}{x-1}\,dx\\
&=&\ln|2x-1|-\ds\frac{3}{2(2x-1)}-\ln|x-1|+C
\eeq
\vspace*{2mm}
	\end{boxsolution}
\vskip 5mm

\begin{minipage}[]{6.5in}
\begin{center}
\includegraphics[trim= 0cm 0cm 0cm 0cm, clip=true, scale=0.6]{partial_fractions_gr03.jpg}
\includegraphics[trim= 0cm 2.5cm 0cm 0cm, clip=true, scale=0.602]{partial_fractions_gr04.jpg}
%\captionof{figure}{}
\label{fig:}
\end{center}
\end{minipage}

\vfill\eject

\begin{minipage}[]{6.5in}
\begin{center}
\includegraphics[trim= 0cm 0cm 0cm 11cm, clip=true, scale=0.6]{partial_fractions_gr04.jpg}
%\captionof{figure}{}
\label{fig:}
\end{center}
\end{minipage}

\fi

\end{example}
\vskip 5mm


%%%%%%%%%%%%%%%%%%%%%%%%%%%%%%%%%%%%%%%%%%%%%%%%%%%%%%%%%
%%%%%%%%%%%%%%%%%%%%%%%%%%%%%%%%%%%%%%%%%%%%%%%%%%%%%%%%%



\begin{example} Evaluate $\ds\int\ds\frac{2x-3}{x^3+x}\,dx$
\vskip 5mm

\noindent{\bf\emph{\underline{Workspace}:}} 

\vfill\eject

\ifnum\longform=1
	\begin{boxsolution}
\vspace*{5mm}
The denominator needs to be factored. 
\[x^3+x=x(x^2+1)\]
Then we notice that the factor $(x^2+1)$ is a quadratic factor. Looking at the previous table, item c., we break apart as follows.
\[\ds\frac{2x-3}{x^3+x}=\ds\frac{2x-3}{x(x^2+1)}=\underbrace{\ds\frac{Ax+B}{x^2+1}}_{\stackrel{\text{\textcolor{red}{Quadratic factors require}}}{\text{\tiny\textcolor{red}{Linear Numerator}}}}+\ds\frac{C}{x}=\ds\frac{(Ax+B)x+C(x^2+1)}{x(x^2+1)}\]
After putting the pieces back together we can solve an equation.
\[\xRightarrow[\stackrel{\text{\textcolor{red}{Set Numerators}}}{\text{\tiny\textcolor{red}{Equal}}}]{}\quad 2x-3=(Ax+B)x+C(x^2+1)\]
We now solve the equation for $A$, $B$, and $C$ using one of our described methods. Equating coefficients we get $A=3$, $B=2$, $C=-3$. Then we have
\beq
\ds\int\ds\frac{2x-3}{x^3+x}\,dx&=&\ds\int\ds\frac{3x+2}{x^2+1}\,dx-3\ds\int\ds\frac{1}{x}\,dx\\
&=&\ds\frac{3}{2}\underbrace{\ds\int\ds\frac{2x}{x^2+1}\,dx}_{\stackrel{\text{\textcolor{red}{This is a}}}{\text{\tiny\textcolor{red}{Log-Form}}}}+2\underbrace{\ds\int\ds\frac{1}{x^2+1}\,dx}_{\stackrel{\text{\textcolor{red}{Arctangent}}}{\text{\tiny\textcolor{red}{Form}}}}-3\underbrace{\ds\int\ds\frac{1}{x}\,dx}_{\stackrel{\text{\textcolor{red}{Log}}}{\text{\tiny\textcolor{red}{Form}}}}\\
&=&\ds\frac{3}{2}\ln(x^2+1)+2\tan^{-1}(x)-3\ln|x|+C\\
\eeq
\vspace*{5mm}
	\end{boxsolution}
\vskip 5mm

\fi

\end{example}


Let's have you work another example independently.
 
\vskip 5mm


%%%%%%%%%%%%%%%%%%%%%%%%%%%%%%%%%%%%%%%%%%%%%%%%%%%%%%%%%
%%%%%%%%%%%%%%%%%%%%%%%%%%%%%%%%%%%%%%%%%%%%%%%%%%%%%%%%%

\begin{example} Evaluate $\ds\int\ds\frac{1}{x^3-8}\,dx$
\vskip 2mm
\hint Since $x=2$ is a zero of $x^3-8$, in order to factor the denominator divide $x-2$ into $x^3-8$.
\vskip 5mm

\noindent{\bf\emph{\underline{Workspace}:}} 

\vfill\eject

\noindent{\bf\emph{\underline{Workspace Continued}:}} 

\vfill\eject

\ifnum\longform=1
	\begin{boxsolution}
\vspace*{5mm}
\begin{minipage}[]{6.5in}
\begin{flushleft}
\includegraphics[trim= 0cm 0cm 2cm 0cm, clip=true, scale=0.7]{partial_fractions_gr05.jpg}
\vskip 5mm
\includegraphics[trim= 0cm 0cm 2cm 0cm, clip=true, scale=0.7]{partial_fractions_gr06.jpg}
%\captionof{figure}{}
\label{fig:}
\end{flushleft}
\end{minipage}

\vspace*{5mm}
	\end{boxsolution}
\vskip 5mm

\fi 

\end{example}


%%%%%%%%%%%%%%%%%%%%%%%%%%%%%%%%%%%%%%%%%%%%%%%%%%%%%%%%%
%%%%%%%%%%%%%%%%%%%%%%%%%%%%%%%%%%%%%%%%%%%%%%%%%%%%%%%%%


\section*{What if Numerator Degree is Too Large?}

%%%%%%%%%%%%%%%%%%%%%%%%%%%%%%%%%%%%%%%%%%%%%%%%%%%%%%%%%
%%%%%%%%%%%%%%%%%%%%%%%%%%%%%%%%%%%%%%%%%%%%%%%%%%%%%%%%%
\vskip 2mm
For partial fractions to work the degree of the numerator MUST BE smaller than the degree of the denominator! 

\begin{example} Evaluate $\ds\int\ds\frac{x^2+3x+1}{x^2-4}\,dx$
\vskip 5mm

\noindent{\bf\emph{\underline{Workspace}:}} 

\vfill\eject

\ifnum\longform=1
	\begin{boxsolution}
\vspace*{5mm}

\begin{minipage}[]{6.5in}
\begin{flushleft}
\includegraphics[trim= 0cm 0cm 0.6cm 0cm, clip=true, scale=0.69]{partial_fractions_gr07.jpg}
%\captionof{figure}{}
\label{fig:}
\end{flushleft}
\end{minipage}
\vspace*{5mm}
	\end{boxsolution}
\vskip 5mm

\fi 

\end{example}


%%%%%%%%%%%%%%%%%%%%%%%%%%%%%%%%%%%%%%%%%%%%%%%%%%%%%%%%%
%%%%%%%%%%%%%%%%%%%%%%%%%%%%%%%%%%%%%%%%%%%%%%%%%%%%%%%%%
%%%%%%%%%%%%%%%%%%%%%%%%%%%%%%%%%%%%%%%%%%%%%%%%%%%%%%%%%
%%%%%%%%%%%%%%%%%%%%%%%%%%%%%%%%%%%%%%%%%%%%%%%%%%%%%%%%%

	
%%%%%%%%%%%%%%%%%%%%%%%%%%%%%%%%%%%%%%%%%%%%%%%%%%%%%%
%%%%%%%%%%%%%%%%%%%%%%%%%%%%%%%%%%%%%%%%%%%%%%%%%%%%%%


\ifnum\longform=1
\vskip 1cm
\hrule
\vskip 5mm
\begin{center}{\bf Please let me know if you have any questions, comments, or corrections!}
\end{center}	

\fi

%%%%%%%%%%%%%%%%%%%%%%%%%%%%%%%%%%%%%%%%%%%%%%%%%%%%%%
\end{document}
%%%%%%%%%%%%%%%%%%%%%%%%%%%%%%%%%%%%%%%%%%%%%%%%%%%%%%
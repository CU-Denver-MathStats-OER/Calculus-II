\documentclass[11pt]{article}
\usepackage[suffix=Solutions]{teaching-header}

\def\classnum{2411}
\def\classtitle{Calculus II}
\def\classtitleshort{Calc 2}
\def\classsec{H01}
\def\instructor{Dr. Rostermundt}
\def\classterm{Spring 2025}


%%%%%%%%%%%%%%%%%%%%%%%%%%%%%%%%%%%%%%%%%%%%%%%%%%%%%%%%%%%%%%%%%%%%%%%%%%%%
%%%%%%%%%%%%%%%%%%%%%%%%%%%%%%%%%%%%%%%%%%%%%%%%%%%%%%%%%%%%%%%%%%%%%%%%%%%%

%This is defined in the teaching-header style file
%\ifnum\printsol=0 (when no solutions printed)
%Do something
%	\else  (when solutions are printed)
%Do something else
%\fi


% Package and setting included in teachin-header style file
%\RequirePackage{amsmath,amsfonts,amssymb,amsthm,graphicx, pgfplots, tcolorbox, xcolor,latexsym,color,verbatim,float,xcolor,setspace}
%%tikzsymbols
%
%\RequirePackage{enumerate}
%\RequirePackage{multicol}
%\RequirePackage{tikz}
%\RequirePackage{cancel}
%\usetikzlibrary{shapes.geometric}
%\usetikzlibrary{calc, positioning, arrows}
%\RequirePackage[margin=1in,letterpaper]{geometry}
%\RequirePackage[colorlinks=true,allcolors=blue]{hyperref}
%\usepackage[final]{pdfpages}
%%\usepackage{capt-of}
%
%
%\setlength{\textheight}{9in}
%\setlength{\textwidth}{6.5in}
%\addtolength{\topmargin}{0cm}
%%\addtolength{\oddsidemargin}{0cm}
%\parindent=0in
%\parskip=.35em
%\singlespacing
%%\pagestyle{empty}  % remove page numbers

%Add captions without being in figure environment
%\captioof{figure}{\text}\label[fig:]
\usepackage{capt-of}
\usepackage{mathtools}

\vfuzz2pt % Don't report over-full v-boxes if over-edge is small
\hfuzz2pt % Don't report over-full h-boxes if over-edge is small


%%%%%%%%%%%%%%%%%%%%%%%%%%%%%%%%%%%%%%%%%%%%%%%%%%%%%%
%%%%%%%%%%%%%%%%%%%%%%%%%%%%%%%%%%%%%%%%%%%%%%%%%%%%%%

\pagestyle{myheadings}

%%%%%%%%%%%%%%%%%%%%%%%%%%%%%%%%%%%%%%%%%%%%%%%%%%%%%%
%%%%%%%%%%%%%%%%%%%%%%%%%%%%%%%%%%%%%%%%%%%%%%%%%%%%%%


%%%%%%%%%%%%%%%%%%%%%%%%%%%%%%%%%%%%%%%%%%%%%%%%%%%%%%
%%%%%%%%%%%%%%%%%%%%%%%%%   Document Body   %%%%%%%%%%
%%%%%%%%%%%%%%%%%%%%%%%%%%%%%%%%%%%%%%%%%%%%%%%%%%%%%%

%Information from classinfo.tex file
%\def\classnum{2411}
%\def\classtitle{Calculus II}
%\def\classtitleshort{Calc 2}
%\def\classsec{001}
%\def\instructor{Rostermundt}
%\def\classterm{Fall 2024}
\def\topic{Integration by Parts}
\def\topicshort{Integration by Parts}

	\title{\vspace{-1in}Math\classnum\;-\;\classtitle\\
	%Section\;\classsec\;-\;\classterm\\
	Guided Lecture Notes\\
	\topic}
	\author{University of Colorado Denver / College of Liberal Arts and Sciences}
	\date{Department of Mathematics}

	\markright{Math\classnum\;-\;\classtitleshort, University of Colorado Denver,\;\topicshort}
	

%%%%%%%%%%%%%%%%%%%%%%%%%%%%%%%%%%%%%%%%%%%%%%%%%%%%%%
\begin{document}\maketitle\thispagestyle{empty}
%%%%%%%%%%%%%%%%%%%%%%%%%%%%%%%%%%%%%%%%%%%%%%%%%%%%%%

\hrule

\section*{\topic\; Introduction:}

\[\ds\int xe^{x^2}\,dx\qquad\text{versus}\qquad\ds\int xe^{x}\,dx\]

We need a technique to evaluate integrals of products, where $u$-sub does not work. Something like a ``product rule" for integration.

\[\ds\frac{d}{dx}\big[u(x)v(x)\big]=u'(x)v(x)+u(x)v'(x)\quad\xRightarrow[\stackrel{\text{\textcolor{red}{Integrate both sides}}}{\text{\textcolor{red}{with respect to x}}}]{}\quad u(x)v(x)=\ds\int u'(x)v(x)\,dx+\ds\int u(x)v'(x)\,dx\]

We can rewrite this as

\[\ds\int u(x)v'(x)\,dx=u(x)v(x)-\ds\int v(x)u'(x)\,dx\]

or in differential form (by suppressing the variable $x$).

\[\ds\int \underbrace{u(x)}_{\text{Write as }u\;\;}\underbrace{v'(x)\,dx}_{\text{Write as }dv}=\underbrace{u(x)v(x)}_{\text{Write as }uv}-\ds\int \underbrace{v(x)}_{\text{Write as }v\;\;}\underbrace{u'(x)\,dx}_{\text{Write as }dv}=\ds\int udv=uv-\ds\int vdu.\]


%%%%%%%%%%%%%%%%%%%%%%%%%%%%%%%%%%%%%%%%%%%%%%%%%%%%%%%%%
%%%%%%%%%%%%%%%%%%%%%%%%%%%%%%%%%%%%%%%%%%%%%%%%%%%%%%%%%


\section*{Integration by Parts Examples:}

Let’s work an example together.
\vskip 2mm

\begin{example}

Evaluate the integral $\ds\int xe^{x}\,dx$ 
\vskip 5mm
\noindent{\bf\emph{\underline{Workspace}:}}
\vfill\eject

\ifnum\longform=1
	\begin{boxsolution}
\vspace*{5mm}
Choose $u(x)=x$ and $v'(x)=e^x$ so that $u'(x)=\frac{d}{dx}[x]=1$ and then we have $v(x)=\int e^x\,dx = e^x$ \\\\ 
So we have that
\[\ds\int xe^{x}\,dx=xe^x-\int e^x\cdot 1\,dx=xe^x-e^x+C=e^x(x-1)+C.\]
So 
\[\int xe^x\,dx=xe^x-e^x+C=e^x(x-1)+C.\]
\vspace*{5mm}
	\end{boxsolution}
\vskip 5mm

\fi

\end{example}


%%%%%%%%%%%%%%%%%%%%%%%%%%%%%%%%%%%%%%%%%%%%%%%%%%%%%%%%%
%%%%%%%%%%%%%%%%%%%%%%%%%%%%%%%%%%%%%%%%%%%%%%%%%%%%%%%%%

Suppose we had decided to choose $u(x)=e^x$ and $v'(x)=x$. Then $u'(x)=e^x$ and $v(x)=x^2/2$. Check the solution using integration by parts. What do you notice?
\vskip 5mm
\noindent{\bf\emph{\underline{Workspace}:}}
\vfill\eject

\ifnum\longform=1
	\begin{boxsolution}
\vspace*{5mm}
If we choose $u(x)=e^x$ and $v'(x)=x$. Then $u'(x)=e^x$ and $v(x)=x^2/2$ and we have
\[\ds\int xe^{x}\,dx=\ds\frac{1}{2}x^2e^x-\ds\frac{1}{2}\ds\int x^2e^x\,dx.\]
This expression is correct, but the new integral produced by parts is more complicated.
\vskip 5mm
\noindent{\bf\emph{\underline{Question}:}} Can you think of a criteria for choosing $u(x)$ and $v'(x)$?
\vskip 2cm
\begin{itemize}
	\item[$\bullet$] We want our new integral to be simpler (or at the least not more complicated)! As a general rule, if there is a polynomial term such as a $x^k$, we will choose $u(x)=x^k$ because its derivative is simpler:
\[\ds\frac{d}{dx}\left[x^k\right]=kx^{k-1}.\]
\end{itemize}
\vspace*{5mm}
	\end{boxsolution}
\vskip 5mm

\fi


Later, we will see examples where we don't make this choice. But first, try this next example on your own.
\vskip 5mm


%%%%%%%%%%%%%%%%%%%%%%%%%%%%%%%%%%%%%%%%%%%%%%%%%%%%%%%%%
%%%%%%%%%%%%%%%%%%%%%%%%%%%%%%%%%%%%%%%%%%%%%%%%%%%%%%%%%


\begin{example}

Evaluate the integral $\ds\int 2x\sin{x}\,dx $
\vskip 5mm
\noindent{\bf\emph{\underline{Workspace}:}}

\vfill\eject

\ifnum\longform=1
	\begin{boxsolution}
\vspace*{5mm}
Choose $u(x)=2x$, and $v'(x)=\sin{x}$. So that $u'(x)=\frac{d}{dx}[2x]=2$ and $v(x)= \int\sin{x}\,dx = -\cos{x}$ \\\\ Then we have that: 
\beq
\int x\sin{x}\,dx &=&2x\big(-\cos(x)\big)-\int-2\cos{x}\,dx\\
&=&-2x\cos{x}+2\int\cos{x}\,dx\\
&=&-2x\cos{x}+2\sin{x}+C\\
\eeq
So the integral evaluates to: $$\int x\sin{x}\,dx=-2x\cos{x}+2\sin{x}+C$$
\vspace*{5mm}
	\end{boxsolution}
\vskip 5mm

\fi

\end{example}

Let's try another example on your own.


%%%%%%%%%%%%%%%%%%%%%%%%%%%%%%%%%%%%%%%%%%%%%%%%%%%%%%%%%
%%%%%%%%%%%%%%%%%%%%%%%%%%%%%%%%%%%%%%%%%%%%%%%%%%%%%%%%%
\vskip 5mm

\begin{example}

Evaluate the integral using integration by parts $\ds\int x\sqrt{x+1}\,dx$.
\vskip 5mm
\noindent{\bf\emph{\underline{Workspace}:}}

\vfill\eject

\ifnum\longform=1
	\begin{boxsolution}
\vspace*{5mm}
You may notice that you can solve this integral using u-sub from calculus I. You can also use integration by parts. So for the sake of practice, let's use integration by parts to solve this integral. \\\\ 
Choose $u(x)=x$ and $v'(x)=\sqrt{x+1}$. So that $u'(x)=\frac{d}{dx}[x]=1$ and $v(x)=\int\sqrt{x+1}\,dx=\frac{2}{3}(x+1)^{\frac{3}{2}}$ \\\\
So we have that that 
\beq
\int x\sqrt{x+1}\,dx&=&x\cdot\ds\frac{2}{3}(x+1)^{\frac{3}{2}}-\int\ds\frac{2}{3}(x+1)^{\frac{3}{2}}\cdot1\,dx\\
&=&\frac{2}{3}x(x+1)^{\frac{3}{2}}-\ds\frac{2}{3}\cdot\frac{2}{5}(x+1)^{\frac{5}{2}}\\
\eeq
So the integral evaluates to
\[\int x\sqrt{x+1}\,dx=\ds\frac{2}{3}x(x+1)^{\frac{3}{2}}-\frac{4}{15}(x+1)^{\frac{5}{2}}+C.\]
\vspace*{5mm}
	\end{boxsolution}
\vskip 5mm

\fi

\end{example}

Let’s have you try another example on your own.


%%%%%%%%%%%%%%%%%%%%%%%%%%%%%%%%%%%%%%%%%%%%%%%%%%%%%%%%%
%%%%%%%%%%%%%%%%%%%%%%%%%%%%%%%%%%%%%%%%%%%%%%%%%%%%%%%%%
\vskip 5mm

\begin{example}

Evaluate the integral using integration by parts $\ds\int x\ln(x)\,dx$.
\vskip 5mm
\noindent{\bf\emph{\underline{Workspace}:}}

\vfill\eject

\ifnum\longform=1
	\begin{boxsolution}
\vspace*{5mm}
Suppose we choose $u(x)=x$ and $v'(x)=\ln(x)$. Then $u'(x)=\frac{d}{dx}[x]=1$. But we don't know $\int\ln(x)\,dx$. So it looks like we have no choice but to try $u(x)=\ln(x)$ and $v'(x)=x$. Then $u'(x)=\frac{d}{dx}[\ln(x)]=1/x$ and $v(x)=\int x\,dx=x^2/2$. Then we have  
\beq
\int x\ln(x)\,dx&=&\ds\frac{1}{2}x^2\ln(x)-\ds\frac{1}{2}\int\ds\frac{1}{x}\cdot x\,dx\\
&=&\ds\frac{1}{2}x^2\ln(x)-\ds\frac{1}{2}\int 1\,dx\\
\eeq
So the integral evaluates to
\[\int x\ln(x)\,dx=\ds\frac{1}{2}x^2\ln(x)-\frac{x}{2}+C.\]
\vspace*{5mm}
	\end{boxsolution}
\vskip 5mm

\fi

\end{example}

%%%%%%%%%%%%%%%%%%%%%%%%%%%%%%%%%%%%%%%%%%%%%%%%%%%%%%%%%
%%%%%%%%%%%%%%%%%%%%%%%%%%%%%%%%%%%%%%%%%%%%%%%%%%%%%%%%%
\vskip 2mm
So if we our unable to integrate our choice of $v'(x)$ then we have to make another choice. Some people prefer a helpful mnemonic device to make these choice. Here is a popular one for choosing $u(x)$.
\vskip 1cm
{\large\bf\emph{\underline{Choosing a $u(x)$}:}} We begin integration by parts problems by choosing a u. A helpful mnemonic device to help you choose a u is "LIATE". This does not always work, but is a good way to start a problem if you're stuck. Choose $u(x)$ by which terms comes first: \\\\
L: Logarithmic functions \\
I: Inverse trigonometric functions \\
A: Algebraic functions (things like $x^2, x^3$) \\
T: Trigonometric functions (such as $\sin{x}, \cos{x}, \tan{x}$) \\
E: Exponential functions (such as $e^x, 3^x)$\\
\vskip 5mm
Let’s have you try another example on your own.
\vskip 5mm

%%%%%%%%%%%%%%%%%%%%%%%%%%%%%%%%%%%%%%%%%%%%%%%%%%%%%%%%%
%%%%%%%%%%%%%%%%%%%%%%%%%%%%%%%%%%%%%%%%%%%%%%%%%%%%%%%%%


\begin{example}

Evaluate the integral $\ds\int x^2\sin{(10x)}\,dx$.
\vskip 5mm
\noindent{\bf\emph{\underline{Workspace}:}}

\vfill\eject

\ifnum\longform=1

\noindent{\bf\emph{\underline{Workspace Cont}:}}

\vfill\eject

	\begin{boxsolution}
\vspace*{5mm}
Choose $u(x)=x^2$ and $v'(x)=\sin{(10x)}$. So that $u'(x)=\frac{d}{dx}[x^2]=2x$ and $v(x)=\int \sin{(10x)}\,dx=-\frac{1}{10}\cos{(10x)}$. \\\\
It follows that 
\beq
\ds\int x^2\sin{(10x)}\,dx&=&x^2\Big(-\ds\frac{1}{10}\cos{(10x)}\Big)-\int-\ds\frac{1}{10}\cos{(10x)}\cdot2x\,dx.\\
&=&-\frac{x^2}{10}\cos{(10x)}+\frac{1}{5}\int x\cos(10x)\,dx\\
\eeq
\noindent{\bf\emph{\underline{Observation}:}} Our new integral produced by parts is indeed simpler. (How?) So we now choose a strategy to evaluate the second integral. Here it looks like integration by parts is needed to solve $\int x\cos{(10x)} dx$.
\vskip 2mm
To solve this integral choose $u(x)=x$, and $v'(x)=\cos{(10x)}$. So that $u'(x)=1$ and $v(x)=\frac{1}{10}\sin{(10x)}$. It follows that 
\[\ds\int x\cos{(10x)}\,dx=\frac{1}{10}x\sin{(10x)}-\frac{1}{10}\int \sin{(10x)}\,dx = \frac{x}{10}\sin{(10x)}+\frac{1}{10}\cos{(10x)}.\]
Putting it altogether we have that
\beq
\ds\int x^2\sin{(10x)}\,dx&=&-\frac{x^2}{10}\cos{(10x)}+\frac{1}{5}\left(\frac{x}{10}\sin{(10x)}+\frac{1}{10}\cos{(10x)}\right)+C\\
&=&-\frac{x^2}{10}\cos{(10x)}+\frac{x}{50}\sin{(10x)}+\frac{1}{500}\cos{(10x)}+C\\
\eeq
\vspace*{5mm}
	\end{boxsolution}
\vskip 5mm

\fi

\end{example}

%%%%%%%%%%%%%%%%%%%%%%%%%%%%%%%%%%%%%%%%%%%%%%%%%%%%%%%%%
%%%%%%%%%%%%%%%%%%%%%%%%%%%%%%%%%%%%%%%%%%%%%%%%%%%%%%%%%
\vskip 5mm

\section*{The Tabular Method}


\begin{example}

Evaluate the integral $\ds\int x^2e^{3x}\,dx$
\vskip 5mm
\noindent{\bf\emph{\underline{Workspace}:}}
\vfill\eject

\ifnum\longform=1

\noindent{\bf\emph{\underline{Workspace Cont}:}}

\vfill\eject

	\begin{boxsolution}
\vspace*{5mm}
We have
\beq
\ds\int x^2e^{3x}\,dx&=&\ds\frac{1}{3}x^2e^{3x}-\ds\frac{2}{3}\ds\int xe^{3x}\,dx\\
&=&\ds\frac{1}{3}x^2e^{3x}-\ds\frac{2}{3}\left(\ds\frac{1}{3}xe^{3x}-\ds\frac{1}{3}\ds\int e^{3x}\,dx\right)\\
&=&\ds\frac{1}{3}x^2e^{3x}-\ds\frac{2}{9}xe^{3x}-\ds\frac{2}{27}e^{3x}+C\\
\eeq
We will keep track of the previous steps using a table.

\vfill\eject

\vskip 5mm
\[\begin{array}{c|ccccc}
\text{Step\;\#}&u&&v'&&\text{Results}\\
\hline   
&&&&&\\
1&x^2&&e^{3x}&&\\
&&&&&\\
&&\searrow\,\text{\tiny $\times$}&&\xrightarrow[\text{\textcolor{red}{Result after one step}}]{}&\ds\frac{1}{3}x^2e^{3x}-\ds\int2x\cdot\ds\frac{1}{3}e^{3x}\,dx\\
&&&&&\\
2&2x&\xleftrightarrow[\text{\textcolor{red}{Integrate}}]{\text{\huge\textcolor{blue}{-}}}&\ds\frac{1}{3}e^{3x}&&\\
&&&&&\\
&&\searrow_{\;\text{\huge\textcolor{blue}{-}}}\text{\tiny $\times$}&&\xrightarrow[\text{\textcolor{red}{Result after two steps}}]{}&\ds\frac{1}{3}x^2e^{3x}-\ds\frac{2}{9}x e^{3x}+\ds\int 2\cdot\ds\frac{1}{9}e^{3x}\,dx\\
&&&&&\\
3&2&\xleftrightarrow[\text{\textcolor{red}{Integrate}}]{}&\ds\frac{1}{9}e^{3x}&&\\
&&&&&\\
&&\searrow\,\text{\tiny $\times$}&&\xrightarrow[\text{\textcolor{red}{Result after three steps}}]{}&\ds\frac{1}{3}x^2e^{3x}-\ds\frac{2}{9}x e^{3x}+\ds\frac{2}{27}e^{3x}+C\\
&&&&&\\
&0&\xleftrightarrow[\text{\textcolor{red}{Integrate}}]{\text{\huge\textcolor{blue}{-}}}&\ds\frac{1}{27}e^{3x}&\ds\int 0\,dx=C&\\
\end{array}\]
\vskip 5mm
Let's put this in a more compact and readable table.
\[\vdots\]
\vspace*{5mm}
	\end{boxsolution}
	
\vskip 5mm

	\begin{boxsolutioncont}
\vspace*{5mm}
\[\vdots\]
\vskip 5mm
\[\begin{array}{ccccc}
u&&v'&\;\;&+/-\\
\hline   
&&&&\\
x^2&&e^{3x}&&+\\
&\searrow\,\text{\tiny $+$}&&&\\
2x&&\ds\frac{1}{3}e^{3x}&&-\\
&\searrow\,\text{\tiny $-$}&&\\
2&&\ds\frac{1}{9}e^{3x}&&+\\
&\searrow\,\text{\tiny $+$}&&&\\
0&&\ds\frac{1}{27}e^{3x}&&-\\
\end{array}\]
\vskip 5mm
We build the table by differentiating down the $u$-column until we reach zero, and integrating down the $v'$ column. Then reading the products by the arrows and changing signs every other term we arrive at the answer.
\vskip 2mm
\[\ds\int x^2e^{3x}\,dx=\ds\frac{1}{3}x^2e^{3x}-\ds\frac{2}{9}x e^{3x}+\ds\frac{2}{27}e^{3x}+C\]
\vspace*{5mm}
	\end{boxsolutioncont}
\vskip 5mm

\fi

\end{example}

\vskip 5mm

%%%%%%%%%%%%%%%%%%%%%%%%%%%%%%%%%%%%%%%%%%%%%%%%%%%%%%%%%
%%%%%%%%%%%%%%%%%%%%%%%%%%%%%%%%%%%%%%%%%%%%%%%%%%%%%%%%%


\begin{example}

Evaluate the integral $\ds\int x^4\cos(5x)\,dx$ using the tabular method.
\vskip 5mm
\noindent{\bf\emph{\underline{Workspace}:}}

\vfill\eject

\ifnum\longform=1

\noindent{\bf\emph{\underline{Workspace Cont}:}}

\vfill\eject

	\begin{boxsolution}
\vspace*{5mm}
We choose $u(x)=x^4$ and $v'(x)=\cos(5x)$
\vskip 5mm
Now complete the table.
\vskip 5mm
\[\begin{array}{ccccc}
u&&v'&\;\;&+/-\\\\
\hline   
&&&&\\
x^4&&\cos(5x)&&+\\
&\searrow\,\text{\tiny $+$}&&&\\
4x^3&&\ds\frac{1}{5}\sin(5x)&&-\\
&\searrow\,\text{\tiny $-$}&&\\
12x^2&&-\ds\frac{1}{25}\cos(5x)&&+\\
&\searrow\,\text{\tiny $+$}&&&\\
24x&&-\ds\frac{1}{125}\sin(5x)&&-\\
&\searrow\,\text{\tiny $-$}&&&\\
24&&\ds\frac{1}{625}\cos(5x)&&+\\
&\searrow\,\text{\tiny $+$}&&&\\
0&&\ds\frac{1}{3125}\sin(5x)&&-\\
\end{array}\]
\vskip 5mm
We can now read off the answer.
\vskip 2mm
\[\ds\int x^4\cos(5x)\,dx=\ds\frac{1}{5}x^4\sin(5x)+\ds\frac{4}{25}x^3\cos(5x)-\ds\frac{12}{125}x^2\sin(5x)-\ds\frac{24}{625}x\cos(5x)+\ds\frac{24}{3125}\sin(5x)+C\]
\vspace*{5mm}
	\end{boxsolution}
\vskip 5mm

\fi

\end{example}

\vskip 5mm

\section*{Definite Integrals Using Integration By Parts}

We have the following formula.
\[\ds\int^{x=b}_{x=a}u(x)v'(x)\,dx=u(x)v(x)\ds\Bigg|^{x=b}_{x=a}-\ds\int^{x=b}_{x=a}v(x)u'(x)\,dx\]
The notation in this formula often leads to some confusion. So let's consider an example.

\vfill\eject

%%%%%%%%%%%%%%%%%%%%%%%%%%%%%%%%%%%%%%%%%%%%%%%%%%%%%%%%%
%%%%%%%%%%%%%%%%%%%%%%%%%%%%%%%%%%%%%%%%%%%%%%%%%%%%%%%%%

\begin{example}

Evaluate the integral $\ds\int^{x=\pi/3}_{x=0} 2x\sin{x}\,dx$
\vskip 5mm
\noindent{\bf\emph{\underline{Workspace}:}}
\vfill\eject

\ifnum\longform=1
	\begin{boxsolution}
\vspace*{5mm}
We have already found the antiderivative earlier. We'll use this. 
\[\int x\sin{x}\,dx=-2x\cos{x}+2\sin{x}+C\]
Then the definite integral evaluates to: 
\beq
\ds\int^{x=\pi/3}_{x=0} 2x\sin{x}\,dx&=&-2x\cos{x}+2\sin{x}\,\ds\Bigg|^{x=\pi/3}_{x=0}\\
\\
&=&\left(-2\cdot\ds\frac{\pi}{3}\cdot\cos(\pi/3)+2\sin(\pi/3)\right)-\Big(0+0\Big)\\
\\
&=&-\ds\frac{\pi}{3}+\sqrt{3}
\eeq
\vspace*{5mm}
	\end{boxsolution}
\vskip 5mm

\fi

\end{example}

Let's try another example.


%%%%%%%%%%%%%%%%%%%%%%%%%%%%%%%%%%%%%%%%%%%%%%%%%%%%%%%%%
%%%%%%%%%%%%%%%%%%%%%%%%%%%%%%%%%%%%%%%%%%%%%%%%%%%%%%%%%


\begin{example}

Evaluate the integral $\ds\int^{x=\pi/12}_{x=0}\ds\frac{x^2}{\sec(4x)}\,dx$
\vskip 3mm
\hint Can you rewrite $1/\sec(4x)$?
\vskip 5mm
\noindent{\bf\emph{\underline{Workspace}:}}

\vfill\eject

\ifnum\longform=1
	\begin{boxsolution}
\vspace*{5mm}
We find the antiderivative by parts first seeing that $1/\sec(4x)=\cos(4x)$. 

\[\begin{array}{ccccc}
u&&v'&\;\;&+/-\\
\hline   
&&&&\\
x^2&&\cos(4x)&&+\\
&\searrow\,\text{\tiny $+$}&&&\\
2x&&\ds\frac{1}{4}\sin(4x)&&-\\
&\searrow\,\text{\tiny $-$}&&\\
2&&-\ds\frac{1}{16}\cos(4x)&&+\\
&\searrow\,\text{\tiny $+$}&&&\\
0&&-\ds\frac{1}{64}\sin(4x)&&-\\
\end{array}\]

\vskip 2mm

Now we have
\vskip 2mm
\beq
\ds\int^{x=\pi/12}_{x=0}\ds\frac{x^2}{\sec(4x)}\,dx&=&\ds\frac{x^2}{4}\sin(4x)+\ds\frac{x}{8}\cos(4x)-\ds\frac{1}{32}\sin(4x)\ds\Bigg|^{x=\pi/12}_{x=0}\\
\\
&=&\left[\;\ds\frac{\pi^2}{576}\cdot\sin\left(\ds\frac{\pi}{3}\right)+\ds\frac{\pi}{96}\cdot\cos\left(\ds\frac{\pi}{3}\right)-\ds\frac{1}{32}\sin\left(\ds\frac{\pi}{3}\right)\;\right]-\left[0+0-\ds\frac{1}{32}\sin(0)\right]\\
\\
&=&\left(\ds\frac{\pi^2}{576}\right)\cdot\left(\ds\frac{\sqrt{3}}{2}\right)+\left(\ds\frac{\pi}{96}\right)\cdot\left(\ds\frac{1}{2}\right)-\left(\ds\frac{1}{32}\right)\cdot\left(\ds\frac{\sqrt{3}}{2}\right)\\
\\
&=&\ds\frac{\pi^2\sqrt{3}}{1152}+\ds\frac{\pi}{192}-\ds\frac{\sqrt{3}}{64}\approx 0.004
\eeq
\vspace*{5mm}
	\end{boxsolution}
\vskip 5mm

\fi

\end{example}


%%%%%%%%%%%%%%%%%%%%%%%%%%%%%%%%%%%%%%%%%%%%%%%%%%%%%%%%%
%%%%%%%%%%%%%%%%%%%%%%%%%%%%%%%%%%%%%%%%%%%%%%%%%%%%%%%%%
\vskip 2mm

\section*{Some Interesting and Important Examples}

\begin{example}

Evaluate the integral $\ds\int e^x\sin{x}\,dx$ 
\vskip 5mm
\noindent{\bf\emph{\underline{Workspace}:}}

\vfill\eject

\ifnum\longform=1

\noindent{\bf\emph{\underline{Workspace Cont.}:}}

\vfill\eject

	\begin{boxsolution}
\vspace*{5mm}
Choose $u(x)=e^x$, and $v'(x)=\sin{x}$. So that $u'(x)=e^x$ and $v(x)=-\cos{x}$. Then we have
\[\int e^x\sin{x}\,dx=-e^x\cos{x}+\int e^xcos(x)\,dx.\]
We will need to use integration by parts to solve $\int e^x\cos{x}\,dx$. So choose $u(x)=e^x$ and $v'(x)=\cos{x}$. So that $u'(x)=e^x$ and $v(x)=\sin{x}$. It follows that \[\int e^x\cos{x}\,dx=e^x\sin{x}-\int e^x\sin{x}\,dx.\] 
Putting it together we have
\[\int e^x\sin{x}\,dx=-e^x\cos{x}+e^x\sin{x}-\int e^x\sin{x}\,dx.\] 
\vskip 5mm

Or if you prefer using the tabular method:
\[\begin{array}{ccccc}
u&&v'&\;\;&+/-\\\\
\hline   
&&&&\\
e^x&&\sin(x)&&+\\
&\searrow\,\text{\tiny $+$}&&&\\
e^x&&-\cos(x)&&-\\
&\searrow\,\text{\tiny $-$}&&\\
\\
e^x&\xleftrightarrow[\text{\textcolor{red}{Integrate}}]{+}&-\sin(x)&&+\\
\end{array}\]
Remember we can exit the tabular algorithm whenever we want (such as the third row) using integration. Then reading off the table we have
\[\int e^x\sin{x}\,dx=-e^x\cos{x}+e^x\sin{x}-\int e^x\sin{x}\,dx.\] 

\vskip 2mm

It may appear that we have simply gone in a circle. But we can add $\int e^x\sin{x}\,dx$ on each side of the equation to obtain:
\[2\int e^x\sin{x}\,dx=-e^x\cos{x}+e^x\sin{x}+C.\]

Therefore
\vskip 2mm
\[\int e^x\sin{x}\,dx=\frac{e^x\sin{x}-e^x\cos{x}}{2}+C.\]
\vspace*{5mm}
	\end{boxsolution}
\vskip 5mm

\fi

\end{example}



%%%%%%%%%%%%%%%%%%%%%%%%%%%%%%%%%%%%%%%%%%%%%%%%%%%%%%%%%
%%%%%%%%%%%%%%%%%%%%%%%%%%%%%%%%%%%%%%%%%%%%%%%%%%%%%%%%%

\ifnum\longform=1
\vfill\eject

\fi

\begin{example}

Evaluate the integral $\ds\int\ln(x)\,dx$.
\vskip 5mm
\noindent{\bf\emph{\underline{Workspace}:}}
\vfill\eject

\ifnum\longform=1
	\begin{boxsolution}
\vspace*{5mm}
We can write $\ln(x)=1\cdot\ln(x)$ and then choose $u(x)=\ln(x)$ and $v'(x)=1$ giving us $u'(x)=1/x$ and $v(x)=x$.
\[\ds\int\ln(x)\;dx=x\ln(x)-\ds\int 1\,dx=x\ln(x)-x+C.\]
\vspace*{5mm}
	\end{boxsolution}
\vskip 5mm

\fi

\end{example}

\vskip 5mm
Now see if you can use the same idea on another example.


%%%%%%%%%%%%%%%%%%%%%%%%%%%%%%%%%%%%%%%%%%%%%%%%%%%%%%%%%
%%%%%%%%%%%%%%%%%%%%%%%%%%%%%%%%%%%%%%%%%%%%%%%%%%%%%%%%%
\vskip 1cm

\begin{example}

Evaluate the integral $\ds\int\tan^{-1}(x)\,dx$.
\vskip 5mm
\noindent{\bf\emph{\underline{Workspace}:}}
\vfill\eject

\ifnum\longform=1
	\begin{boxsolution}
\vspace*{5mm}
We can write $\tan^{-1}(x)=1\cdot\tan^{-1}(x)$ and then choose $u(x)=\tan^{-1}(x)$ and $v'(x)=1$ giving us $u'(x)=1/(x^2+1)$ and $v(x)=x$.
\[\ds\int\tan^{-1}(x)\;dx=x\tan^{-1}(x)-\ds\int\ds\frac{x}{x^2+1}\,dx=x\tan^{-1}(x)-\ds\frac{1}{2}\ln(x^2+1)+C.\]
\vspace*{5mm}
	\end{boxsolution}
\vskip 5mm

\fi

\end{example}

\vskip 1cm

The idea is if we are integrating a function where we know its derivative but not its antiderivative sometimes mutliplying the function by 1 and then using integration by parts will be helpful. But not always. The following antiderivative does NOT have an elementary solution. None can be found no matter the technique tried.
\[\ds\int e^{-x^2}\,dx.\]
It should be enlightening to realize that not all mathematical problems have elementary solution techniques. But integration by parts allows to integrate many more functions than we could before.
\vskip 1cm

%%%%%%%%%%%%%%%%%%%%%%%%%%%%%%%%%%%%%%%%%%%%%%%%%%%%%%%%%
%%%%%%%%%%%%%%%%%%%%%%%%%%%%%%%%%%%%%%%%%%%%%%%%%%%%%%%%%
%%%%%%%%%%%%%%%%%%%%%%%%%%%%%%%%%%%%%%%%%%%%%%%%%%%%%%%%%
%%%%%%%%%%%%%%%%%%%%%%%%%%%%%%%%%%%%%%%%%%%%%%%%%%%%%%%%%

\ifnum\longform=1
\vskip 1cm
\hrule
\begin{center}
{\bf Please let me know if you have any questions, comments, or corrections!}
\end{center}	

\fi

%%%%%%%%%%%%%%%%%%%%%%%%%%%%%%%%%%%%%%%%%%%%%%%%%%%%%%
\end{document}
%%%%%%%%%%%%%%%%%%%%%%%%%%%%%%%%%%%%%%%%%%%%%%%%%%%%%%
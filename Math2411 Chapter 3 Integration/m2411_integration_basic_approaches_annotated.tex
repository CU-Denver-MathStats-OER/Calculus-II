\documentclass[11pt]{article}
\usepackage[suffix=Solutions]{teaching-header}

\def\classnum{2411}
\def\classtitle{Calculus II}
\def\classtitleshort{Calc 2}
\def\classsec{H01}
\def\instructor{Dr. Rostermundt}
\def\classterm{Spring 2025}


%%%%%%%%%%%%%%%%%%%%%%%%%%%%%%%%%%%%%%%%%%%%%%%%%%%%%%%%%%%%%%%%%%%%%%%%%%%%
%%%%%%%%%%%%%%%%%%%%%%%%%%%%%%%%%%%%%%%%%%%%%%%%%%%%%%%%%%%%%%%%%%%%%%%%%%%%

%This is defined in the teaching-header style file
%\ifnum\printsol=0 (when no solutions printed)
%Do something
%	\else  (when solutions are printed)
%Do something else
%\fi


% Package and setting included in teachin-header style file
%\RequirePackage{amsmath,amsfonts,amssymb,amsthm,graphicx, pgfplots, tcolorbox, xcolor,latexsym,color,verbatim,float,xcolor,setspace}
%%tikzsymbols
%
%\RequirePackage{enumerate}
%\RequirePackage{multicol}
%\RequirePackage{tikz}
%\RequirePackage{cancel}
%\usetikzlibrary{shapes.geometric}
%\usetikzlibrary{calc, positioning, arrows}
%\RequirePackage[margin=1in,letterpaper]{geometry}
%\RequirePackage[colorlinks=true,allcolors=blue]{hyperref}
%\usepackage[final]{pdfpages}
%%\usepackage{capt-of}
%
%
%\setlength{\textheight}{9in}
%\setlength{\textwidth}{6.5in}
%\addtolength{\topmargin}{0cm}
%%\addtolength{\oddsidemargin}{0cm}
%\parindent=0in
%\parskip=.35em
%\singlespacing
%%\pagestyle{empty}  % remove page numbers

%Add captions without being in figure environment
%\captioof{figure}{\text}\label[fig:]
\usepackage{capt-of}
\usepackage{mathtools}

\vfuzz2pt % Don't report over-full v-boxes if over-edge is small
\hfuzz2pt % Don't report over-full h-boxes if over-edge is small


%%%%%%%%%%%%%%%%%%%%%%%%%%%%%%%%%%%%%%%%%%%%%%%%%%%%%%
%%%%%%%%%%%%%%%%%%%%%%%%%%%%%%%%%%%%%%%%%%%%%%%%%%%%%%

\pagestyle{myheadings}

%%%%%%%%%%%%%%%%%%%%%%%%%%%%%%%%%%%%%%%%%%%%%%%%%%%%%%
%%%%%%%%%%%%%%%%%%%%%%%%%%%%%%%%%%%%%%%%%%%%%%%%%%%%%%


%%%%%%%%%%%%%%%%%%%%%%%%%%%%%%%%%%%%%%%%%%%%%%%%%%%%%%
%%%%%%%%%%%%%%%%%%%%%%%%%   Document Body   %%%%%%%%%%
%%%%%%%%%%%%%%%%%%%%%%%%%%%%%%%%%%%%%%%%%%%%%%%%%%%%%%

%Information from classinfo.tex file
%\def\classnum{2411}
%\def\classtitle{Calculus II}
%\def\classtitleshort{Calc 2}
%\def\classsec{001}
%\def\instructor{Rostermundt}
%\def\classterm{Fall 2024}
\def\topic{Integration -- Basic Approaches}
\def\topicshort{Integration Approaches}

	\title{\vspace{-1in}Math\classnum\;-\;\classtitle\\
	%Section\;\classsec\;-\;\classterm\\
	Guided Lecture Notes\\
	\topic}
	\author{University of Colorado Denver / College of Liberal Arts and Sciences}
	\date{Department of Mathematics}

	\markright{Math\classnum\;-\;\classtitleshort, University of Colorado Denver,\;\topicshort}
	

%%%%%%%%%%%%%%%%%%%%%%%%%%%%%%%%%%%%%%%%%%%%%%%%%%%%%%
\begin{document}\maketitle\thispagestyle{empty}
%%%%%%%%%%%%%%%%%%%%%%%%%%%%%%%%%%%%%%%%%%%%%%%%%%%%%%

\hrule

\section*{\topic:}

We are looking at some basic approaches to integration used when a function does not exactly match one of our known formulas. Often the strategy is to use identities and/or manipulate the integrand algebraically so that it fits into a know form. We will work through a number of examples below.

%%%%%%%%%%%%%%%%%%%%%%%%%%%%%%%%%%%%%%%%%%%%%%%%%%%%%%%%%
%%%%%%%%%%%%%%%%%%%%%%%%%%%%%%%%%%%%%%%%%%%%%%%%%%%%%%%%%


\section*{Integration Examples:}

Let’s work an example together.
\vskip 2mm

\begin{example}

Evaluate the integral $\ds\int\cos^2(x)\,dx$
\vskip 5mm
\noindent{\bf\emph{\underline{Workspace}:}}
\vfill\eject

\ifnum\longform=1
	\begin{boxsolution}
\vspace*{5mm}
We integrate as follows.

\beq
\ds\int\cos^2(x)\,dx&=&\ds\frac{1}{2}\ds\int 1+\cos(2x)\,dx\\
&=&\ds\frac{1}{2}\left(\;x+\ds\frac{1}{2}\sin(2x)\;\right)+C\\
&=&\ds\frac{1}{2}x+\ds\frac{1}{4}\sin(2x)+C
\eeq
\vspace*{5mm}
	\end{boxsolution}
\vskip 5mm

\fi

\end{example}

\vskip 1cm


%%%%%%%%%%%%%%%%%%%%%%%%%%%%%%%%%%%%%%%%%%%%%%%%%%%%%%%%%
%%%%%%%%%%%%%%%%%%%%%%%%%%%%%%%%%%%%%%%%%%%%%%%%%%%%%%%%%
\vskip 2mm

\begin{example}

Evaluate the integral $\ds\int\tan(x)\,dx$
\vskip 5mm
\noindent{\bf\emph{\underline{Workspace}:}}
\vfill\eject

\ifnum\longform=1
	\begin{boxsolution}
\vspace*{5mm}
We integrate as follows.

\beq
\ds\int\tan(x)\,dx&=&\ds\int\ds\frac{\sin(x)}{\cos(x)}\,dx\qquad\text{Let }u=\cos(x)\text{ and }du=-\sin(x)\,dx.\\
&=&-\ln\big|\cos(x)\big|+C\\
&=&\ln\big|\sec(x)\big|+C
\eeq
\vspace*{5mm}
	\end{boxsolution}
\vskip 5mm

\fi

\end{example}

\vskip 1cm


%%%%%%%%%%%%%%%%%%%%%%%%%%%%%%%%%%%%%%%%%%%%%%%%%%%%%%%%%
%%%%%%%%%%%%%%%%%%%%%%%%%%%%%%%%%%%%%%%%%%%%%%%%%%%%%%%%%
\vskip 2mm

\begin{example}

Evaluate the integral $\ds\int\sec(x)\,dx$
\vskip 5mm
\noindent{\bf\emph{\underline{Workspace}:}}
\vfill\eject

\ifnum\longform=1
	\begin{boxsolution}
\vspace*{5mm}
We integrate as follows.

\beq
\ds\int\sec(x)\,dx&=&\ds\int\sec(x)\ds\frac{\sec(x)+\tan(x)}{\sec(x)+\tan(x)}\,dx\\
&=&\ds\int\ds\frac{\sec^2(x)+\sec(x)\tan(x)}{\sec(x)+\tan(x)}\,dx\qquad\text{Let }u=\sec(x)+\tan(x),\;du=\sec^2(x)+\sec(x)\tan(x)dx\\
&=&\ln\big|\sec(x)+\tan(x)\big|+C
\eeq
\vspace*{5mm}
	\end{boxsolution}
\vskip 5mm

\fi

\end{example}

\vskip 1cm


%%%%%%%%%%%%%%%%%%%%%%%%%%%%%%%%%%%%%%%%%%%%%%%%%%%%%%%%%
%%%%%%%%%%%%%%%%%%%%%%%%%%%%%%%%%%%%%%%%%%%%%%%%%%%%%%%%%
\vskip 2mm

\begin{example}

Evaluate the integral $\ds\int\ds\frac{1}{e^{x}+e^{-x}}\,dx$
\vskip 5mm
\noindent{\bf\emph{\underline{Workspace}:}}
\vfill\eject

\ifnum\longform=1
	\begin{boxsolution}
\vspace*{5mm}
We integrate as follows.

\beq
\ds\int\ds\frac{1}{e^{x}+e^{-x}}\,dx&=&\ds\int\ds\frac{e^x}{e^x}\cdot\ds\frac{1}{e^{x}+e^{-x}}\,dx\\
&=&\ds\int\ds\frac{e^x}{e^{2x}+1}\,dx\qquad\text{Let }u=e^{x}\text{ and }du=e^x\,dx\\
&=&\ds\frac{1}{u^2+1}\,du\\
&=&\tan^{-1}(u)+C\\
&=&\tan^{-1}\left(e^x\right)+C
\eeq
\vspace*{5mm}
	\end{boxsolution}
\vskip 5mm

\fi

\end{example}

\vskip 1cm


%%%%%%%%%%%%%%%%%%%%%%%%%%%%%%%%%%%%%%%%%%%%%%%%%%%%%%%%%
%%%%%%%%%%%%%%%%%%%%%%%%%%%%%%%%%%%%%%%%%%%%%%%%%%%%%%%%%
\vskip 2mm

\begin{example}

Evaluate the integral $\ds\int\ds\frac{1}{x^2+4x+5}\,dx$
\vskip 5mm
\noindent{\bf\emph{\underline{Workspace}:}}
\vfill\eject

\ifnum\longform=1
	\begin{boxsolution}
\vspace*{5mm}
We integrate as follows.

\beq
\ds\int\ds\frac{1}{x^2+4x+5}\,dx&=&\ds\int\ds\frac{1}{x^2+4x+4+1}\,dx\\
&=&\ds\int\ds\frac{1}{(x+2)^2+1}\,dx\qquad\text{Let }u=x+2\text{ and }du=dx\\
&=&\ds\int\ds\frac{1}{u^2+1}\,du\\
&=&\tan^{-1}(u)+C\\
&=&\tan^{-1}(x+2)+C
\eeq
\vspace*{5mm}
	\end{boxsolution}
\vskip 5mm

\fi

\end{example}

\vskip 1cm



%%%%%%%%%%%%%%%%%%%%%%%%%%%%%%%%%%%%%%%%%%%%%%%%%%%%%%%%%
%%%%%%%%%%%%%%%%%%%%%%%%%%%%%%%%%%%%%%%%%%%%%%%%%%%%%%%%%
%%%%%%%%%%%%%%%%%%%%%%%%%%%%%%%%%%%%%%%%%%%%%%%%%%%%%%%%%
%%%%%%%%%%%%%%%%%%%%%%%%%%%%%%%%%%%%%%%%%%%%%%%%%%%%%%%%%

\ifnum\longform=1
\vskip 1cm
\hrule
\begin{center}
{\bf Please let me know if you have any questions, comments, or corrections!}
\end{center}	

\fi



%%%%%%%%%%%%%%%%%%%%%%%%%%%%%%%%%%%%%%%%%%%%%%%%%%%%%%
\end{document}
%%%%%%%%%%%%%%%%%%%%%%%%%%%%%%%%%%%%%%%%%%%%%%%%%%%%%%
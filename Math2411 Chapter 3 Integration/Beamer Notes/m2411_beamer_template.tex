\documentclass{beamer}
\usetheme{Warsaw}

%\mode<presentation>
%{
%  \usetheme{Pittsburgh}
%  \setbeamercovered{transparent}
%}

\usepackage[english]{babel}
%\usepackage[latin1]{inputenc}
%\usepackage{times}
%\usepackage[normalem]{ulem}
%\usepackage{listings}

\setbeamertemplate{footline}[frame number]


\usepackage{amsmath,amssymb,amsthm,latexsym,graphicx,hyperref,bbm,rotating}



%%%%%%%%%%%%%%%%%%%%%%%%%%%%%%%%%%%%%%%%%%%%%%%%%%%%%%%%%%%%%%%%%%%%%%%%%
%%%%%%%%%%%%%%%%%%%%%%%%%%%%%% Defined Fonts %%%%%%%%%%%%%%%%%%%%%%%%%%%%
%%%%%%%%%%%%%%%%%%%%%%%%%%%%%%%%%%%%%%%%%%%%%%%%%%%%%%%%%%%%%%%%%%%%%%%%%


\font\minihelv=phvr at 6pt
\font\helv=phvr at 10pt
\font\medhelv=phvr at 16pt
\font\bighelv=phvr at 20pt
\font\hugehelv=phvr at 36pt
\font\mybigfont=phvr at 16pt
\font\mymediumfont=phvr at 14pt
\font\mediumhelv=phvr at 14pt
\font\mybfit=ptmbi at 12pt


%\def\classnum{2411}
%\def\classtitle{Calculus II}
%\def\classtitleshort{Calc 2}
%\def\classsec{H01}
%\def\classterm{Spring 2025}
%\def\instructor{Dr. Robert Rostermundt}


\def\classnum{2411}
\def\classtitle{Calculus II}
\def\classtitleshort{Calc 2}
\def\classsec{H01}
\def\instructor{Dr. Rostermundt}
\def\classterm{Spring 2025}
  % Class Specific Information for title header
\input{latexmacros}



%%%%%%%%%%%%%%%%%%%%%%%%%%%%%%%%%%%%%%%%%%%%%%%%%%%%%%

%%%%%%%%%%%%%%%%%%%%%%%%%%%%%%%%%%%%%%%%%%%%%%%%%%%%%%


\title{Taylor Polynomial Graphs and Approximations}
\subtitle{Math\classnum\;\classtitle\;Lecture - \classterm}
\author{Instructor: \instructor}
\institute{University of Colorado Denver}
\date{}
%\begin{figure}
%\includegraphics[trim= 0cm 0cm 0cm 0cm, clip=true, scale=0.35]{cycloid_regular_pendulum.jpg}
%\end{figure}}


\newcommand\Mydiv[2]{%
$\strut#1$\kern.25em\smash{\raise.3ex\hbox{$\big)$}}$\mkern-8mu
        \overline{\enspace\strut#2}$}

%%%%%%%%%%%%%%%%%%%%%%%%%%%%%%%%%%%%%%%%%%%%%%%%%%%%%%
\begin{document}
%%%%%%%%%%%%%%%%%%%%%%%%%%%%%%%%%%%%%%%%%%%%%%%%%%%%%%


\frame{\titlepage}


%%%%%%%%%%%%%%%%%%%%%%%%%%%%%%%%%%%%%%%%%%%%%%%%%%%%%%%%%%%%%%%%%
%%%%%%%%%%%%%%%%%%%%%%%%%%%%%%%%%%%%%%%%%%%%%%%%%%%%%%%%%%%%%%%%%

\renewcommand{\S}{\mathbb{S}} % For defining a set S 

%\begin{tabular}{ccc}
%&&$x$\\
%&\multicolumn{2}{c}{\Mydiv{x^2+1}{x^3+6}}\\
%&&\qquad\,\underline{$x^3+1$}\\
%\end{tabular}


%%%%%%%%%%%%%%%%%%%%%%%%%%%%%%%%%%%%%%%%%%%%%%%%%%%%%%%%%%%%%%%%%
%%%%%%%%%%%%%%%%%%%%%%%%%%%%%%%%%%%%%%%%%%%%%%%%%%%%%%%%%%%%%%%%%


%%%%%%%%%%%%%%%%%%%%%%%%%%%%%%%%%%%%%%%%%%%%%%%%%%%%%%%%%%%%%%%%%
%\section[????]{????}
%%%%%%%%%%%%%%%%%%%%%%%%%%%%%%%%%%%%%%%%%%%%%%%%%%%%%%%%%%%%%%%%%


%%%%%%%%%%%%%%%%%%%%%%%%%%%%%%%%%%%%%%%%%%%%%%%%%%%%%%%%%%%%%%%%%%
%\subsection[]{??????}
%%%%%%%%%%%%%%%%%%%%%%%%%%%%%%%%%%%%%%%%%%%%%%%%%%%%%%%%%%%%%%%%%%



%%%%%%%%%%%%%%%%%%%%%%%%%%%%%%%%%%%%%%%%%%%%%%%%%%%%%%%%%%%%%%%%%%%%%%
\begin{frame}{$f(x)=\sin(x)$}
%%%%%%%%%%%%%%%%%%%%%%%%%%%%%%%%%%%%%%%%%%%%%%%%%%%%%%%%%%%%%%%%%%%%%%


\begin{center}
\includegraphics[scale=0.4]{sinxplot.pdf}
\end{center}
\vskip 5mm
Taylor Polynomial for $f(x)=\sin(x)$ centered at $a=0$.
\beq
p(x)&=&\ds\sum^{n}_{k=0}(-1)^k\ds\frac{x^{2k+1}}{(2k+1)!}\\
&=&x-\ds\frac{x^3}{3!}+\ds\frac{x^5}{5!}+\cdots+(-1)^n\ds\frac{x^{2n+1}}{(2n+1)!}
\eeq




%%%%%%%%%%%%%%%%%%%%%%%%%%%%%%%%%%%%%%%%%%%%%%%%%%%%%%%%%%%%%%%%%%%%%%
\end{frame}
%%%%%%%%%%%%%%%%%%%%%%%%%%%%%%%%%%%%%%%%%%%%%%%%%%%%%%%%%%%%%%%%%%%%%%


%%%%%%%%%%%%%%%%%%%%%%%%%%%%%%%%%%%%%%%%%%%%%%%%%%%%%%%%%%%%%%%%%%%%%%
\begin{frame}{$f(x)=\sin(x)$}
%%%%%%%%%%%%%%%%%%%%%%%%%%%%%%%%%%%%%%%%%%%%%%%%%%%%%%%%%%%%%%%%%%%%%%

\begin{itemize}
\item Taylor Polynomial of degree 1 for $f(x)=\sin(x)$ is $p(x)=x$.
\item Taylor polynomial approximation of $\sin(1)$ equals $p(1)=1$.
\item True function value equals $\sin(1)=0.841471$.
\item Approximation error equals $p(1)-\sin(1)=0.158529$.
\end{itemize}
\vspace*{5mm}
\begin{center}
\includegraphics[scale=0.4]{sinxapproxdeg1.pdf}
\end{center}


%%%%%%%%%%%%%%%%%%%%%%%%%%%%%%%%%%%%%%%%%%%%%%%%%%%%%%%%%%%%%%%%%%%%%%
\end{frame}
%%%%%%%%%%%%%%%%%%%%%%%%%%%%%%%%%%%%%%%%%%%%%%%%%%%%%%%%%%%%%%%%%%%%%%


%%%%%%%%%%%%%%%%%%%%%%%%%%%%%%%%%%%%%%%%%%%%%%%%%%%%%%%%%%%%%%%%%%%%%%
\begin{frame}{$f(x)=\sin(x)$}
%%%%%%%%%%%%%%%%%%%%%%%%%%%%%%%%%%%%%%%%%%%%%%%%%%%%%%%%%%%%%%%%%%%%%%

\begin{itemize}
\item Taylor Polynomial of degree 7 for $f(x)=\sin(x)$ is $p(x)=x-\ds\frac{x^3}{6}+\ds\frac{x^5}{120}-\ds\frac{x^7}{5040}$.
\item Taylor polynomial approximation of $\sin(1)$ equals $p(1)=0.841468$.
\item True function value equals $\sin(1)=0.841471$.
\item Approximation error equals $p(1)-\sin(1)=-2.73084\times 10^{-6}$.
\end{itemize}
\vspace*{5mm}
\begin{center}
\includegraphics[scale=0.4]{sinxapproxdeg7.pdf}
\end{center}


%%%%%%%%%%%%%%%%%%%%%%%%%%%%%%%%%%%%%%%%%%%%%%%%%%%%%%%%%%%%%%%%%%%%%%
\end{frame}
%%%%%%%%%%%%%%%%%%%%%%%%%%%%%%%%%%%%%%%%%%%%%%%%%%%%%%%%%%%%%%%%%%%%%%


%%%%%%%%%%%%%%%%%%%%%%%%%%%%%%%%%%%%%%%%%%%%%%%%%%%%%%%%%%%%%%%%%%%%%%
\begin{frame}{$f(x)=\sin(x)$}
%%%%%%%%%%%%%%%%%%%%%%%%%%%%%%%%%%%%%%%%%%%%%%%%%%%%%%%%%%%%%%%%%%%%%%

\begin{itemize}
\item Taylor Polynomial of degree 15 for $f(x)=\sin(x)$ is $p(x)=x-\ds\frac{x^3}{6}+\ds\frac{x^5}{120}-\cdots+\ds\frac{x^{15}}{1,307,674,368,000}$.
\item Taylor polynomial approximation of $\sin(1)$ equals $p(1)=0.841471$.
\item True function value equals $\sin(1)=0.841471$.
\item Approximation error equals $p(1)-\sin(1)=-2.77556\times 10^{-15}$.
\end{itemize}
\vspace*{5mm}
\begin{center}
\includegraphics[scale=0.4]{sinxapproxdeg15.pdf}
\end{center}


%%%%%%%%%%%%%%%%%%%%%%%%%%%%%%%%%%%%%%%%%%%%%%%%%%%%%%%%%%%%%%%%%%%%%%
\end{frame}
%%%%%%%%%%%%%%%%%%%%%%%%%%%%%%%%%%%%%%%%%%%%%%%%%%%%%%%%%%%%%%%%%%%%%%

%%%%%%%%%%%%%%%%%%%%%%%%%%%%%%%%%%%%%%%%%%%%%%%%%%%%%%%%%%%%%%%%%%%%%%
\begin{frame}{$f(x)=\log(x)$}
%%%%%%%%%%%%%%%%%%%%%%%%%%%%%%%%%%%%%%%%%%%%%%%%%%%%%%%%%%%%%%%%%%%%%%

\begin{center}
\includegraphics[scale=0.4]{logxplot.pdf}
\end{center}
\vskip 5mm
Taylor Polynomial for $f(x)=\log(x)$ centered at $a=1$.
\beq
p(x)&=&\ds\sum^{n}_{k=1}(-1)^{k+1}\ds\frac{(x-1)^{k}}{k}\\
&=&(x-1)-\ds\frac{(x-1)^2}{2}+\ds\frac{(x-1)^3}{3}+\cdots+(-1)^n\ds\frac{(x-1)^n}{n}
\eeq

%%%%%%%%%%%%%%%%%%%%%%%%%%%%%%%%%%%%%%%%%%%%%%%%%%%%%%%%%%%%%%%%%%%%%%
\end{frame}
%%%%%%%%%%%%%%%%%%%%%%%%%%%%%%%%%%%%%%%%%%%%%%%%%%%%%%%%%%%%%%%%%%%%%%


%%%%%%%%%%%%%%%%%%%%%%%%%%%%%%%%%%%%%%%%%%%%%%%%%%%%%%%%%%%%%%%%%%%%%%
\begin{frame}{$f(x)=\log(x)$}
%%%%%%%%%%%%%%%%%%%%%%%%%%%%%%%%%%%%%%%%%%%%%%%%%%%%%%%%%%%%%%%%%%%%%%

\begin{itemize}
\item Taylor Polynomial of degree 7 for $f(x)=\log(x)$ is $p(x)=x-1$.
\item Taylor polynomial approximation of $\log(1.9)$ equals $p(1.9)=0.9$.
\item True function value equals $\log(1.9)=0.641854$.
\item Approximation error equals $p(1.9)-\log(1.9)=0.258146$.
\vskip 2mm
\item Taylor polynomial approximation of $\log(2.1)$ equals $p(2.1)=1.1$.
\item True function value equals $\log(2.1)=0.741937$.
\item Approximation error equals $p(2.1)-\log(2.1)=0.358063$.
\end{itemize}
\vspace*{2mm}
\begin{center}
\includegraphics[scale=0.35]{logxapproxdeg1.pdf}
\end{center}

%%%%%%%%%%%%%%%%%%%%%%%%%%%%%%%%%%%%%%%%%%%%%%%%%%%%%%%%%%%%%%%%%%%%%%
\end{frame}
%%%%%%%%%%%%%%%%%%%%%%%%%%%%%%%%%%%%%%%%%%%%%%%%%%%%%%%%%%%%%%%%%%%%%%

%%%%%%%%%%%%%%%%%%%%%%%%%%%%%%%%%%%%%%%%%%%%%%%%%%%%%%%%%%%%%%%%%%%%%%
\begin{frame}{$f(x)=\log(x)$}
%%%%%%%%%%%%%%%%%%%%%%%%%%%%%%%%%%%%%%%%%%%%%%%%%%%%%%%%%%%%%%%%%%%%%%

\begin{itemize}
\item Taylor Polynomial of degree 7 for $f(x)=\log(x)$ is $p(x)=(x-1)-\ds\frac{(x-1)^2}{2}+\ds\frac{(x-1)^3}{3}+\cdots+\ds\frac{(x-1)^7}{7}$.
\item Taylor polynomial approximation of $\log(1.9)$ equals $p(1.9)=0.671828$.
\item True function value equals $\log(1.9)=0.641854$.
\item Approximation error equals $p(1.9)-\log(1.9)=0.0299737$.
\vskip 2mm
\item Taylor polynomial approximation of $\log(2.1)$ equals $p(2.1)=0.877872$.
\item True function value equals $\log(2.1)=0.741937$.
\item Approximation error equals $p(2.1)-\log(2.1)=0.135934$.
\end{itemize}
\vspace*{2mm}
\begin{center}
\includegraphics[scale=0.35]{logxapproxdeg7.pdf}
\end{center}

%%%%%%%%%%%%%%%%%%%%%%%%%%%%%%%%%%%%%%%%%%%%%%%%%%%%%%%%%%%%%%%%%%%%%%
\end{frame}
%%%%%%%%%%%%%%%%%%%%%%%%%%%%%%%%%%%%%%%%%%%%%%%%%%%%%%%%%%%%%%%%%%%%%%


%%%%%%%%%%%%%%%%%%%%%%%%%%%%%%%%%%%%%%%%%%%%%%%%%%%%%%%%%%%%%%%%%%%%%%
\begin{frame}{$f(x)=\log(x)$}
%%%%%%%%%%%%%%%%%%%%%%%%%%%%%%%%%%%%%%%%%%%%%%%%%%%%%%%%%%%%%%%%%%%%%%

\begin{itemize}
\item Taylor Polynomial of degree 50 for $f(x)=\log(x)$ is $p(x)=(x-1)-\ds\frac{(x-1)^2}{2}+\ds\frac{(x-1)^3}{3}+\cdots-\ds\frac{(x-1)^{50}}{50}$.
\item Taylor polynomial approximation of $\log(1.9)$ equals $p(1.9)=0.641806$.
\item True function value equals $\log(1.9)=0.641854$.
\item Approximation error equals $p(1.9)-\log(1.9)=-0.0000483119$.
\vskip 2mm
\item Taylor polynomial approximation of $\log(2.1)$ equals $p(2.1)=-0.47615$.
\item True function value equals $\log(2.1)=0.741937$.
\item Approximation error equals $p(2.1)-\log(2.1)=-1.21809$.
\end{itemize}
\vspace*{2mm}
\begin{center}
\includegraphics[scale=0.35]{logxapproxdeg50.pdf}
\end{center}

%%%%%%%%%%%%%%%%%%%%%%%%%%%%%%%%%%%%%%%%%%%%%%%%%%%%%%%%%%%%%%%%%%%%%%
\end{frame}
%%%%%%%%%%%%%%%%%%%%%%%%%%%%%%%%%%%%%%%%%%%%%%%%%%%%%%%%%%%%%%%%%%%%%%



%%%%%%%%%%%%%%%%%%%%%%%%%%%%%%%%%%%%%%%%%%%%%%%%%%%%%%
\end{document}
%%%%%%%%%%%%%%%%%%%%%%%%%%%%%%%%%%%%%%%%%%%%%%%%%%%%%%
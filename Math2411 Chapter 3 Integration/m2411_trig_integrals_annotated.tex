\documentclass[11pt]{article}
\usepackage[suffix=Solutions]{teaching-header}

\def\classnum{2411}
\def\classtitle{Calculus II}
\def\classtitleshort{Calc 2}
\def\classsec{H01}
\def\instructor{Dr. Rostermundt}
\def\classterm{Spring 2025}


%%%%%%%%%%%%%%%%%%%%%%%%%%%%%%%%%%%%%%%%%%%%%%%%%%%%%%%%%%%%%%%%%%%%%%%%%%%%
%%%%%%%%%%%%%%%%%%%%%%%%%%%%%%%%%%%%%%%%%%%%%%%%%%%%%%%%%%%%%%%%%%%%%%%%%%%%

%This is defined in the teaching-header style file
%\ifnum\printsol=0 (when no solutions printed)
%Do something
%	\else  (when solutions are printed)
%Do something else
%\fi


% Package and setting included in teachin-header style file
%\RequirePackage{amsmath,amsfonts,amssymb,amsthm,graphicx, pgfplots, tcolorbox, xcolor,latexsym,color,verbatim,float,xcolor,setspace}
%%tikzsymbols
%
%\RequirePackage{enumerate}
%\RequirePackage{multicol}
%\RequirePackage{tikz}
%\RequirePackage{cancel}
%\usetikzlibrary{shapes.geometric}
%\usetikzlibrary{calc, positioning, arrows}
%\RequirePackage[margin=1in,letterpaper]{geometry}
%\RequirePackage[colorlinks=true,allcolors=blue]{hyperref}
%\usepackage[final]{pdfpages}
%%\usepackage{capt-of}
%
%
%\setlength{\textheight}{9in}
%\setlength{\textwidth}{6.5in}
%\addtolength{\topmargin}{0cm}
%%\addtolength{\oddsidemargin}{0cm}
%\parindent=0in
%\parskip=.35em
%\singlespacing
%%\pagestyle{empty}  % remove page numbers

%Add captions without being in figure environment
%\captioof{figure}{\text}\label[fig:]
\usepackage{capt-of}
\usepackage{mathtools}

\vfuzz2pt % Don't report over-full v-boxes if over-edge is small
\hfuzz2pt % Don't report over-full h-boxes if over-edge is small


%%%%%%%%%%%%%%%%%%%%%%%%%%%%%%%%%%%%%%%%%%%%%%%%%%%%%%
%%%%%%%%%%%%%%%%%%%%%%%%%%%%%%%%%%%%%%%%%%%%%%%%%%%%%%

\pagestyle{myheadings}

%%%%%%%%%%%%%%%%%%%%%%%%%%%%%%%%%%%%%%%%%%%%%%%%%%%%%%
%%%%%%%%%%%%%%%%%%%%%%%%%%%%%%%%%%%%%%%%%%%%%%%%%%%%%%


%%%%%%%%%%%%%%%%%%%%%%%%%%%%%%%%%%%%%%%%%%%%%%%%%%%%%%
%%%%%%%%%%%%%%%%%%%%%%%%%   Document Body   %%%%%%%%%%
%%%%%%%%%%%%%%%%%%%%%%%%%%%%%%%%%%%%%%%%%%%%%%%%%%%%%%

%Information from classinfo.tex file
%\def\classnum{2411}
%\def\classtitle{Calculus II}
%\def\classtitleshort{Calc 2}
%\def\classsec{001}
%\def\instructor{Rostermundt}
%\def\classterm{Fall 2024}
\def\topic{Trigonometric Integrals}
\def\topicshort{Trig Integrals}

	\title{\vspace{-1in}Math\classnum\;-\;\classtitle\\
	%Section\;\classsec\;-\;\classterm\\
	Guided Lecture Notes\\
	\topic}
	\author{University of Colorado Denver / College of Liberal Arts and Sciences}
	\date{Department of Mathematics}

	\markright{Math\classnum\;-\;\classtitleshort, University of Colorado Denver,\;\topicshort}


%%%%%%%%%%%%%%%%%%%%%%%%%%%%%%%%%%%%%%%%%%%%%%%%%%%%%%
\begin{document}\maketitle\thispagestyle{empty}
%%%%%%%%%%%%%%%%%%%%%%%%%%%%%%%%%%%%%%%%%%%%%%%%%%%%%%

\hrule

\section*{\topic\; Introduction:}

Our objective is to integrate function involving of products and powers of $\sin(x)$ and $\cos(x)$, or products and powers of $\sec(x)$ and $\tan(x)$  
\vskip 5mm
We will need a few basic trig identities.
\beq
\sin^2(x)+\cos^2(x)=1&&\sin(2x)=2\sin(x)\cos(x)\\
\cos^2(x)=\ds\frac{1+\cos(2x)}{2}&&\tan^2(x)=\sec^2(x)-1\\
\sin^2(x)=\ds\frac{1-\cos(2x)}{2}&&\\
\eeq

%%%%%%%%%%%%%%%%%%%%%%%%%%%%%%%%%%%%%%%%%%%%%%%%%%%%%%%%%
%%%%%%%%%%%%%%%%%%%%%%%%%%%%%%%%%%%%%%%%%%%%%%%%%%%%%%%%%


\section*{Integrals Involving $\sin^{k}(x)$ and $\cos^{j}(x)$:}

Let’s consider a simple example together.
\vskip 2mm


\begin{minipage}[]{6.5in}
%\begin{center}
\includegraphics[scale=0.8]{trig_integrals_gr1.jpg}
%\captionof{figure}{}
\label{fig:}
%\end{center}
\end{minipage}

Notice this is just a basic $u$-substitution problem. Let's consider another more complicated example.
 
\vfill\eject


%%%%%%%%%%%%%%%%%%%%%%%%%%%%%%%%%%%%%%%%%%%%%%%%%%%%%%%%%
%%%%%%%%%%%%%%%%%%%%%%%%%%%%%%%%%%%%%%%%%%%%%%%%%%%%%%%%%


\begin{example} Integrating $\ds\int\cos^j(x)\sin^k(x)\,dx$ when $j$ or $k$ is odd. As a concrete example, evaluate the integral 
\[\ds\int\cos^2(x)\sin^3(x)\,dx\]

\begin{minipage}[]{6.5in}
%\begin{center}
\includegraphics[scale=0.7]{trig_integrals_gr2.jpg}
%\captionof{figure}{}
\label{fig:}
%\end{center}
\end{minipage}

Notice after the first step the $\sin(x)\,dx$ looks like $du$ in a $u$-substitution, so this leads us to write all other trig functions in terms of $\cos(x)$.
\[\ds\int\cos^2(x)\sin^2(x)\stackrel{\stackrel{\stackrel{\text{\textcolor{red}{Wants to be our $du$}}}{\text{\textcolor{red}{term in a $u$-sub}}}}{\downarrow}}{\big[\sin(x)\,dx\big]}\quad\xRightarrow[\stackrel{\text{\textcolor{red}{Natural choice}}}{\text{\textcolor{red}{for our $u$-sub}}}]{}\quad u=\cos(x)\quad\xRightarrow[\stackrel{\text{\textcolor{red}{We have a}}}{\text{\textcolor{red}{perfect match}}}]{}\quad -du=\sin(x)\,dx\]

\noindent{\bf\emph{\underline{Observation}:}} If the exponent over $\sin(x)$ is odd we can ``attach" one of the $\sin(x)$ terms to the differential $dx$ and prepare for a $u$-substitution. Let's set up another example.

\end{example}
\vskip 2mm

%%%%%%%%%%%%%%%%%%%%%%%%%%%%%%%%%%%%%%%%%%%%%%%%%%%%%%%%%%%%%%%%%%%%%%%%%%%%%%%%%%%%%%%%%%%%%%%
%%%%%%%%%%%%%%%%%%%%%%%%%%%%%%%%%%%%%%%%%%%%%%%%%%%%%%%%%%%%%%%%%%%%%%%%%%%%%%%%%%%%%%%%%%%%%%%


\begin{example} Evaluate the following integral.

\[\ds\int\cos^3(x)\sin^5(x)\,dx\]

\noindent{\bf\emph{\underline{Workspace}:}}

\vfill\eject

\ifnum\longform=1
	\begin{boxsolution}
\vspace*{5mm} 
\beq
\ds\int\cos^3(x)\sin^5(x)\,dx&=&\ds\int\cos^3(x)\stackrel{\stackrel{\text{\textcolor{red}{This is $\Big(\sin^2(x)\Big)^2$\;\;}}}{\downarrow}}{\sin^4(x)}\stackrel{\stackrel{\stackrel{\text{\textcolor{red}{\;\;Wants to be our $du$}}}{\text{\textcolor{red}{term in a $u$-sub}}}}{\downarrow}}{\big[\sin(x)\,dx\big]}\\
\\
&=&\ds\int\cos^3(x)\Big(1-\cos^2(x)\Big)^2\sin(x)\,dx\\
\\
&=&-\ds\int u^3\Big(1-u^2\Big)^2\,du\qquad\text{Letting $u=\cos(x)\;\Longrightarrow\;-du=\sin(x)\,dx$}\\
\\
&=&-\ds\int u^3-2u^5+u^7\,du\\
\\
&=&-\ds\frac{u^4}{4}+\ds\frac{2u^6}{6}-\ds\frac{u^8}{8}+C\\
\\
&=&-\ds\frac{\cos^4(x)}{4}+\ds\frac{2\cos^6(x)}{6}-\ds\frac{\cos^8(x)}{8}+C\\
\eeq
\vspace*{5mm}
	\end{boxsolution}
\vskip 5mm

\fi 

\end{example}

Let’s have you try another example on your own.


%%%%%%%%%%%%%%%%%%%%%%%%%%%%%%%%%%%%%%%%%%%%%%%%%%%%%%%%%
%%%%%%%%%%%%%%%%%%%%%%%%%%%%%%%%%%%%%%%%%%%%%%%%%%%%%%%%%
\vskip 5mm

\begin{example}

Evaluate the integral $\ds\int\cos^3(x)\sin^6(x)\,dx$
\vskip 5mm

\noindent{\bf\emph{\underline{Workspace}:}}

\vfill\eject

\ifnum\longform=1
	\begin{boxsolution}
\vspace*{5mm} 
\beq
\ds\int\cos^3(x)\sin^6(x)\,dx&=&\ds\int\cos^2(x)\sin^6(x)\stackrel{\stackrel{\stackrel{\text{\textcolor{red}{\;\;Wants to be our $du$}}}{\text{\textcolor{red}{term in a $u$-sub}}}}{\downarrow}}{\big[\cos(x)\,dx\big]}\\
\\
&=&\ds\int\Big(1-\sin^2(x)\Big)\sin^6(x)\cos(x)\,dx\\
\\
&=&\ds\int\Big(1-u^2\Big)u^6\,du\qquad\text{Letting $u=\sin(x)\;\Longrightarrow\;du=\cos(x)\,dx$}\\
\\
&=&\ds\int u^6-u^8\,du\\
\\
&=&\ds\frac{u^7}{7}-\ds\frac{u^9}{9}+C\\
\\
&=&\ds\frac{\sin^7(x)}{7}-\ds\frac{\sin^9(x)}{9}+C\\
\eeq
\vspace*{5mm}
	\end{boxsolution}
\vskip 5mm

\fi 

\end{example}

Let’s have you try another example on your own.


%%%%%%%%%%%%%%%%%%%%%%%%%%%%%%%%%%%%%%%%%%%%%%%%%%%%%%%%%
%%%%%%%%%%%%%%%%%%%%%%%%%%%%%%%%%%%%%%%%%%%%%%%%%%%%%%%%%
\vskip 5mm

\begin{example} Evaluate the integral $\ds\int\cos^2(x)\sin^2(x)\,dx$.
\vskip 5mm
{\bf\emph{\underline{Question}:}} What do you notice about the exponents?
\vskip 5mm
\noindent{\bf\emph{\underline{Workspace}:}}

\vfill\eject

\ifnum\longform=1
	\begin{boxsolution}
\vspace*{5mm} 
\beq
\ds\int\cos^2(x)\sin^2(x)\,dx&=&\ds\int\left(\ds\frac{1+\cos(2x)}{2}\right)\left(\ds\frac{1-\cos(2x)}{2}\right)\,dx\\
\\
&=&\ds\frac{1}{4}\ds\int 1-\cos^2(2x)\,dx\qquad\Big[\text{Can also write the integrand as}\;\sin^2(2x)\Big]\\
\\
&=&\ds\frac{1}{4}\ds\int 1-\left(\ds\frac{1+\cos(4x)}{2}\right)\,dx\\
\\
&=&\ds\frac{1}{8}\ds\int 1-\cos(4x)\,dx\\
\\
&=&\ds\frac{1}{8}x-\ds\frac{1}{32}\sin(4x)+C\\
\eeq

Another possible solution method is
\beq
\ds\int\cos^2(x)\sin^2(x)\,dx&=&\ds\frac{1}{4}\ds\int\sin^2(2x)\,dx\\
\\
&=&\ds\frac{1}{8}\ds\int 1-\cos(4x)\,dx\\
\\
&=&\ds\frac{1}{8}x-\ds\frac{1}{32}\sin(4x)+C
\eeq
\vspace*{5mm}
	\end{boxsolution}
\vskip 5mm

\fi 

\end{example}

Let’s try another example together.


%%%%%%%%%%%%%%%%%%%%%%%%%%%%%%%%%%%%%%%%%%%%%%%%%%%%%%%%%
%%%%%%%%%%%%%%%%%%%%%%%%%%%%%%%%%%%%%%%%%%%%%%%%%%%%%%%%%
\vskip 5mm

\begin{example} Evaluate the integral $\ds\int\tan^6(x)\sec^4(x)\,dx$.
\vskip 5mm
\noindent{\bf\emph{\underline{Workspace}:}}

\vfill\eject

\ifnum\longform=1
	\begin{boxsolution}
\vspace*{5mm}
\begin{minipage}[]{6.5in}
%\begin{center}
\includegraphics[trim= 0cm 0cm 1cm 0cm, clip=true, scale=0.65]{trig_integrals_gr3.jpg}
%\captionof{figure}{}
\label{fig:}
%\end{center}
\end{minipage}
\vspace*{5mm}
	\end{boxsolution}
\vskip 5mm

\fi 

\end{example}

Let's consider another well-known example.


%%%%%%%%%%%%%%%%%%%%%%%%%%%%%%%%%%%%%%%%%%%%%%%%%%%%%%%%%
%%%%%%%%%%%%%%%%%%%%%%%%%%%%%%%%%%%%%%%%%%%%%%%%%%%%%%%%%
\vskip 5mm

\begin{example} Evaluate the integral $\ds\int\sec^3(x)\,dx$.
\vskip 5mm

\ifnum\longform=1
	\begin{boxsolution}
\vspace*{5mm}
\begin{minipage}[]{6.5in}
%\begin{center}
\includegraphics[trim= 0cm 0cm 1cm 0cm, clip=true, scale=0.65]{trig_integrals_gr4.jpg}
%\captionof{figure}{}
\label{fig:}
%\end{center}
\end{minipage}
\vspace*{5mm}
	\end{boxsolution}
\vskip 5mm

\fi 


\end{example}

Below are some general strategies.

\vfill\eject

%%%%%%%%%%%%%%%%%%%%%%%%%%%%%%%%%%%%%%%%%%%%%%%%%%%%%%%%%
%%%%%%%%%%%%%%%%%%%%%%%%%%%%%%%%%%%%%%%%%%%%%%%%%%%%%%%%%

\begin{minipage}[]{6.5in}
%\begin{center}
\includegraphics[scale=0.7]{trig_integrals_gr5.jpg}\\
\vskip 1cm
\includegraphics[scale=0.7]{trig_integrals_gr6.jpg}
%\captionof{figure}{}
\label{fig:}
%\end{center}
\end{minipage}


%%%%%%%%%%%%%%%%%%%%%%%%%%%%%%%%%%%%%%%%%%%%%%%%%%%%%%%%%
%%%%%%%%%%%%%%%%%%%%%%%%%%%%%%%%%%%%%%%%%%%%%%%%%%%%%%%%%
%%%%%%%%%%%%%%%%%%%%%%%%%%%%%%%%%%%%%%%%%%%%%%%%%%%%%%%%%
%%%%%%%%%%%%%%%%%%%%%%%%%%%%%%%%%%%%%%%%%%%%%%%%%%%%%%%%%

	
%%%%%%%%%%%%%%%%%%%%%%%%%%%%%%%%%%%%%%%%%%%%%%%%%%%%%%
%%%%%%%%%%%%%%%%%%%%%%%%%%%%%%%%%%%%%%%%%%%%%%%%%%%%%%

\ifnum\longform=1
\vskip 5mm
\hrule
\vskip 5mm
\begin{center}{\bf Please let me know if you have any questions, comments, or corrections!}
\end{center}	

\fi

%%%%%%%%%%%%%%%%%%%%%%%%%%%%%%%%%%%%%%%%%%%%%%%%%%%%%%
\end{document}
%%%%%%%%%%%%%%%%%%%%%%%%%%%%%%%%%%%%%%%%%%%%%%%%%%%%%%
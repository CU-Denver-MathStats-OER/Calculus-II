\documentclass[11pt]{article}
\usepackage[suffix=Solutions]{teaching-header}

\def\classnum{2411}
\def\classtitle{Calculus II}
\def\classtitleshort{Calc 2}
\def\classsec{H01}
\def\instructor{Dr. Rostermundt}
\def\classterm{Spring 2025}


%%%%%%%%%%%%%%%%%%%%%%%%%%%%%%%%%%%%%%%%%%%%%%%%%%%%%%%%%%%%%%%%%%%%%%%%%%%%
%%%%%%%%%%%%%%%%%%%%%%%%%%%%%%%%%%%%%%%%%%%%%%%%%%%%%%%%%%%%%%%%%%%%%%%%%%%%

%This is defined in the teaching-header style file
%\ifnum\printsol=0 (when no solutions printed)
%Do something
%	\else  (when solutions are printed)
%Do something else
%\fi


% Package and setting included in teachin-header style file
%\RequirePackage{amsmath,amsfonts,amssymb,amsthm,graphicx, pgfplots, tcolorbox, xcolor,latexsym,color,verbatim,float,xcolor,setspace}
%%tikzsymbols
%
%\RequirePackage{enumerate}
%\RequirePackage{multicol}
%\RequirePackage{tikz}
%\RequirePackage{cancel}
%\usetikzlibrary{shapes.geometric}
%\usetikzlibrary{calc, positioning, arrows}
%\RequirePackage[margin=1in,letterpaper]{geometry}
%\RequirePackage[colorlinks=true,allcolors=blue]{hyperref}
%\usepackage[final]{pdfpages}
%%\usepackage{capt-of}
%
%
%\setlength{\textheight}{9in}
%\setlength{\textwidth}{6.5in}
%\addtolength{\topmargin}{0cm}
%%\addtolength{\oddsidemargin}{0cm}
%\parindent=0in
%\parskip=.35em
%\singlespacing
%%\pagestyle{empty}  % remove page numbers

%Add captions without being in figure environment
%\captioof{figure}{\text}\label[fig:]
\usepackage{capt-of}
\usepackage{mathtools}

\vfuzz2pt % Don't report over-full v-boxes if over-edge is small
\hfuzz2pt % Don't report over-full h-boxes if over-edge is small


%%%%%%%%%%%%%%%%%%%%%%%%%%%%%%%%%%%%%%%%%%%%%%%%%%%%%%
%%%%%%%%%%%%%%%%%%%%%%%%%%%%%%%%%%%%%%%%%%%%%%%%%%%%%%

\pagestyle{myheadings}

%%%%%%%%%%%%%%%%%%%%%%%%%%%%%%%%%%%%%%%%%%%%%%%%%%%%%%
%%%%%%%%%%%%%%%%%%%%%%%%%%%%%%%%%%%%%%%%%%%%%%%%%%%%%%


%%%%%%%%%%%%%%%%%%%%%%%%%%%%%%%%%%%%%%%%%%%%%%%%%%%%%%
%%%%%%%%%%%%%%%%%%%%%%%%%   Document Body   %%%%%%%%%%
%%%%%%%%%%%%%%%%%%%%%%%%%%%%%%%%%%%%%%%%%%%%%%%%%%%%%%

%Information from classinfo.tex file
%\def\classnum{2411}
%\def\classtitle{Calculus II}
%\def\classtitleshort{Calc 2}
%\def\classsec{001}
%\def\instructor{Rostermundt}
%\def\classterm{Fall 2024}
\def\topic{Trigonometric Substitution}
\def\topicshort{Trig-Sub}

	\title{\vspace{-1in}Math\classnum\;-\;\classtitle\\
	%Section\;\classsec\;-\;\classterm\\
	Guided Lecture Notes\\
	\topic}
	\author{University of Colorado Denver / College of Liberal Arts and Sciences}
	\date{Department of Mathematics}

	\markright{Math\classnum\;-\;\classtitleshort, University of Colorado Denver,\;\topicshort}

%%%%%%%%%%%%%%%%%%%%%%%%%%%%%%%%%%%%%%%%%%%%%%%%%%%%%%
\begin{document}\maketitle\thispagestyle{empty}
%%%%%%%%%%%%%%%%%%%%%%%%%%%%%%%%%%%%%%%%%%%%%%%%%%%%%%

\hrule

\section*{\topic\; Introduction:}

Our objective is to integrate function involving square roots of differences and sums of squares.
\[\sqrt{a^2-x^2}\qquad\sqrt{x^2-a^2}\qquad\sqrt{x^2+a^2}\]
\vskip 5mm
We will need a few basic trig identities.
\beq
\sin^2(x)+\cos^2(x)=1&&\sin(2x)=\sin^2(x)=\ds\frac{1-\cos(2x)}{2}\\
\cos^2(x)=\ds\frac{1+\cos(2x)}{2}&&\tan^2(x)=\sec^2(x)-1
\eeq

%%%%%%%%%%%%%%%%%%%%%%%%%%%%%%%%%%%%%%%%%%%%%%%%%%%%%%%%%
%%%%%%%%%%%%%%%%%%%%%%%%%%%%%%%%%%%%%%%%%%%%%%%%%%%%%%%%%


\section*{Integrals involving $\sqrt{a^2-x^2}$:}

Let’s consider an example together. The general strategy is to make a substitution $x=a\sin(\theta)$.
\vskip 2mm

\begin{minipage}[]{6.5in}
\begin{center}
\includegraphics[scale=0.6]{trig_sub_gr07.jpg}
%\captionof{figure}{}
\label{fig:}
\end{center}
\end{minipage}

Then our square root quantity is converted into a simple trig function. Here is a right triangle for reference.
\vskip 2mm

\begin{minipage}[]{6.5in}
\begin{center}
\includegraphics[scale=0.55]{trig_sub_gr01.jpg}
\captionof{figure}{Reference triangle for $\sqrt{a^2-x^2}$.}
\label{fig:}
\end{center}
\end{minipage}

\vfill\eject


%%%%%%%%%%%%%%%%%%%%%%%%%%%%%%%%%%%%%%%%%%%%%%%%%%%%%%%%%
%%%%%%%%%%%%%%%%%%%%%%%%%%%%%%%%%%%%%%%%%%%%%%%%%%%%%%%%%


\begin{example} Evaluate $\ds\int\sqrt{9-x^2}\,dx$.
\vskip 5mm
\noindent{\bf\emph{\underline{Workspace}:}}

\vfill\eject

\ifnum\longform=1
	\begin{boxsolution}
\vspace*{5mm}
We let $x=3\sin(\theta)$ and construct our reference triangle.

\begin{minipage}[]{6.5in}
\begin{center}
\includegraphics[scale=0.7]{trig_sub_gr08.jpg}
\captionof{figure}{Reference triangle for $\sqrt{9-x^2}$.}
\label{fig:}
\end{center}
\end{minipage}

Then we have

\begin{minipage}[]{6.5in}
\begin{center}
\includegraphics[scale=0.7]{trig_sub_gr09.jpg}
%\captionof{figure}{}
\label{fig:}
\end{center}
\end{minipage}
\vspace*{5mm}
	\end{boxsolution}
\vskip 5mm

\fi

\end{example}

\vskip 5mm
Let's now have you work an example. 
\vskip 5mm


%%%%%%%%%%%%%%%%%%%%%%%%%%%%%%%%%%%%%%%%%%%%%%%%%%%%%%%%%
%%%%%%%%%%%%%%%%%%%%%%%%%%%%%%%%%%%%%%%%%%%%%%%%%%%%%%%%%

\begin{example} Evaluate the integral $\ds\int x^3\sqrt{1-x^2}\,dx$.

\ifnum\longform=1
\vfill\eject
	\else
\vskip 5mm

\fi

\noindent{\bf\emph{\underline{Workspace}:}}

\vfill\eject

\ifnum\longform=1
	\begin{boxsolution}
\vspace*{5mm}

\begin{minipage}[]{6.5in}
\begin{center}
\includegraphics[scale=0.7]{trig_sub_gr10.jpg}
%\captionof{figure}{}
\label{fig:}
\end{center}
\end{minipage}
\vspace*{5mm}
	\end{boxsolution}
\vskip 5mm

\fi

We can also solve his using $u$-sub. See if you can solve this using $u$-sub.
\vskip 5mm
\noindent{\bf\emph{\underline{Workspace}:}}

\vfill\eject

\ifnum\longform=1
	\begin{boxsolution}
\vspace*{5mm}
We let $u=1-x^2$ so that $du=2x\,dx$.
\vskip 5mm
\begin{minipage}[]{6.5in}
\begin{center}
\includegraphics[scale=0.7]{trig_sub_gr11.jpg}
%\captionof{figure}{}
\label{fig:}
\end{center}
\end{minipage}
\vspace*{5mm}
	\end{boxsolution}
\vskip 5mm

\fi

\end{example}

%%%%%%%%%%%%%%%%%%%%%%%%%%%%%%%%%%%%%%%%%%%%%%%%%%%%%%%%%
%%%%%%%%%%%%%%%%%%%%%%%%%%%%%%%%%%%%%%%%%%%%%%%%%%%%%%%%%

\section*{Integrals involving $\sqrt{x^2+a^2}$:}

Let's consider integrals with a term $\sqrt{x^2+a^2}$. We let $x=a\tan(\theta)$ and build a reference triangle.
\vskip 2mm
\begin{minipage}[]{6.5in}
\begin{center}
\includegraphics[scale=0.6]{trig_sub_gr03.jpg}
\captionof{figure}{Reference triangle for $\sqrt{x^2+a^2}$}
\label{fig:}
\end{center}
\end{minipage}
\vskip 2mm
We can use the following problem solving strategy.
\vskip 5mm
\begin{minipage}[]{6.5in}
\begin{center}
\includegraphics[scale=0.6]{trig_sub_gr04pt1.jpg}\\
\includegraphics[scale=0.6]{trig_sub_gr04pt2.jpg}
%\captionof{figure}{}
\label{fig:}
\end{center}
\end{minipage}

\ifnum\longform=1
\vfill\eject
	\else
\vskip 5mm

\fi

\begin{example} Calculate the length of the curve $y=x^2$ on the interval $[0,1/2]$. Our arclength formula gives us
\[s=\ds\int^{x=b}_{x=a}\sqrt{1+\big[f'(x)\big]^2}\,dx=\ds\int^{x=1/2}_{x=0}\sqrt{1+4x^2}\,dx.\]
This looks like a tricky integral since there is no obvious $u$-substitution. We try a trig-sub.
We let $x=\frac{1}{2}\tan(\theta)$ so that $dx=\frac{1}{2}\sec^2(\theta)\,d\theta$. Now continue on your own.
\vskip 5mm
\note Notice that the quantity $\sqrt{1+4x^2}$ suggests a reference triangle with leg lengths of $1$ and $2x$ giving a hypotenuse with length of $\sqrt{1+4x^2}$. Build your own reference triangle.
\vskip 5mm
\noindent{\bf\emph{\underline{Workspace}:}}

\vfill\eject

\ifnum\longform=0
\noindent{\bf\emph{\underline{Workspace Cont.}:}}

\vfill\eject

\fi

\ifnum\longform=1
	\begin{boxsolution}
\vspace*{5mm}
Now we have
\vskip 5mm
\begin{minipage}[]{6.5in}
\begin{center}
\includegraphics[scale=0.75]{trig_sub_gr12.jpg}
%\captionof{figure}{}
\label{fig:}
\end{center}
\end{minipage}
\vskip 2mm
Notice that even in trig-sub we {\bf\emph{change our limits of integration}} after the change of variables. Let's try another example
\vspace*{5mm}
	\end{boxsolution}
\vskip 5mm

\fi

\end{example}



%%%%%%%%%%%%%%%%%%%%%%%%%%%%%%%%%%%%%%%%%%%%%%%%%%%%%%%%%
%%%%%%%%%%%%%%%%%%%%%%%%%%%%%%%%%%%%%%%%%%%%%%%%%%%%%%%%%


\begin{example} Evaluate the integral $\ds\int\ds\frac{1}{\sqrt{1+x^2}}\,dx$.
\vskip 5mm
Start by letting $x=\tan(\theta)$ and forming the reference triangle.
\vskip 5mm
\begin{minipage}[]{6.5in}
\begin{center}
\includegraphics[scale=0.8]{trig_sub_gr14.jpg}
\captionof{figure}{Reference triangle for $\sqrt{1+x^2}$}
\label{fig:}
\end{center}
\end{minipage}
\vskip 2mm
Now solve the integral by yourself.
\vskip 5mm
\noindent{\bf\emph{\underline{Workspace}:}}

\vfill\eject

\ifnum\longform=1
	\begin{boxsolution}
\vspace*{5mm}
\begin{minipage}[]{6.5in}
\begin{center}
\includegraphics[scale=0.8]{trig_sub_gr15.jpg}
%\captionof{figure}{}
\label{fig:}
\end{center}
\end{minipage}
\vspace*{5mm}
	\end{boxsolution}
\vskip 5mm

\fi

\end{example}

%%%%%%%%%%%%%%%%%%%%%%%%%%%%%%%%%%%%%%%%%%%%%%%%%%%%%%%%%
%%%%%%%%%%%%%%%%%%%%%%%%%%%%%%%%%%%%%%%%%%%%%%%%%%%%%%%%%

\section*{Integrals involving $\sqrt{x^2-a^2}$:}


\begin{example} Evaluate the integral $\ds\int^{x=5}_{x=3}\sqrt{x^2-9}\,dx$.
\vskip 5mm
The geometry suggests we let $x=3\sec(\theta)$ and so then have $dx=3\sec(\theta)\tan(\theta)\,d\theta$.
\vskip 5mm
\begin{minipage}[]{6.5in}
\begin{center}
\includegraphics[scale=0.7]{trig_sub_gr05.jpg}
\captionof{figure}{Reference triangle for $\sqrt{x^2-a^2}$}
\label{fig:}
\end{center}
\end{minipage}

\vskip 5mm

Notice there are different reference triangles depending on whether the $x$-values are positive (so we have  $x>a$) or negative (so we have $x<-a$). The main consequences is when $x$ is positive we have $\sqrt{x^2-a^2}=a\tan(\theta)$. When $x$ is negative we have $\sqrt{x^2-a^2}=-a\tan(\theta)$.

\ifnum\longform=1
\vfill\eject
	\else
\vskip 5mm

\fi

\begin{minipage}[]{6.5in}
\begin{center}
\includegraphics[scale=0.7]{trig_sub_gr06.jpg}
%\captionof{figure}{}
\label{fig:}
\end{center}
\end{minipage}

\ifnum\longform=1
	\begin{boxsolution}
\vspace*{5mm}
\begin{minipage}[]{6.5in}
\begin{center}
\includegraphics[scale=0.62]{trig_sub_gr16.jpg}
%\captionof{figure}{}
\label{fig:}
\end{center}
\end{minipage}
\vspace*{5mm}
	\end{boxsolution}
\vskip 5mm

\fi

\end{example}

Let's try an example on your own.
\vskip 5mm

%%%%%%%%%%%%%%%%%%%%%%%%%%%%%%%%%%%%%%%%%%%%%%%%%%%%%%%%%
%%%%%%%%%%%%%%%%%%%%%%%%%%%%%%%%%%%%%%%%%%%%%%%%%%%%%%%%%

\begin{example} Evaluate the integral $\ds\int\ds\frac{1}{\sqrt{x^2-4}}\,dx$, assuming that $x<-2$.
\vskip 1mm
\note The domain of the integrand $f(x)=1/\sqrt{x^2-4}$ is $\big(-\infty,-2\big)\cup\big(2,\infty\big)$.

\ifnum\longform=1
\vfill\eject
	\else
\vskip 5mm

\fi

\noindent{\bf\emph{\underline{Workspace}:}}

\vfill


\end{example}

%%%%%%%%%%%%%%%%%%%%%%%%%%%%%%%%%%%%%%%%%%%%%%%%%%%%%%%%%
%%%%%%%%%%%%%%%%%%%%%%%%%%%%%%%%%%%%%%%%%%%%%%%%%%%%%%%%%

%%%%%%%%%%%%%%%%%%%%%%%%%%%%%%%%%%%%%%%%%%%%%%%%%%%%%%%%%
%%%%%%%%%%%%%%%%%%%%%%%%%%%%%%%%%%%%%%%%%%%%%%%%%%%%%%%%%
%%%%%%%%%%%%%%%%%%%%%%%%%%%%%%%%%%%%%%%%%%%%%%%%%%%%%%%%%
%%%%%%%%%%%%%%%%%%%%%%%%%%%%%%%%%%%%%%%%%%%%%%%%%%%%%%%%%

	
%%%%%%%%%%%%%%%%%%%%%%%%%%%%%%%%%%%%%%%%%%%%%%%%%%%%%%
%%%%%%%%%%%%%%%%%%%%%%%%%%%%%%%%%%%%%%%%%%%%%%%%%%%%%%

\ifnum\longform=1
\vskip 1cm
\hrule
\vskip 5mm
\begin{center}{\bf Please let me know if you have any questions, comments, or corrections!}
\end{center}	

\fi

%%%%%%%%%%%%%%%%%%%%%%%%%%%%%%%%%%%%%%%%%%%%%%%%%%%%%%
\end{document}
%%%%%%%%%%%%%%%%%%%%%%%%%%%%%%%%%%%%%%%%%%%%%%%%%%%%%%
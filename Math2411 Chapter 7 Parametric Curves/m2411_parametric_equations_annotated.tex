\documentclass[11pt]{article}
\usepackage[suffix=Solutions]{teaching-header}

\def\classnum{2411}
\def\classtitle{Calculus II}
\def\classtitleshort{Calc 2}
\def\classsec{H01}
\def\instructor{Dr. Rostermundt}
\def\classterm{Spring 2025}


%%%%%%%%%%%%%%%%%%%%%%%%%%%%%%%%%%%%%%%%%%%%%%%%%%%%%%%%%%%%%%%%%%%%%%%%%%%%
%%%%%%%%%%%%%%%%%%%%%%%%%%%%%%%%%%%%%%%%%%%%%%%%%%%%%%%%%%%%%%%%%%%%%%%%%%%%

%This is defined in the teaching-header style file
%\ifnum\printsol=0 (when no solutions printed)
%Do something
%	\else  (when solutions are printed)
%Do something else
%\fi


% Package and setting included in teachin-header style file
%\RequirePackage{amsmath,amsfonts,amssymb,amsthm,graphicx, pgfplots, tcolorbox, xcolor,latexsym,color,verbatim,float,xcolor,setspace}
%%tikzsymbols
%
%\RequirePackage{enumerate}
%\RequirePackage{multicol}
%\RequirePackage{tikz}
%\RequirePackage{cancel}
%\usetikzlibrary{shapes.geometric}
%\usetikzlibrary{calc, positioning, arrows}
%\RequirePackage[margin=1in,letterpaper]{geometry}
%\RequirePackage[colorlinks=true,allcolors=blue]{hyperref}
%\usepackage[final]{pdfpages}
%%\usepackage{capt-of}
%
%
%\setlength{\textheight}{9in}
%\setlength{\textwidth}{6.5in}
%\addtolength{\topmargin}{0cm}
%%\addtolength{\oddsidemargin}{0cm}
%\parindent=0in
%\parskip=.35em
%\singlespacing
%%\pagestyle{empty}  % remove page numbers

%Add captions without being in figure environment
%\captioof{figure}{\text}\label[fig:]
\usepackage{capt-of}
\usepackage{mathtools}

\vfuzz2pt % Don't report over-full v-boxes if over-edge is small
\hfuzz2pt % Don't report over-full h-boxes if over-edge is small


%%%%%%%%%%%%%%%%%%%%%%%%%%%%%%%%%%%%%%%%%%%%%%%%%%%%%%
%%%%%%%%%%%%%%%%%%%%%%%%%%%%%%%%%%%%%%%%%%%%%%%%%%%%%%

\pagestyle{myheadings}

%%%%%%%%%%%%%%%%%%%%%%%%%%%%%%%%%%%%%%%%%%%%%%%%%%%%%%
%%%%%%%%%%%%%%%%%%%%%%%%%%%%%%%%%%%%%%%%%%%%%%%%%%%%%%


%%%%%%%%%%%%%%%%%%%%%%%%%%%%%%%%%%%%%%%%%%%%%%%%%%%%%%
%%%%%%%%%%%%%%%%%%%%%%%%%   Document Body   %%%%%%%%%%
%%%%%%%%%%%%%%%%%%%%%%%%%%%%%%%%%%%%%%%%%%%%%%%%%%%%%%

%Information from classinfo.tex file
%\def\classnum{2411}
%\def\classtitle{Calculus II}
%\def\classtitleshort{Calc 2}
%\def\classsec{001}
%\def\instructor{Rostermundt}
%\def\classterm{Fall 2024}
\def\topic{Parametric Equations}
\def\topicshort{Parametric Equations}

	\title{\vspace{-1in}Math\classnum\;-\;\classtitle\\
	%Section\;\classsec\;-\;\classterm\\
	Guided Lecture Notes\\
	\topic}
	\author{University of Colorado Denver / College of Liberal Arts and Sciences}
	\date{Department of Mathematics}

	\markright{Math\classnum\;-\;\classtitleshort, University of Colorado Denver,\;\topicshort}



%%%%%%%%%%%%%%%%%%%%%%%%%%%%%%%%%%%%%%%%%%%%%%%%%%%%%%
\begin{document}\maketitle\thispagestyle{empty}
%%%%%%%%%%%%%%%%%%%%%%%%%%%%%%%%%%%%%%%%%%%%%%%%%%%%%%

\hrule

\section*{\topic\, Introduction:}

Our objective is to examine parametric equations and their graphs. In the two-dimensional coordinate system, parametric
equations are useful for describing curves that are not necessarily functions. The parameter is an independent variable that
both $x$ and $y$ depend on, and as the parameter increases, the values of $x$ and $y$ trace out a path along a plane curve. For
example, if the parameter is $t$ (a common choice), then t might represent time. Then $x$ and $y$ are defined as functions of time,
and $(x(t),y(t))$ can describe the position in the plane of a given object as it moves along a curved path. One example could be describing the orbit of the Earth around the sun.

\vskip 5mm


\begin{minipage}[]{6.5in}
\begin{center}
\includegraphics[scale=0.6]{earth_orbit.jpg}
\captionof{figure}{Coordinate Axes Superimposed on the Orbit of Earth}
\label{fig:orbit}
\end{center}
\end{minipage}

\vskip 1cm

\begin{minipage}[]{6.5in}
\begin{center}
\includegraphics[trim= 0cm 0.4cm 0cm 0cm, clip=true, scale=0.7]{parametric_curve_def_pt1.jpg}
\includegraphics[trim= 0cm 0cm 0.1cm 0.3cm, clip=true, scale=0.7]{parametric_curve_def_pt2.jpg}
%\captionof{figure}{}
\label{fig:def}
\end{center}
\end{minipage}

\vfill\eject

%%%%%%%%%%%%%%%%%%%%%%%%%%%%%%%%%%%%%%%%%%%%%%%%%%%%%%%%%
%%%%%%%%%%%%%%%%%%%%%%%%%%%%%%%%%%%%%%%%%%%%%%%%%%%%%%%%%

\section*{Parametric Curve Examples:}


\begin{minipage}[]{6.5in}
\begin{center}
\includegraphics[trim= 0cm 0cm 0cm 0cm, clip=true, scale=0.7]{parametric_curve_gr1.jpg}
\captionof{figure}{Graph of Parametric Curve with $x(t)=t-1$ and $y(t)=2t+4$ where $-3\le t\le 2$}
\label{fig:parametricgr1}
\end{center}
\vskip 1cm
\begin{center}
\includegraphics[trim= 0cm 0cm 0cm 0cm, clip=true, scale=0.7]{parametric_curve_gr2.jpg}
\captionof{figure}{Graph of Parametric Curve with $x(t)=t^2-3$ and $y(t)=2t+1$ where $-2\le t\le 3$}
\label{fig:parametricgr2}
\end{center}
\end{minipage}
	
\vfill\eject

\begin{minipage}[]{6.5in}
\begin{center}
\includegraphics[trim= 0cm 0cm 0cm 0cm, clip=true, scale=0.7]{parametric_curve_gr3.jpg}
\captionof{figure}{Graph of Parametric Curve with $x(t)=4\cos(t)$ and $y(t)=4\sin(t)$ where $0\le t\le 2\pi$}
\label{fig:parametricgr3}
\end{center}
\end{minipage}

%%%%%%%%%%%%%%%%%%%%%%%%%%%%%%%%%%%%%%%%%%%%%%%%%%%%%%%%%
%%%%%%%%%%%%%%%%%%%%%%%%%%%%%%%%%%%%%%%%%%%%%%%%%%%%%%%%%

\section*{Eliminating the Parameter:}

Suppose we have a parametric curve described as $x(t)=t-1$ and $y(t)=2t+4$ where $-3\le t\le 2$. Can we write this in the form $y=f(x)$? This is called eliminating the parameter.
\vskip 5mm
If possible, solve $x=x(t)$ for $t$. $x=t-1$ gives us $t=x+1$. Then substitute back into $y(t)=2t+4$ to get
\[y=2t+4=2(x+1)+4=2x+6.\]
Because $-3\le t\le 2$ we have $-3\le x+1\le 2$ or $-4\le x\le 1$. So the parametric curve is the same graph as the straight line given by $y=2x+6$ on the interval $[-4,1]$. This agrees with our earlier sketch of the parametric curve.
\vskip 5mm
Let's have you work an example independently.


%%%%%%%%%%%%%%%%%%%%%%%%%%%%%%%%%%%%%%%%%%%%%%%%%%%%%%%%%
%%%%%%%%%%%%%%%%%%%%%%%%%%%%%%%%%%%%%%%%%%%%%%%%%%%%%%%%%

\vskip 5mm
\begin{example} Eliminate the parameter for the parametric curve given by $x(t)=4\cos(t)$ and $y(t)=4\sin(t)$ where $0\le t\le 2\pi$.
\vskip 5mm
\noindent{\bf\emph{\underline{Workspace}:}}

\vfill\eject

\ifnum\shortform=1
	\begin{boxsolution}
We could try to solve $x(t)=4\cos(t)$ for $t$ arriving at $t=\cos^{-1}(x/4)$ so that 
\[y=\sin\left(\cos^{-1}\left(\ds\frac{x}{4}\right)\right)\]
where $0\le t\le 2\pi$ is equivalent to $-4\le x\le 4$. Some basic trigonometry allows us to write
\[y=\pm\sqrt{16-x^2}\]
with $-4\le x\le 4$ which is the equation of a circle of radius 4. Maybe there is an easier approach?
\vskip 1cm
Fortunately, there is! We can also proceed as follows: $x(t)=4\cos(t)$ and $y(t)=4\sin(t)$ implies
\[x^2+y^2=16\cos^2(t)+16\sin^2(t)=16\quad\Longrightarrow\quad y=\pm\sqrt{16-x^2}.\]
	\end{boxsolution}
\fi


\end{example}

%%%%%%%%%%%%%%%%%%%%%%%%%%%%%%%%%%%%%%%%%%%%%%%%%%%%%%%%%
%%%%%%%%%%%%%%%%%%%%%%%%%%%%%%%%%%%%%%%%%%%%%%%%%%%%%%%%%

\vskip 5mm
\begin{example} Eliminate the parameter for the parametric curve given by $x(t)=t^2-3$ and $y(t)=2t+1$ where $-2\le t\le 3$.
\vskip 5mm
\noindent{\bf\emph{\underline{Workspace}:}}

\vfill\eject

\ifnum\shortform=1
	\begin{boxsolution}
Let's solve $y(t)=2t+1$ for $t$ getting $t=(y-1)/2$. Now a substitution gives us
\[x=t^2-3=\left(\ds\frac{y-1}{2}\right)^2-3=\ds\frac{1}{4}\big(y-1\big)^2-3.\]
Moreover, $-2\le t\le 3$ is gives us $-2\le (y-1)/2\le 3$ which is equivalent to $-3\le y\le 7$. So the parametric curve is the as the graph of
\[x=\ds\frac{1}{4}\big(y-1\big)^2-3\]
which is a scaled parabola whose vertex is shifted up one unit and left 3 units on the interval $-3\le y\le 7$. This agrees with the previous graph. 
	\end{boxsolution}
\fi
	
\end{example}

\vskip 5mm
Let's work another example.
\vskip 5mm

%%%%%%%%%%%%%%%%%%%%%%%%%%%%%%%%%%%%%%%%%%%%%%%%%%%%%%%%%
%%%%%%%%%%%%%%%%%%%%%%%%%%%%%%%%%%%%%%%%%%%%%%%%%%%%%%%%%

\vskip 5mm
\begin{example} Eliminate the parameter for the parametric curve given by $x(t)=\sqrt{2t+4}$ and $y(t)=2t+1$ where $-2\le t\le 6$.
\vskip 5mm
\noindent{\bf\emph{\underline{Workspace}:}}

\vfill\eject

\ifnum\shortform=1
	\begin{boxsolution}
Let's solve $x(t)=\sqrt{2t+4}$ for $t$ getting $t=(x^2-4)/2$. Now a substitution gives us
\[y=2t+1=2\left(\ds\frac{x^2-4}{2}\right)+1=x^2-3.\]
Moreover, $-2\le t\le 6$ is gives us $-2\le (x^2-4)/2\le 6$ which is equivalent to $0\le x\le 4$ since $\sqrt{2t+4}$ in increasing. So the parametric curve is the as the graph of
\[y=x^2-3\]
on the interval $0\le x\le 4$.

\vskip 1cm

\begin{minipage}[]{6.5in}
\begin{center}
\includegraphics[trim= 0cm 0cm 0cm 0cm, clip=true, scale=0.7]{parametric_curve_gr4.jpg}
\captionof{figure}{Parametric Curve with $x(t)=\sqrt{2t+4}$ and $y(t)=2t+1$ where $-2\le t\le 6$}
\label{fig:parametricgr4}
\end{center}
\end{minipage}

\vspace*{1cm}

	\end{boxsolution}
\fi

\end{example}
 
%%%%%%%%%%%%%%%%%%%%%%%%%%%%%%%%%%%%%%%%%%%%%%%%%%%%%%
%%%%%%%%%%%%%%%%%%%%%%%%%%%%%%%%%%%%%%%%%%%%%%%%%%%%%%
 
 
\section*{Parameterize a Curve:}

Suppose we have $y=2x^2-3$ for all $x$-values. The simplest way to parameterize the curve is to let $x(t)=t$ and $y(t)=2t^2-3$. But this is not the only choice. We could choose any $x(t)$ that takes on all $x$-values. For example, we could let $x(t)=3t-2$ for all $t$-values. Then we have
\[y(t)=2x^2-3=2\big(3t-2\big)^2-3=18t^2-24t+6.\] 
We have the parametric curve $x(t)=3t-2$ and $y(t)=18t^2-24t+6$ for all $t$-values.

%%%%%%%%%%%%%%%%%%%%%%%%%%%%%%%%%%%%%%%%%%%%%%%%%%%%%%
%%%%%%%%%%%%%%%%%%%%%%%%%%%%%%%%%%%%%%%%%%%%%%%%%%%%%%
 
\ifnum\shortform=1
\section*{An Interesting Example - The Cycloid:}

Imagine going on a bicycle ride through the country. The tires stay in contact with the road and rotate in a predictable
pattern. Now suppose a very determined ant is tired after a long day and wants to get home. So he hangs onto the side of
the tire and gets a free ride. The path that this ant travels down a straight road is called a {\bf\emph{cycloid}}. A cycloid
generated by a circle (or bicycle wheel) of radius $a$ is given by the parametric equations
\[x(t)=a\big(t-\sin t\big)\qquad y(t)=a\big(1-\cos t\big).\]

\vskip 5mm

\begin{minipage}[]{6.5in}
\begin{center}
\includegraphics[trim= 0cm 0cm 0cm 0cm, clip=true, scale=0.7]{cycloid_graphic.jpg}
\captionof{figure}{Graph of Cycloid Curve with $x(t)=a\big(t-\sin t\big)$ and $y(t)=a\big(1-\cos t\big)$}
\label{fig:cycloid}
\end{center}
\end{minipage}
\vskip 5mm
There are many interesting variations and applications of cycloid curves. Anyone who is interested can easily find many resources about cycloid curves.
\fi

%%%%%%%%%%%%%%%%%%%%%%%%%%%%%%%%%%%%%%%%%%%%%%%%%%%%%%
%%%%%%%%%%%%%%%%%%%%%%%%%%%%%%%%%%%%%%%%%%%%%%%%%%%%%%



 
%%%%%%%%%%%%%%%%%%%%%%%%%%%%%%%%%%%%%%%%%%%%%%%%%%%%%%%%%
%%%%%%%%%%%%%%%%%%%%%%%%%%%%%%%%%%%%%%%%%%%%%%%%%%%%%%%%%
%%%%%%%%%%%%%%%%%%%%%%%%%%%%%%%%%%%%%%%%%%%%%%%%%%%%%%%%%
%%%%%%%%%%%%%%%%%%%%%%%%%%%%%%%%%%%%%%%%%%%%%%%%%%%%%%%%%

	
%%%%%%%%%%%%%%%%%%%%%%%%%%%%%%%%%%%%%%%%%%%%%%%%%%%%%%
%%%%%%%%%%%%%%%%%%%%%%%%%%%%%%%%%%%%%%%%%%%%%%%%%%%%%%


\vskip 1cm
\hrule
\vskip 5mm
\begin{center}{\bf Please let me know if you have any questions, comments, or corrections!}
\end{center}	


%%%%%%%%%%%%%%%%%%%%%%%%%%%%%%%%%%%%%%%%%%%%%%%%%%%%%%
\end{document}
%%%%%%%%%%%%%%%%%%%%%%%%%%%%%%%%%%%%%%%%%%%%%%%%%%%%%%
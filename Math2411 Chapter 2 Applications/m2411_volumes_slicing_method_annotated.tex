\documentclass[11pt]{article}
\usepackage[suffix=Solutions]{teaching-header}

\def\classnum{2411}
\def\classtitle{Calculus II}
\def\classtitleshort{Calc 2}
\def\classsec{H01}
\def\instructor{Dr. Rostermundt}
\def\classterm{Spring 2025}


%%%%%%%%%%%%%%%%%%%%%%%%%%%%%%%%%%%%%%%%%%%%%%%%%%%%%%%%%%%%%%%%%%%%%%%%%%%%
%%%%%%%%%%%%%%%%%%%%%%%%%%%%%%%%%%%%%%%%%%%%%%%%%%%%%%%%%%%%%%%%%%%%%%%%%%%%

%This is defined in the teaching-header style file
%\ifnum\printsol=0 (when no solutions printed)
%Do something
%	\else  (when solutions are printed)
%Do something else
%\fi


% Package and setting included in teachin-header style file
%\RequirePackage{amsmath,amsfonts,amssymb,amsthm,graphicx, pgfplots, tcolorbox, xcolor,latexsym,color,verbatim,float,xcolor,setspace}
%%tikzsymbols
%
%\RequirePackage{enumerate}
%\RequirePackage{multicol}
%\RequirePackage{tikz}
%\RequirePackage{cancel}
%\usetikzlibrary{shapes.geometric}
%\usetikzlibrary{calc, positioning, arrows}
%\RequirePackage[margin=1in,letterpaper]{geometry}
%\RequirePackage[colorlinks=true,allcolors=blue]{hyperref}
%\usepackage[final]{pdfpages}
%%\usepackage{capt-of}
%
%
%\setlength{\textheight}{9in}
%\setlength{\textwidth}{6.5in}
%\addtolength{\topmargin}{0cm}
%%\addtolength{\oddsidemargin}{0cm}
%\parindent=0in
%\parskip=.35em
%\singlespacing
%%\pagestyle{empty}  % remove page numbers

%Add captions without being in figure environment
%\captioof{figure}{\text}\label[fig:]
\usepackage{capt-of}
\usepackage{mathtools}

\vfuzz2pt % Don't report over-full v-boxes if over-edge is small
\hfuzz2pt % Don't report over-full h-boxes if over-edge is small


%%%%%%%%%%%%%%%%%%%%%%%%%%%%%%%%%%%%%%%%%%%%%%%%%%%%%%
%%%%%%%%%%%%%%%%%%%%%%%%%%%%%%%%%%%%%%%%%%%%%%%%%%%%%%

\pagestyle{myheadings}

%%%%%%%%%%%%%%%%%%%%%%%%%%%%%%%%%%%%%%%%%%%%%%%%%%%%%%
%%%%%%%%%%%%%%%%%%%%%%%%%%%%%%%%%%%%%%%%%%%%%%%%%%%%%%


%%%%%%%%%%%%%%%%%%%%%%%%%%%%%%%%%%%%%%%%%%%%%%%%%%%%%%
%%%%%%%%%%%%%%%%%%%%%%%%%   Document Body   %%%%%%%%%%
%%%%%%%%%%%%%%%%%%%%%%%%%%%%%%%%%%%%%%%%%%%%%%%%%%%%%%

%Information from classinfo.tex file
%\def\classnum{2411}
%\def\classtitle{Calculus II}
%\def\classtitleshort{Calc 2}
%\def\classsec{001}
%\def\instructor{Rostermundt}
%\def\classterm{Fall 2024}
\def\topic{Volumes - The Slicing Method}
\def\topicshort{The Slicing Method}

	\title{\vspace{-1in}Math\classnum\;-\;\classtitle\\
	%Section\;\classsec\;-\;\classterm\\
	Guided Lecture Notes\\
	\topic}
	\author{University of Colorado Denver / College of Liberal Arts and Sciences}
	\date{Department of Mathematics}

	\markright{Math\classnum\;-\;\classtitleshort, University of Colorado Denver,\;\topicshort}


%%%%%%%%%%%%%%%%%%%%%%%%%%%%%%%%%%%%%%%%%%%%%%%%%%%%%%
\begin{document}\maketitle\thispagestyle{empty}
%%%%%%%%%%%%%%%%%%%%%%%%%%%%%%%%%%%%%%%%%%%%%%%%%%%%%%

\hrule

\section*{\topic\; Introduction}

Suppose that we want to determine the volume of a 3D region/solid with a varying cross-section such as shown below.

\begin{figure}[h]
\begin{center}
\includegraphics[scale=0.6]{slicing_gr01.jpg}
\caption{3D region/solid with a varying cross-section}
\end{center}
\end{figure}

In this section we will start by dividing the solid into slices along the $x$-axis and then analyze the volume of the slices.

\begin{figure}[h]
\begin{center}
\includegraphics[scale=0.6]{slicing_gr02.jpg}
\caption{3D region/solid divided into three slices}
\end{center}
\end{figure}

The volume of a slice located at $x^{*}_k$ can be estimated as $Vol\big(\,slice_k\,\big)=A(x^{*}_{k})\Delta x$ where $A(x^{*}_{k})$ is the area of a cross-section at point $x^{*}_{k}$. Then, if there are $n$ total slices we have
\[\text{Total Volume}\approx\ds\sum^{n}_{k=1}Vol\Big(\,slice_k\,\Big)=\ds\sum^{n}_{k=1}A(x^{*}_{k})\Delta x.\]
As the number of slices increases, the width of the slices decreases and we can calculate the total volume as an integral.
\[Volume=\stackrel{\stackrel{\text{\textcolor{red}{Sum}}}{\downarrow}}{\int^{x=b}_{x=a}}\;\underbrace{\stackrel{\stackrel{\text{\textcolor{red}{\;Cross-Section Area\;}}}{\downarrow}}{A(x)}\;\;\stackrel{\stackrel{\text{\textcolor{red}{\;Width of Slice\;}}}{\downarrow}}{dx}}_{\text{\textcolor{blue}{Volume of Slice}}}\] 

We will focus on solids formed by revolving regions in the $(x,y)$-plane about a coordinate axis.

\section*{Solids of Revolution}

Start with a region below the graph $y=f(x)$ and revolve around the $x$-axis.

\begin{figure}[h]
\begin{center}
\includegraphics[scale=0.5]{slicing_gr03.jpg}$\Longrightarrow$\includegraphics[scale=0.5]{slicing_gr04.jpg}$\Longrightarrow$\includegraphics[scale=0.5]{slicing_gr05.jpg}
\caption{Creating a solid of revolution}
\end{center}
\end{figure}


\subsection*{The Disc Method:}

Start with a concrete example.

%%%%%%%%%%%%%%%%%%%%%%%%%%%%%%%%%%%%%%%%%%%%%%%%%%%%%%%%%
%%%%%%%%%%%%%%%%%%%%%%%%%%%%%%%%%%%%%%%%%%%%%%%%%%%%%%%%%
\vskip 2mm

\begin{example} Find the volume of the solid of revolution formed by rotating the region below $y=(x-1)^2+1$ on the interval $[-1,3]$ about the $x$-axis. 
\vskip 5mm
Start by graphing the region.

\begin{figure}[h]
\begin{center}
\includegraphics[scale=0.6]{slicing_gr06.jpg}$\xRightarrow[\text{\textcolor{red}{Create solid of revolution}}]{}$\includegraphics[scale=0.6]{slicing_gr07.jpg}
\caption{Solid of revolution created from a region below $y=(x-1)^2+1$}
\end{center}
\end{figure}

We now analyze the volume of a slice of the region.

\vfill\eject

\begin{figure}[h]
\begin{center}
\includegraphics[scale=0.6]{slicing_gr08.jpg}$\xRightarrow[\stackrel{\text{\textcolor{red}{Creating a Slice of the Region}}}{\text{\textcolor{red}{\tiny By Revolving a Rectangle}}}]{}$\includegraphics[scale=0.6]{slicing_gr09.jpg}
\caption{The slices of the 3D solid are circular discs}
\end{center}
\end{figure}

The cross-sectional area of a circular disc located at $x$ is $A(x)=\pi\big[f(x)\big]^2$. So we evaluate the following integral.

\beq
Volume&=&\pi\ds\int^{x=b}_{x=a}\big[f(x)\big]^2\,dx\\
\\
&=&\pi\ds\int^{x=3}_{x=-1}\big[(x-1)^2+1\big]^2\,dx\\
\\
&=&\pi\ds\int^{x=3}_{x=-1}(x-1)^4+2(x-1)^2+1\,dx\\
\\
&=&\pi\left[\ds\frac{(x-1)^5}{5}+\ds\frac{2}{3}(x-1)^3+x\ds\right]^{x=3}_{x=-1}\\
\\
&=&\pi\left[\left(\ds\frac{2^5}{5}+\ds\frac{2}{3}\cdot 2^3+3\right)-\left(\ds\frac{(-2)^5}{5}+\ds\frac{2}{3}\cdot(-2)^3-1\right)\right]\\
\\
&=&\ds\frac{412}{15}\pi
\eeq  

\end{example}

Now try an example on your own.

%%%%%%%%%%%%%%%%%%%%%%%%%%%%%%%%%%%%%%%%%%%%%%%%%%%%%%%%%
%%%%%%%%%%%%%%%%%%%%%%%%%%%%%%%%%%%%%%%%%%%%%%%%%%%%%%%%%
\vskip 2mm

\begin{example} Find the volume of the solid of revolution formed by rotating the region below $y=\sqrt{x}$ on the interval $[1,4]$ about the $x$-axis. 
\vskip 5mm
\noindent{\bf\emph{\underline{Workspace}:}}

\vfill\eject

\ifnum\longform=1
	\begin{boxsolution}
\vspace*{5mm}
Start by graphing the region:
\vskip 5mm
\begin{minipage}[]{6.5in}
\begin{center}
\includegraphics[scale=0.7]{slicing_gr10.jpg}$\xRightarrow[\text{\textcolor{red}{Create solid of revolution}}]{}$\includegraphics[scale=0.7]{slicing_gr11.jpg}
\captionof{figure}{Solid of revolution created from the region below $y=\sqrt{x}$ on the interval $[1,4]$.}
\label{fig:}
\end{center}
\end{minipage}
\vskip 1cm
The slices of the region are circular discs and so we simply evaluate the following integral.
\beq
Volume&=&\pi\ds\int^{x=b}_{x=a}\big[f(x)\big]^2\,dx\\
\\
&=&\pi\ds\int^{x=4}_{x=1}\big[\sqrt{x}\big]^2\,dx\\
\\
&=&\pi\ds\int^{x=4}_{x=1}x\,dx\\
\\
&=&\pi\left[\ds\frac{x^2}{2}\ds\right]^{x=4}_{x=1}\\
\\
&=&\pi\left[\ds\frac{4^2}{2}-\ds\frac{1^2}{2}\right]\\
\\
&=&\ds\frac{15}{2}\pi\\
\eeq  
\vspace*{5mm}
	\end{boxsolution}
\vskip 5mm

\fi

\end{example}


\subsection*{The Washer Method:}

Sometimes the slices will not be discs. Consider the following example where we rotate the region between $y=\sqrt{x}$ and $y=1$ on the interval $[1,4]$ about the $x$-axis.

\begin{minipage}[]{6.5in}
\begin{center}
\includegraphics[scale=0.7]{slicing_gr14.jpg}$\xRightarrow[\text{\textcolor{red}{Create solid of revolution}}]{}$\includegraphics[scale=0.7]{slicing_gr15.jpg}
\captionof{figure}{Rotate the region between $y=\sqrt{x}$ and $y=1$ on the interval $[1,4]$.}
\label{fig:}
\end{center}
\end{minipage}
\vskip 5mm
Look at the slices of the solid. They are not discs.
\vskip 5mm
\begin{minipage}[]{6.5in}
\begin{center}
\includegraphics[scale=0.7]{slicing_gr16.jpg}$\xRightarrow[\stackrel{\text{\textcolor{red}{Creating a slice of the}}}{\text{\textcolor{red}{\tiny 3D solid of revolution}}}]{}$\includegraphics[scale=0.7]{slicing_gr17.jpg}
\captionof{figure}{Slices after rotating the region between $y=\sqrt{x}$ and $y=1$ on the interval $[1,4]$.}
\label{fig:}
\end{center}
\end{minipage}
\vspace*{2mm} 

The slices are ``washers." Fortunately, the cross-sectional area of a washer is easy. 
\[A(x)=\pi\big[R(x)\big]^2-\pi\big[r(x)\big]^2=\pi\left(\big[R(x)\big]^2-\big[r(x)\big]^2\right),\]
where $R(x)$ is the larger radius to the outside edge of the washer, and $r(x)$ is the smaller radius to the inside edge of the washer. And so we simply evaluate the following integral.

\[Volume=\pi\ds\int^{x=b}_{x=a}\big[R(x)\big]^2-\big[r(x)\big]^2\,dx\]

Now try another example on your own. Be careful with the cross-sectional area.

%%%%%%%%%%%%%%%%%%%%%%%%%%%%%%%%%%%%%%%%%%%%%%%%%%%%%%%%%
%%%%%%%%%%%%%%%%%%%%%%%%%%%%%%%%%%%%%%%%%%%%%%%%%%%%%%%%%

\ifnum\longform=1
\vfill\eject

\fi

\begin{example} Find the volume of the solid of revolution formed by rotating the region between $y=x$ and $y=1/x$ on the interval $[1,4]$ about the $x$-axis. 
\vskip 5mm
\noindent{\bf\emph{\underline{Workspace}:}}

\vfill\eject

\ifnum\longform=1
	\begin{boxsolution}
\vspace*{5mm}
Start by graphing the region:
\vskip 5mm
\begin{minipage}[]{6.5in}
\begin{center}
\includegraphics[scale=0.6]{slicing_gr12.jpg}$\xRightarrow[\text{\textcolor{red}{Create solid of revolution}}]{}$\includegraphics[scale=0.7]{slicing_gr13.jpg}
\captionof{figure}{Rotate the region between $y=x$ and $y=1/x$ on the interval $[1,4]$.}
\label{fig:}
\end{center}
\end{minipage}
\vskip 1cm
The slices of the region are circular washers. The cross-section area of a washer located at point $x$ is $\pi\big[R(x)\big]^2-\pi\big[r(x)\big]^2$, where $R(x)$ is the larger radius to the outside edge of the washer, and $r(x)$ is the smaller radius to the inside edge of the washer. And so we simply evaluate the following integral.
\beq
Volume&=&\pi\ds\int^{x=b}_{x=a}\big[R(x)\big]^2-\big[r(x)\big]^2\,dx\\
\\
&=&\pi\ds\int^{x=4}_{x=1}x^2-\left(\ds\frac{1}{x}\right)^2\,dx\\
\\
&=&\pi\ds\int^{x=4}_{x=1}x^2-\ds\frac{1}{x^2}\,dx\\
\\
&=&\pi\left[\ds\frac{x^3}{3}+\ds\frac{1}{x}\ds\right]^{x=4}_{x=1}\\
\\
&=&\pi\left[\left(\ds\frac{4^3}{3}+\ds\frac{1}{4}\right)-\left(\ds\frac{1^3}{3}+\ds\frac{1}{1}\right)\right]\\
\\
&=&\ds\frac{81}{4}\pi
\eeq  
\vspace*{5mm}
	\end{boxsolution}
\vskip 5mm

\fi

\end{example}

\ifnum\longform=1
\vfill\eject

\fi

%%%%%%%%%%%%%%%%%%%%%%%%%%%%%%%%%%%%%%%%%%%%%%%%%%%%%%%%%
%%%%%%%%%%%%%%%%%%%%%%%%%%%%%%%%%%%%%%%%%%%%%%%%%%%%%%%%%

\begin{example}
We can rotate a regions about the $y$-axis as well. Suppose we have the region under $y=4-x^2$ on the interval $[0,2]$ about the $y$-axis. Since we rotate about the $y$-axis we should probably switch variables to have $x=\sqrt{4-y}$.
\vskip 5mm
\noindent{\bf\emph{\underline{Workspace}:}}

\vfill\eject

\ifnum\longform=1
	\begin{boxsolution}
\vspace*{5mm}
Start by graphing the region:
\vskip 5mm
\begin{minipage}[]{6.5in}
\begin{flushleft}
\includegraphics[scale=0.45]{slicing_gr18.jpg}$\xRightarrow[\text{\textcolor{red}{Create solid of revolution}}]{}$\includegraphics[scale=0.45]{slicing_gr19.jpg}
\captionof{figure}{Rotate the region $x=\sqrt{4-y}$ on the interval $[0,4]$.}
\label{fig:}
\end{flushleft}
\end{minipage}
\vskip 1cm
\begin{minipage}[]{6.5in}
\begin{flushleft}
\includegraphics[scale=0.45]{slicing_gr20.jpg}$\xRightarrow[\stackrel{\text{\textcolor{red}{Creating a slice of the}}}{\text{\textcolor{red}{\tiny 3D solid of revolution}}}]{}$\includegraphics[scale=0.45]{slicing_gr21.jpg}
\captionof{figure}{Slices after rotating the region under $x=\sqrt{4-y}$ on the interval $[0,4]$.}
\label{fig:}
\end{flushleft}
\end{minipage}
\vskip 5mm
The slices are discs and so $A(y)=\pi\big[r(y)\big]^2=\pi\big[\sqrt{4-y}\big]^2$. And so we simply evaluate the following integral.
\beq
Volume&=&\pi\ds\int^{y=4}_{y=0}\big[\sqrt{4-y}\big]^2\,dy\\
\\
&=&\pi\ds\int^{y=4}_{y=0}4-y\,dy\\
\\
&=&\pi\left[4y-\ds\frac{y^2}{2}\ds\right]^{y=4}_{y=0}\\
\\
&=&\pi\left[\left(16-\ds\frac{4^2}{2}\right)-\Big(0-0\Big)\right]\\
\\
&=&8\pi
\eeq  
	\end{boxsolution}

\fi

\end{example}

%%%%%%%%%%%%%%%%%%%%%%%%%%%%%%%%%%%%%%%%%%%%%%%%%%%%%%%%%
%%%%%%%%%%%%%%%%%%%%%%%%%%%%%%%%%%%%%%%%%%%%%%%%%%%%%%%%%
%%%%%%%%%%%%%%%%%%%%%%%%%%%%%%%%%%%%%%%%%%%%%%%%%%%%%%%%%
%%%%%%%%%%%%%%%%%%%%%%%%%%%%%%%%%%%%%%%%%%%%%%%%%%%%%%%%%

	
%%%%%%%%%%%%%%%%%%%%%%%%%%%%%%%%%%%%%%%%%%%%%%%%%%%%%%
%%%%%%%%%%%%%%%%%%%%%%%%%%%%%%%%%%%%%%%%%%%%%%%%%%%%%%

\ifnum\longform=1
\vskip 1cm
\hrule
\vskip 5mm
\begin{center}{\bf Please let me know if you have any questions, comments, or corrections!}
\end{center}	

\fi

%%%%%%%%%%%%%%%%%%%%%%%%%%%%%%%%%%%%%%%%%%%%%%%%%%%%%%
\end{document}
%%%%%%%%%%%%%%%%%%%%%%%%%%%%%%%%%%%%%%%%%%%%%%%%%%%%%%
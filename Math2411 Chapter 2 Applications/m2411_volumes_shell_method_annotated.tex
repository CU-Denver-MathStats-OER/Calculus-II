\documentclass[11pt]{article}
\usepackage[suffix=Solutions]{teaching-header}

\def\classnum{2411}
\def\classtitle{Calculus II}
\def\classtitleshort{Calc 2}
\def\classsec{H01}
\def\instructor{Dr. Rostermundt}
\def\classterm{Spring 2025}


%%%%%%%%%%%%%%%%%%%%%%%%%%%%%%%%%%%%%%%%%%%%%%%%%%%%%%%%%%%%%%%%%%%%%%%%%%%%
%%%%%%%%%%%%%%%%%%%%%%%%%%%%%%%%%%%%%%%%%%%%%%%%%%%%%%%%%%%%%%%%%%%%%%%%%%%%

%This is defined in the teaching-header style file
%\ifnum\printsol=0 (when no solutions printed)
%Do something
%	\else  (when solutions are printed)
%Do something else
%\fi


% Package and setting included in teachin-header style file
%\RequirePackage{amsmath,amsfonts,amssymb,amsthm,graphicx, pgfplots, tcolorbox, xcolor,latexsym,color,verbatim,float,xcolor,setspace}
%%tikzsymbols
%
%\RequirePackage{enumerate}
%\RequirePackage{multicol}
%\RequirePackage{tikz}
%\RequirePackage{cancel}
%\usetikzlibrary{shapes.geometric}
%\usetikzlibrary{calc, positioning, arrows}
%\RequirePackage[margin=1in,letterpaper]{geometry}
%\RequirePackage[colorlinks=true,allcolors=blue]{hyperref}
%\usepackage[final]{pdfpages}
%%\usepackage{capt-of}
%
%
%\setlength{\textheight}{9in}
%\setlength{\textwidth}{6.5in}
%\addtolength{\topmargin}{0cm}
%%\addtolength{\oddsidemargin}{0cm}
%\parindent=0in
%\parskip=.35em
%\singlespacing
%%\pagestyle{empty}  % remove page numbers

%Add captions without being in figure environment
%\captioof{figure}{\text}\label[fig:]
\usepackage{capt-of}
\usepackage{mathtools}

\vfuzz2pt % Don't report over-full v-boxes if over-edge is small
\hfuzz2pt % Don't report over-full h-boxes if over-edge is small


%%%%%%%%%%%%%%%%%%%%%%%%%%%%%%%%%%%%%%%%%%%%%%%%%%%%%%
%%%%%%%%%%%%%%%%%%%%%%%%%%%%%%%%%%%%%%%%%%%%%%%%%%%%%%

\pagestyle{myheadings}

%%%%%%%%%%%%%%%%%%%%%%%%%%%%%%%%%%%%%%%%%%%%%%%%%%%%%%
%%%%%%%%%%%%%%%%%%%%%%%%%%%%%%%%%%%%%%%%%%%%%%%%%%%%%%


%%%%%%%%%%%%%%%%%%%%%%%%%%%%%%%%%%%%%%%%%%%%%%%%%%%%%%
%%%%%%%%%%%%%%%%%%%%%%%%%   Document Body   %%%%%%%%%%
%%%%%%%%%%%%%%%%%%%%%%%%%%%%%%%%%%%%%%%%%%%%%%%%%%%%%%

%Information from classinfo.tex file
%\def\classnum{2411}
%\def\classtitle{Calculus II}
%\def\classtitleshort{Calc 2}
%\def\classsec{001}
%\def\instructor{Rostermundt}
%\def\classterm{Fall 2024}
\def\topic{Volumes - The Shell Method}
\def\topicshort{The Shell Method}

	\title{\vspace{-1in}Math\classnum\;-\;\classtitle\\
	%Section\;\classsec\;-\;\classterm\\
	Guided Lecture Notes\\
	\topic}
	\author{University of Colorado Denver / College of Liberal Arts and Sciences}
	\date{Department of Mathematics}

	\markright{Math\classnum\;-\;\classtitleshort, University of Colorado Denver,\;\topicshort}



%%%%%%%%%%%%%%%%%%%%%%%%%%%%%%%%%%%%%%%%%%%%%%%%%%%%%%
\begin{document}\maketitle\thispagestyle{empty}
%%%%%%%%%%%%%%%%%%%%%%%%%%%%%%%%%%%%%%%%%%%%%%%%%%%%%%

\hrule

\section*{\topic\; Introduction}

Suppose that we want to determine the volume of a 3D region/solid with a varying cross-section such as shown below.

\begin{minipage}[]{6.5in}
\begin{center}
\includegraphics[scale=0.5]{shell_method_gr01.jpg}
\captionof{figure}{3D region/solid with a varying cross-section}
\label{fig:}
\end{center}
\end{minipage}

In this section, instead of slicing and building 3D discs, we will build cylindrical shells to approximate the area of the solid of interest. To do this we will use rectangles that are parallel to the axis of revolution.  

\begin{minipage}[]{6.5in}
\begin{center}
\includegraphics[scale=0.5]{shell_method_gr02.jpg}
\captionof{figure}{Creating Cylindrical Shells}
\label{fig:}
\end{center}
\end{minipage}
\vskip 5mm

Now we need to determine the volume of a shell to complete our approximation.

\begin{minipage}[]{6.5in}
\begin{center}
\includegraphics[scale=0.5]{shell_method_gr03.jpg}
\captionof{figure}{Determining volume of a representative shell.}
\label{fig:}
\end{center}
\end{minipage}
\vskip 5mm

Multiplying base, height, and thickness of the rectangular slab we get
\[Vol_{_{Shell}}\approx 2\pi x^{*}\cdot f(x^{*})\cdot\Delta x.\]
We now have a estimation formula which leads to an integral as we let the width of the rectangles approach zero.
\[\text{Total Volume}\approx\ds\sum^{n}_{i=1}2\pi x^{*}_{i}\cdot f(x^{*}_{i})\cdot\Delta x_i\quad\xRightarrow[\text{\textcolor{red}{Let $\Delta x\to 0$}}]{}\quad\text{Total Volume}=\ds\int^{x=b}_{x=a}2\pi xf(x)\,dx\]
\vskip 2mm
Let's compute some concrete examples.

\subsection*{Shell Method Examples:}

%%%%%%%%%%%%%%%%%%%%%%%%%%%%%%%%%%%%%%%%%%%%%%%%%%%%%%%%%
%%%%%%%%%%%%%%%%%%%%%%%%%%%%%%%%%%%%%%%%%%%%%%%%%%%%%%%%%
\vskip 2mm

\begin{example} Define $R$ as the region bounded above by the graph of $f(x)=1/x$ and below by the $x$-axis over the interval
$[1,3]$. Find the volume of the solid of revolution formed by revolving $R$ around the $y$-axis
\vskip 5mm
\noindent{\bf\emph{\underline{Workspace}:}}

\vfill\eject

\ifnum\longform=1
	\begin{boxsolution}
\vspace*{5mm}
Start by graphing the region:
\vskip 5mm
\begin{minipage}[]{6.5in}
\begin{center}
\includegraphics[scale=0.7]{shell_method_gr04.jpg}
\captionof{figure}{The region $R$ and the solid of revolution.}
\label{fig:}
\end{center}
\end{minipage}
\vskip 1cm
Then the volume of the solid is given by
\beq
\ds\int^{x=b}_{x=a}2\pi xf(x)\,dx&=&2\pi\ds\int^{x=3}_{x=1}x\ds\frac{1}{x}\,dx\\
&=&2\pi\ds\int^{x=3}_{x=1}\,dx\\
&=&2\pi\Big(3-1\Big)\\
&=&4\pi
\eeq
\vspace*{5mm}
	\end{boxsolution}
\vskip 5mm

\fi

\end{example}

Now try another example on your own.

%%%%%%%%%%%%%%%%%%%%%%%%%%%%%%%%%%%%%%%%%%%%%%%%%%%%%%%%%%
%%%%%%%%%%%%%%%%%%%%%%%%%%%%%%%%%%%%%%%%%%%%%%%%%%%%%%%%%%
\vskip 2mm

\begin{example} Define $R$ as the region bounded above by the graph of $f(x)=2x-x^2$ and below by the x-axis over the interval
$[0,2]$. Find the volume of the solid of revolution formed by revolving $R$ around the $y$-axis.
\vskip 5mm
\noindent{\bf\emph{\underline{Workspace}:}}

\vfill\eject

\ifnum\longform=1
	\begin{boxsolution}
\vspace*{5mm}
Start by graphing the region:
\vskip 5mm
\begin{minipage}[]{6.5in}
\begin{center}
\includegraphics[scale=0.7]{shell_method_gr05.jpg}
\captionof{figure}{The region $R$ and the solid of revolution.}
\label{fig:}
\end{center}
\end{minipage}
\vskip 5mm
Then the volume of the solid is given by
\beq
\ds\int^{x=b}_{x=a}2\pi xf(x)\,dx&=&2\pi\ds\int^{x=2}_{x=0}x\Big(2x-x^2\big)\,dx\\
&=&2\pi\left[\ds\frac{2}{3}x^3-\ds\frac{1}{4}x^4\ds\right]^{x=2}_{x=0}\\
&=&2\pi\left[\left(\ds\frac{2}{3}\cdot 8-\ds\frac{1}{4}\cdot 16\right)-\Big(0-0\Big)\right]\\
&=&\ds\frac{8\pi}{3}
\eeq
\vspace*{5mm}
	\end{boxsolution}
\vskip 5mm

\fi

\end{example}

Let's consider another example.

%%%%%%%%%%%%%%%%%%%%%%%%%%%%%%%%%%%%%%%%%%%%%%%%%%%%%%%%%
%%%%%%%%%%%%%%%%%%%%%%%%%%%%%%%%%%%%%%%%%%%%%%%%%%%%%%%%%
\vskip 2mm

\begin{example} Define $Q$ as the region bounded on the right by the graph of $f(x)=x^2/4$, the line $y=4$, and on the left by the $y$-axis. The solid of revolution will be formed by revolving $Q$ around the $x$-axis. If we use the slicing method we end up with washers. If we want to use shells we can rewrite everything in terms of the variable $y$. So we have the reqion bounded by $g(y)=2\sqrt{y}$, the line $y=4$, and on the left by the $y$-axis. Find the volume of the solid of revolution formed by revolving $Q$ around the $x$-axis
\vskip 5mm
\noindent{\bf\emph{\underline{Workspace}:}}

\vfill\eject

\ifnum\longform=1
	\begin{boxsolution}
\vspace*{5mm}
Start by graphing the region:
\vskip 5mm
\begin{minipage}[]{6.5in}
\begin{center}
\includegraphics[scale=0.45]{shell_method_gr07.jpg}
\captionof{figure}{The region $Q$ and the solid of revolution.}
\label{fig:}
\end{center}
\end{minipage}
\vskip 1cm
Then the volume of the solid is given by
\beq
\ds\int^{y=d}_{y=c}2\pi yg(y)\,dy&=&2\pi\ds\int^{y=4}_{y=0}y\Big(2\sqrt{y}\Big)\,dy\\
&=&4\pi\ds\int^{y=4}_{y=0}y^{3/2}\,dy\\
&=&4\pi\left[\ds\frac{2}{5}y^{5/2}\ds\right]^{y=4}_{y=0}\\
&=&\ds\frac{256\pi}{5}
\eeq
\vspace*{5mm}
	\end{boxsolution}
\vskip 5mm

\fi

\end{example}

We consider one more final example where the region is bounded between two curves.

%%%%%%%%%%%%%%%%%%%%%%%%%%%%%%%%%%%%%%%%%%%%%%%%%%%%%%
%%%%%%%%%%%%%%%%%%%%%%%%%%%%%%%%%%%%%%%%%%%%%%%%%%%%%%
\vskip 2mm

\begin{example} Define $R$ as the region bounded above by the graph of the function $f(x)=\sqrt{x}$ and below by the graph of the
function $g(x)=1/x$ over the interval $[1,4]$. Find the volume of the solid of revolution generated by revolving
$R$ around the $y$-axis.
\vskip 5mm
\noindent{\bf\emph{\underline{Workspace}:}}

\ifnum\longform=1

\vfill\eject

	\begin{boxsolution}
\vspace*{5mm}
Start by graphing the region:
\vskip 5mm
\begin{minipage}[]{6.5in}
\begin{flushleft}
\includegraphics[scale=0.65]{shell_method_gr09.jpg}
\captionof{figure}{The region $R$ and the solid of revolution.}
\label{fig:}
\end{flushleft}
\end{minipage}
\vskip 1cm
Then the volume of the solid is easily calculated.
\begin{minipage}[]{6.5in}
\begin{flushleft}
\includegraphics[trim= 0.3cm 0cm 0cm 0cm, clip=true, scale=0.75]{shell_method_gr10.jpg}
%\captionof{figure}{}
\label{fig:}
\end{flushleft}
\end{minipage}
\vspace*{5mm}
	\end{boxsolution}
\vskip 5mm

\fi

\end{example}

\vspace*{1cm}


%%%%%%%%%%%%%%%%%%%%%%%%%%%%%%%%%%%%%%%%%%%%%%%%%%%%%%%%%
%%%%%%%%%%%%%%%%%%%%%%%%%%%%%%%%%%%%%%%%%%%%%%%%%%%%%%%%%
%%%%%%%%%%%%%%%%%%%%%%%%%%%%%%%%%%%%%%%%%%%%%%%%%%%%%%%%%
%%%%%%%%%%%%%%%%%%%%%%%%%%%%%%%%%%%%%%%%%%%%%%%%%%%%%%%%%

	
%%%%%%%%%%%%%%%%%%%%%%%%%%%%%%%%%%%%%%%%%%%%%%%%%%%%%%
%%%%%%%%%%%%%%%%%%%%%%%%%%%%%%%%%%%%%%%%%%%%%%%%%%%%%%

\ifnum\longform=1
\vskip 5mm
\hrule
\vskip 5mm
\begin{center}{\bf Please let me know if you have any questions, comments, or corrections!}
\end{center}	

\fi

%%%%%%%%%%%%%%%%%%%%%%%%%%%%%%%%%%%%%%%%%%%%%%%%%%%%%%
\end{document}
%%%%%%%%%%%%%%%%%%%%%%%%%%%%%%%%%%%%%%%%%%%%%%%%%%%%%%
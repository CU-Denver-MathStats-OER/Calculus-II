\documentclass[11pt]{article}
\usepackage[suffix=Solutions]{teaching-header}

\def\classnum{2411}
\def\classtitle{Calculus II}
\def\classtitleshort{Calc 2}
\def\classsec{H01}
\def\instructor{Dr. Rostermundt}
\def\classterm{Spring 2025}


%%%%%%%%%%%%%%%%%%%%%%%%%%%%%%%%%%%%%%%%%%%%%%%%%%%%%%%%%%%%%%%%%%%%%%%%%%%%
%%%%%%%%%%%%%%%%%%%%%%%%%%%%%%%%%%%%%%%%%%%%%%%%%%%%%%%%%%%%%%%%%%%%%%%%%%%%

%This is defined in the teaching-header style file
%\ifnum\printsol=0 (when no solutions printed)
%Do something
%	\else  (when solutions are printed)
%Do something else
%\fi


% Package and setting included in teachin-header style file
%\RequirePackage{amsmath,amsfonts,amssymb,amsthm,graphicx, pgfplots, tcolorbox, xcolor,latexsym,color,verbatim,float,xcolor,setspace}
%%tikzsymbols
%
%\RequirePackage{enumerate}
%\RequirePackage{multicol}
%\RequirePackage{tikz}
%\RequirePackage{cancel}
%\usetikzlibrary{shapes.geometric}
%\usetikzlibrary{calc, positioning, arrows}
%\RequirePackage[margin=1in,letterpaper]{geometry}
%\RequirePackage[colorlinks=true,allcolors=blue]{hyperref}
%\usepackage[final]{pdfpages}
%%\usepackage{capt-of}
%
%
%\setlength{\textheight}{9in}
%\setlength{\textwidth}{6.5in}
%\addtolength{\topmargin}{0cm}
%%\addtolength{\oddsidemargin}{0cm}
%\parindent=0in
%\parskip=.35em
%\singlespacing
%%\pagestyle{empty}  % remove page numbers

%Add captions without being in figure environment
%\captioof{figure}{\text}\label[fig:]
\usepackage{capt-of}
\usepackage{mathtools}

\vfuzz2pt % Don't report over-full v-boxes if over-edge is small
\hfuzz2pt % Don't report over-full h-boxes if over-edge is small


%%%%%%%%%%%%%%%%%%%%%%%%%%%%%%%%%%%%%%%%%%%%%%%%%%%%%%
%%%%%%%%%%%%%%%%%%%%%%%%%%%%%%%%%%%%%%%%%%%%%%%%%%%%%%

\pagestyle{myheadings}

%%%%%%%%%%%%%%%%%%%%%%%%%%%%%%%%%%%%%%%%%%%%%%%%%%%%%%
%%%%%%%%%%%%%%%%%%%%%%%%%%%%%%%%%%%%%%%%%%%%%%%%%%%%%%


%%%%%%%%%%%%%%%%%%%%%%%%%%%%%%%%%%%%%%%%%%%%%%%%%%%%%%
%%%%%%%%%%%%%%%%%%%%%%%%%   Document Body   %%%%%%%%%%
%%%%%%%%%%%%%%%%%%%%%%%%%%%%%%%%%%%%%%%%%%%%%%%%%%%%%%

%Information from classinfo.tex file
%\def\classnum{2411}
%\def\classtitle{Calculus II}
%\def\classtitleshort{Calc 2}
%\def\classsec{001}
%\def\instructor{Rostermundt}
%\def\classterm{Fall 2024}
\def\topic{Area Between Curves}
\def\topicshort{Area Between Curves}

	\title{\vspace{-1in}Math\classnum\;-\;\classtitle\\
	%Section\;\classsec\;-\;\classterm\\
	Guided Lecture Notes\\
	\topic}
	\author{University of Colorado Denver / College of Liberal Arts and Sciences}
	\date{Department of Mathematics}

	\markright{Math\classnum\;-\;\classtitleshort, University of Colorado Denver,\;\topicshort}


%%%%%%%%%%%%%%%%%%%%%%%%%%%%%%%%%%%%%%%%%%%%%%%%%%%%%%
\begin{document}\maketitle\thispagestyle{empty}
%%%%%%%%%%%%%%%%%%%%%%%%%%%%%%%%%%%%%%%%%%%%%%%%%%%%%%

\hrule

\section*{\topic\; Introduction}

Suppose that we want to determine the area between two curves $y=f(x)$ and $y=g(x)$ on an
interval $[a,b]$.

\begin{figure}[h]
\begin{center}
\includegraphics[scale=0.6]{area_between_curves_gr1.jpg}
\caption{Region between graphs $y=f(x)$ and $y=g(x)$ on the interval $[a,b]$}
\end{center}
\end{figure}

It is very natural to approach the problem by subtracting the area below the lower graph
(in this example $y=g(x)$) from the area under the upper graph (in this example $y=f(x)$)
to have
\[\text{Area}=\ds\int^{x=b}_{x=a}f(x)\,dx-\ds\int^{x=b}_{x=a}g(x)\,dx\quad\xRightarrow[\stackrel{\text{\textcolor{red}{By Properties}}}{\text{\textcolor{red}{of Integrals}}}]{}\quad\text{Area}=\ds\int^{x=b}_{x=a}f(x)-g(x)\,dx.\]

Another more general approach is to create a familiar type estimate of the area using
rectangles.

\begin{figure}[h]
\begin{center}
\includegraphics[scale=0.6]{area_between_curves_gr2.jpg}
\caption{Estimated area between graphs $y=f(x)$ and $y=g(x)$ on the interval $[a,b]$.}
\end{center}
\end{figure}

If we improve the estimate by letting the number of rectangles increase we have the same
formula that we just saw above.

\begin{figure}[h]
\begin{center}
\includegraphics[scale=0.6]{area_between_curves_gr3.jpg}
\end{center}
\end{figure}
%\vspace*{5mm}
If we label the upper graph as $y=upper(x)$ and the lower graph as $y=lower(x)$, we can
rewrite the formula 2.1 and also interpret the integral geometrically (as we’ve seen before)
as a sum of the areas of ``very, very, thin" rectangles.

\[\ds\int^{x=b}_{x=a}\big[upper(x)-lower(x)\big]\,dx\quad\xRightarrow[]{\text{\textcolor{red}{Can Be Interpreted As}}}\quad\stackrel{\stackrel{\text{\textcolor{red}{\large Sum}}}{\downarrow}}{\int^{x=b}_{x=a}}\;\underbrace{\stackrel{\stackrel{\text{\textcolor{red}{Rectangle Height}}}{\downarrow}}{\;\big[upper(x)-lower(x)\big]\;}\;\;\stackrel{\stackrel{\text{\textcolor{red}{Rectangle Width}}}{\downarrow}}{dx}}_{\text{\textcolor{blue}{Rectangle Area}}}\]


%%%%%%%%%%%%%%%%%%%%%%%%%%%%%%%%%%%%%%%%%%%%%%%%%%%%%%%%%
%%%%%%%%%%%%%%%%%%%%%%%%%%%%%%%%%%%%%%%%%%%%%%%%%%%%%%%%%


\section*{Area Between Curves Examples:}

Let’s work some examples.
\vskip 2mm

\begin{example}

Find the area between the curves $y=9-x^2/4$ and $y=6-x$.

\begin{figure}[h]
\begin{center}
\includegraphics[scale=0.6]{area_between_curves_gr4.jpg}
\caption{Region between graphs $y=9-x^2/4$ and $y=6-x$.}
\end{center}
\end{figure}
\vspace*{5mm}
Start by finding the interval. At which $x$-values do the curves intersect?

\vfill\eject

We solve the following equation.

\beq
9-x^2/4&=&6-x\\
x^2-4x-12&=&0\\
(x-6)(x+2)&=&0
\eeq

So the interval will be $[-2,6]$. On this interval the upper graph is $y=9-x^2/4$ and the
lower graph is $y=6-x$ and so the integral is
\[\text{Area}=\ds\int^{x=b}_{x=a}\big[upper(x)-lower(x)\big]\,dx=\ds\int^{x=6}_{x=-2}\left[\left(9-\ds\frac{x^2}{4}\right)-\Big(6-x\Big)\right]\,dx\]
We evaluate the integral.

\beq
\ds\int^{x=6}_{x=-2}\left[\left(9-\ds\frac{x^2}{4}\right)-\Big(6-x\Big)\right]\,dx&=&\ds\int^{x=6}_{x=-2}\left[-\ds\frac{x^2}{4}+x+3\right]\,dx\\
\\
&=&-\ds\frac{x^3}{12}+\ds\frac{x^2}{2}+3x\ds\Bigg|^{x=6}_{x=-2}\\
\\
&=&\left(-\ds\frac{6^3}{12}+\ds\frac{6^2}{2}+3\cdot 6\right)-\left(-\ds\frac{(-2)^3}{12}+\ds\frac{(-2)^2}{2}+3\cdot(-2)\right)\\
\\
&=&\ds\frac{64}{3}
\eeq

So the area between the two graphs is Area=64=3. 
 
\end{example}

Let’s have you try an example on your own.


%%%%%%%%%%%%%%%%%%%%%%%%%%%%%%%%%%%%%%%%%%%%%%%%%%%%%%%%%
%%%%%%%%%%%%%%%%%%%%%%%%%%%%%%%%%%%%%%%%%%%%%%%%%%%%%%%%%

\begin{example}

Find the area between the curves $y=x+4$ and $y=3-x/2$ on the interval $[1,4]$.
\vskip 5mm
\noindent{\bf\emph{\underline{Workspace}:}}

\vfill\eject

\ifnum\longform=1
	\begin{boxsolution}
\vspace*{5mm}
\begin{minipage}[]{6.5in}
\begin{center}
\includegraphics[scale=0.4]{area_between_curves_gr5.jpg}
\captionof{figure}{Region between graphs $y=x+4$ and $y=3-x/2$ on interval $[1,4]$.}
\label{fig:}
\end{center}
\end{minipage}
\vskip 5mm
Since the interval $[1,4]$ is given we need to identify the upper graph and lower
graph. Then we have to solve the integral
\beq
\ds\int^{x=b}_{x=a}\big[upper(x)-lower(x)\big]\,dx&=&\ds\int^{x=4}_{x=1}\left[\left(3-\ds\frac{x}{2}\right)-\Big(x+4\Big)\right]\,dx\\
\\
&=&\ds\int^{x=4}_{x=1}\left[1+\ds\frac{3x}{2}\right]\,dx\\
\\
&=&x+\ds\frac{3x^2}{4}\ds\Bigg|^{x=4}_{x=1}\\
\\
&=&\left(4+\ds\frac{3\cdot 4^2}{4}\right)-\left(1+\ds\frac{3\cdot 1^2}{4}\right)\\
\\
&=&\ds\frac{57}{4}
\eeq

So the area between the two graphs is Area$=57/4$. Let’s have you try another example on your own.
\vspace*{5mm}
	\end{boxsolution}
\vskip 5mm

\fi

\end{example}



%%%%%%%%%%%%%%%%%%%%%%%%%%%%%%%%%%%%%%%%%%%%%%%%%%%%%%%%%
%%%%%%%%%%%%%%%%%%%%%%%%%%%%%%%%%%%%%%%%%%%%%%%%%%%%%%%%%
\vskip 5mm

\begin{example}

Find the area between the curves $y=\sin(x)$ and $y=\cos(x)$ on the interval $[0,\pi]$.
\vskip 5mm
\noindent{\bf\emph{\underline{Workspace}:}}

\vfill\eject

\ifnum\longform=1
	\begin{boxsolution}
\vspace*{5mm}
\begin{minipage}[]{6.5in}
\begin{center}
\includegraphics[scale=0.45]{area_between_curves_gr6.jpg}
\captionof{figure}{Region between graphs $y=\sin(x)$ and $y=\cos(x)$ on interval $[0,\pi]$.}
\label{fig:}
\end{center}
\end{minipage}
\vskip 5mm
Since the interval $[0,\pi]$ is given we need to identify the upper graph and lower
graph. But since the graphs cross we consider two intervals. Where do the graphs cross?
We solve $\sin(x)=\cos(x)$ when $0\le x\le\pi$. They functions are equal at $x=\pi/4$ and
so we have two intervals $[0,\pi/4]$ and $[\pi/4,\pi]$. On $[0,\pi/4]$ we have the upper function as
$y=\cos(x)$ and the lower function as $y=\sin(x)$. On the second interval $[\pi/4,\pi]$ we have
the upper function as $y=\sin(x)$ and the lower function as $y=\cos(x)$. We will need two
integrals.
\vskip 5mm
We evaluate as follows.
{\small
\beq
\text{Area}&=&\ds\int^{x=\pi/4}_{x=0}\big[\cos(x)-\sin(x)\big]\,dx+\ds\int^{x=\pi}_{x=\pi/4}\big[\sin(x)-\cos(x)\big]\,dx\\
\\
&=&\Big[\sin(x)+\cos(x)\ds\Big]^{x=\pi/4}_{x=0}+\Big[-\cos(x)-\sin(x)\ds\Big]^{x=\pi}_{x=\pi/4}\\
\\
&=&\left[\left(\sin\left(\ds\frac{\pi}{4}\right)+\left(\ds\frac{\pi}{4}\right)\right)-\Big(\sin(0)+\cos(0)\Big)\right]+\left[\Big(-\cos(\pi)-\sin(\pi)\Big)-\Big(-\cos\left(\ds\frac{\pi}{4}\right)-\sin\left(\ds\frac{\pi}{4}\right)\Big)\right]\\
\\
&=&\left[\left(\ds\frac{\sqrt{2}}{2}+\ds\frac{\sqrt{2}}{2}\right)-\Big(0+1\Big)\right]+\left[\Big(-(-1)-0\Big)-\left(-\ds\frac{\sqrt{2}}{2}-\ds\frac{\sqrt{2}}{2}\right)\right]\\
\\
&=&\Big[\sqrt{2}-1\Big]+\Big[1+\sqrt{2}\Big]\\
\\
&=&2\sqrt{2}
\eeq
}
So the area between the two graphs is Area$=2\sqrt{2}$. 
\vspace*{5mm}
	\end{boxsolution}
\vskip 5mm

\fi

\end{example}

Let’s have you try another example on your own.
\vskip 5mm

%%%%%%%%%%%%%%%%%%%%%%%%%%%%%%%%%%%%%%%%%%%%%%%%%%%%%%%%%
%%%%%%%%%%%%%%%%%%%%%%%%%%%%%%%%%%%%%%%%%%%%%%%%%%%%%%%%%


\begin{example}
Find the area between the curves $y=\sqrt{x}$ and $y=3/2-x/2$ and $y=0$.
\vskip 5mm
\noindent{\bf\emph{\underline{Workspace}:}}

\vfill\eject

\ifnum\longform=1

\noindent{\bf\emph{\underline{Workspace Cont.}:}}

\vfill\eject

	\begin{boxsolution}
\vspace*{5mm}
\begin{minipage}[]{6.5in}
\begin{center}
\includegraphics[scale=0.7]{area_between_curves_gr7.jpg}
\captionof{figure}{Region between graphs $y=\sqrt{x}$ and $y=3/2-x/2$ and $y=0$.}
\label{fig:}
\end{center}
\end{minipage}

You might notice that integrating with respect to x will require two integrals because the
graphs cross. How about using the variable $y$. Look at the following diagram.

\begin{minipage}[]{6.5in}
\begin{center}
\includegraphics[scale=0.6]{area_between_curves_gr8.jpg}$\quad\Longrightarrow\quad$\includegraphics[scale=0.6]{area_between_curves_gr9.jpg}
\captionof{figure}{Estimating the area of a region using the variable $y$.}
\label{fig:}
\end{center}
\end{minipage}

If we write the equations in terms of $y$ we have $y=\sqrt{x}\iff x=y^2$ and $y=3/2-x/2\iff x=3-2y$. The intersection point of the two graphs is $(x,y)=(1,1)$ we can now solve the single integral.
\beq
\text{Area}&=&\ds\int^{y=d}_{y=c}\big[upper(y)-lower(y)\big]\,dy\\
\\
&=&\ds\int^{y=1}_{y=0}\left[\Big(3-2y\Big)-\Big(y^2\Big)\right]\,dy\\
\\
&=&3y-y^2-\ds\frac{y^3}{3}\ds\Bigg|^{y=1}_{y=0}\\
\\
&=&\ds\frac{5}{3}
\eeq
So we have Area$=5/3$.
\vspace*{5mm}
	\end{boxsolution}
\vskip 5mm

\fi

\end{example}

%%%%%%%%%%%%%%%%%%%%%%%%%%%%%%%%%%%%%%%%%%%%%%%%%%%%%%%%%
%%%%%%%%%%%%%%%%%%%%%%%%%%%%%%%%%%%%%%%%%%%%%%%%%%%%%%%%%
%%%%%%%%%%%%%%%%%%%%%%%%%%%%%%%%%%%%%%%%%%%%%%%%%%%%%%%%%
%%%%%%%%%%%%%%%%%%%%%%%%%%%%%%%%%%%%%%%%%%%%%%%%%%%%%%%%%

	
%%%%%%%%%%%%%%%%%%%%%%%%%%%%%%%%%%%%%%%%%%%%%%%%%%%%%%
%%%%%%%%%%%%%%%%%%%%%%%%%%%%%%%%%%%%%%%%%%%%%%%%%%%%%%

\ifnum\longform=1
\vskip 1cm
\hrule
\vskip 5mm
\begin{center}{\bf Please let me know if you have any questions, comments, or corrections!}
\end{center}	

\fi

%%%%%%%%%%%%%%%%%%%%%%%%%%%%%%%%%%%%%%%%%%%%%%%%%%%%%%
\end{document}
%%%%%%%%%%%%%%%%%%%%%%%%%%%%%%%%%%%%%%%%%%%%%%%%%%%%%%
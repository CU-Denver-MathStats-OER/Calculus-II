\documentclass[11pt]{article}
\usepackage[suffix=Solutions]{teaching-header}

\def\classnum{2411}
\def\classtitle{Calculus II}
\def\classtitleshort{Calc 2}
\def\classsec{H01}
\def\instructor{Dr. Rostermundt}
\def\classterm{Spring 2025}


%%%%%%%%%%%%%%%%%%%%%%%%%%%%%%%%%%%%%%%%%%%%%%%%%%%%%%%%%%%%%%%%%%%%%%%%%%%%
%%%%%%%%%%%%%%%%%%%%%%%%%%%%%%%%%%%%%%%%%%%%%%%%%%%%%%%%%%%%%%%%%%%%%%%%%%%%

%This is defined in the teaching-header style file
%\ifnum\printsol=0 (when no solutions printed)
%Do something
%	\else  (when solutions are printed)
%Do something else
%\fi


% Package and setting included in teachin-header style file
%\RequirePackage{amsmath,amsfonts,amssymb,amsthm,graphicx, pgfplots, tcolorbox, xcolor,latexsym,color,verbatim,float,xcolor,setspace}
%%tikzsymbols
%
%\RequirePackage{enumerate}
%\RequirePackage{multicol}
%\RequirePackage{tikz}
%\RequirePackage{cancel}
%\usetikzlibrary{shapes.geometric}
%\usetikzlibrary{calc, positioning, arrows}
%\RequirePackage[margin=1in,letterpaper]{geometry}
%\RequirePackage[colorlinks=true,allcolors=blue]{hyperref}
%\usepackage[final]{pdfpages}
%%\usepackage{capt-of}
%
%
%\setlength{\textheight}{9in}
%\setlength{\textwidth}{6.5in}
%\addtolength{\topmargin}{0cm}
%%\addtolength{\oddsidemargin}{0cm}
%\parindent=0in
%\parskip=.35em
%\singlespacing
%%\pagestyle{empty}  % remove page numbers

%Add captions without being in figure environment
%\captioof{figure}{\text}\label[fig:]
\usepackage{capt-of}
\usepackage{mathtools}

\vfuzz2pt % Don't report over-full v-boxes if over-edge is small
\hfuzz2pt % Don't report over-full h-boxes if over-edge is small


%%%%%%%%%%%%%%%%%%%%%%%%%%%%%%%%%%%%%%%%%%%%%%%%%%%%%%
%%%%%%%%%%%%%%%%%%%%%%%%%%%%%%%%%%%%%%%%%%%%%%%%%%%%%%

\pagestyle{myheadings}

%%%%%%%%%%%%%%%%%%%%%%%%%%%%%%%%%%%%%%%%%%%%%%%%%%%%%%
%%%%%%%%%%%%%%%%%%%%%%%%%%%%%%%%%%%%%%%%%%%%%%%%%%%%%%


%%%%%%%%%%%%%%%%%%%%%%%%%%%%%%%%%%%%%%%%%%%%%%%%%%%%%%
%%%%%%%%%%%%%%%%%%%%%%%%%   Document Body   %%%%%%%%%%
%%%%%%%%%%%%%%%%%%%%%%%%%%%%%%%%%%%%%%%%%%%%%%%%%%%%%%

%Information from classinfo.tex file
%\def\classnum{2411}
%\def\classtitle{Calculus II}
%\def\classtitleshort{Calc 2}
%\def\classsec{001}
%\def\instructor{Rostermundt}
%\def\classterm{Fall 2024}
\def\topic{Arclength}
\def\topicshort{Arclength}

	\title{\vspace{-1in}Math\classnum\;-\;\classtitle\\
	%Section\;\classsec\;-\;\classterm\\
	Guided Lecture Notes\\
	\topic}
	\author{University of Colorado Denver / College of Liberal Arts and Sciences}
	\date{Department of Mathematics}

	\markright{Math\classnum\;-\;\classtitleshort, University of Colorado Denver,\;\topicshort}


%%%%%%%%%%%%%%%%%%%%%%%%%%%%%%%%%%%%%%%%%%%%%%%%%%%%%%
\begin{document}\maketitle\thispagestyle{empty}
%%%%%%%%%%%%%%%%%%%%%%%%%%%%%%%%%%%%%%%%%%%%%%%%%%%%%%

\hrule

\section*{\topic\; Introduction}

Suppose that we want to determine the arclength along a 2D curve $y=f(x)$ where $a\le x\le b$. We start with an estimation using straight lines.

\begin{figure}[h]
\begin{center}
\includegraphics[scale=0.7]{arclength_gr1.jpg}
\caption{Estimating arclength using straight line segments}
\end{center}
\end{figure}

The total arclength will be approximated as the sum of the lengths of all line segments. If we write $\Delta s_k$ for the actual length along the $k^{th}$ segment of the curve and $l_k$ the length of the $k^{th}$ line segment
we have
\[Arclength\approx\ds\sum^{n}_{k=1}l_k\]

\begin{figure}[h!]
\begin{center}
\includegraphics[scale=0.6]{arclength_gr2.jpg}
\caption{Length of the $i^{th}$ line segment}
\end{center}
\end{figure}

Our approximation can be written as follows.
 
\beq
\Delta s_k&\approx&\sqrt{\big(\Delta x_k\big)^2+\big(\Delta y_k\big)^2}\\
&=&\sqrt{1+\left(\ds\frac{\Delta y_k}{\Delta x_k}\right)^2}\,\Delta x_k\\
&=&\sqrt{1+\left(f'(x^{*}_k\right)^2}\,\Delta x_k\qquad\text{by Mean Value Theorem}
\eeq

So we have
\[\text{Total Length}\approx\ds\sum^{n}_{k=1}\sqrt{1+\left(f'(x^{*}_k\right)^2}\,\Delta x_k\quad\xRightarrow[\text{\textcolor{red}{Let $\Delta x\to 0$}}]{\stackrel{\text{\textcolor{red}{Approximation}}}{\text{\textcolor{red}{\tiny Improves}}}}\quad \text{Total Length}=\ds\int^{x=b}_{x=a}\sqrt{1+\big[f'(x)\big]^2}\,dx.\]

As $\Delta x\to 0$ we have $l_k\to\Delta s_k$ and $\Delta s_k\to 0$ and an improved approximation. So we can write the differential
\[ds=\sqrt{1+\left(f'(x^{*}_k\right)^2}\,dx\]
and the integral can be written in two different forms as
\[s=\ds\int^{x=b}_{x=a}\sqrt{1+\big[f'(x)\big]^2}\,dx\quad\text{or}\quad s=\ds\int_{\mathcal{C}}ds.\]


Start with a concrete example.

%%%%%%%%%%%%%%%%%%%%%%%%%%%%%%%%%%%%%%%%%%%%%%%%%%%%%%%%%
%%%%%%%%%%%%%%%%%%%%%%%%%%%%%%%%%%%%%%%%%%%%%%%%%%%%%%%%%
\vskip 2mm

\begin{example} Find the arclength on the curve $y=2x^{3/2}$ when $0\le x\le 1$. 
\vskip 5mm
\noindent{\bf\emph{\underline{Workspace}:}}

\vfill\eject

\ifnum\longform=1
	\begin{boxsolution}
\vspace*{5mm}
While not necessary, we can start by graphing the curve.

\begin{minipage}[]{6.5in}
\begin{center}
\includegraphics[scale=0.6]{arclength_gr3.jpg}
\captionof{figure}{Graph of $y=2x^{3/2}$ on interval $[0,1]$.}
\label{fig:}
\end{center}
\end{minipage}

We can now set up the integral.

\vfill\eject

\beq
s&=&\ds\int^{x=b}_{x=a}\sqrt{1+\big[f'(x)\big]^2}\,dx\\
\\
&=&\ds\int^{x=1}_{x=0}\sqrt{1+\left[3x^{1/2}\right]^2}\,dx\\
\\
&=&\ds\int^{x=1}_{x=0}\sqrt{1+9x}\,dx\\
\\
&=&\ds\frac{1}{9}\ds\int^{u=10}_{u=1}\sqrt{u}\,du\quad\big(\text{Letting }u=1+9x\big)\\
\\
&=&\ds\frac{1}{9}\cdot\ds\frac{2}{3}u^{3/2}\,\ds\Bigg|^{u=10}_{u=1}\\
\\
&=&\ds\frac{2}{27}\Big[\,10\sqrt{10}-1\,\Big]\\
\eeq
\vspace*{5mm}
	\end{boxsolution}
\vskip 5mm

\fi

\end{example}


Now try setting up an example on your own.

%%%%%%%%%%%%%%%%%%%%%%%%%%%%%%%%%%%%%%%%%%%%%%%%%%%%%%%%%%
%%%%%%%%%%%%%%%%%%%%%%%%%%%%%%%%%%%%%%%%%%%%%%%%%%%%%%%%%%
\vskip 5mm

\begin{example} Find the arclength on the curve $y=\sin(3x)$ when $0\le x\le \pi$.
\vskip 5mm
\noindent{\bf\emph{\underline{Workspace}:}}

\vfill\eject

\ifnum\longform=1

\noindent{\bf\emph{\underline{Workspace Cont}:}}

\vspace*{3in}

	\begin{boxsolution}
\vspace*{5mm}
We set up the integral.
\[s=\ds\int^{x=\pi}_{x=0}\sqrt{1+\big[3\cos(3x)\big]^2}\,dx=\ds\int^{x=\pi}_{x=0}\sqrt{1+9\cos^2(3x)}\,dx\approx 6.9872 .\] 
Usually the integral will be too difficult to evaluate by hand and so we will simply set up the integral and allow a computer to estimate the value of the integral.
\vspace*{5mm}
	\end{boxsolution}
\vskip 5mm

\fi

\end{example}


%%%%%%%%%%%%%%%%%%%%%%%%%%%%%%%%%%%%%%%%%%%%%%%%%%%%%%%%%
%%%%%%%%%%%%%%%%%%%%%%%%%%%%%%%%%%%%%%%%%%%%%%%%%%%%%%%%%
%%%%%%%%%%%%%%%%%%%%%%%%%%%%%%%%%%%%%%%%%%%%%%%%%%%%%%%%%
%%%%%%%%%%%%%%%%%%%%%%%%%%%%%%%%%%%%%%%%%%%%%%%%%%%%%%%%%

	
%%%%%%%%%%%%%%%%%%%%%%%%%%%%%%%%%%%%%%%%%%%%%%%%%%%%%%
%%%%%%%%%%%%%%%%%%%%%%%%%%%%%%%%%%%%%%%%%%%%%%%%%%%%%%


\ifnum\longform=1
\vskip 1cm
\hrule
\vskip 5mm
\begin{center}
{\bf Please let me know if you have any questions, comments, or corrections!}
\end{center}	

\fi


%%%%%%%%%%%%%%%%%%%%%%%%%%%%%%%%%%%%%%%%%%%%%%%%%%%%%%
\end{document}
%%%%%%%%%%%%%%%%%%%%%%%%%%%%%%%%%%%%%%%%%%%%%%%%%%%%%%
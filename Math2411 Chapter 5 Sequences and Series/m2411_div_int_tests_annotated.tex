\documentclass[11pt]{article}
\usepackage[suffix=Solutions]{teaching-header}

\def\classnum{2411}
\def\classtitle{Calculus II}
\def\classtitleshort{Calc 2}
\def\classsec{H01}
\def\instructor{Dr. Rostermundt}
\def\classterm{Spring 2025}


%%%%%%%%%%%%%%%%%%%%%%%%%%%%%%%%%%%%%%%%%%%%%%%%%%%%%%%%%%%%%%%%%%%%%%%%%%%%
%%%%%%%%%%%%%%%%%%%%%%%%%%%%%%%%%%%%%%%%%%%%%%%%%%%%%%%%%%%%%%%%%%%%%%%%%%%%

%This is defined in the teaching-header style file
%\ifnum\printsol=0 (when no solutions printed)
%Do something
%	\else  (when solutions are printed)
%Do something else
%\fi


% Package and setting included in teachin-header style file
%\RequirePackage{amsmath,amsfonts,amssymb,amsthm,graphicx, pgfplots, tcolorbox, xcolor,latexsym,color,verbatim,float,xcolor,setspace}
%%tikzsymbols
%
%\RequirePackage{enumerate}
%\RequirePackage{multicol}
%\RequirePackage{tikz}
%\RequirePackage{cancel}
%\usetikzlibrary{shapes.geometric}
%\usetikzlibrary{calc, positioning, arrows}
%\RequirePackage[margin=1in,letterpaper]{geometry}
%\RequirePackage[colorlinks=true,allcolors=blue]{hyperref}
%\usepackage[final]{pdfpages}
%%\usepackage{capt-of}
%
%
%\setlength{\textheight}{9in}
%\setlength{\textwidth}{6.5in}
%\addtolength{\topmargin}{0cm}
%%\addtolength{\oddsidemargin}{0cm}
%\parindent=0in
%\parskip=.35em
%\singlespacing
%%\pagestyle{empty}  % remove page numbers

%Add captions without being in figure environment
%\captioof{figure}{\text}\label[fig:]
\usepackage{capt-of}
\usepackage{mathtools}

\vfuzz2pt % Don't report over-full v-boxes if over-edge is small
\hfuzz2pt % Don't report over-full h-boxes if over-edge is small


%%%%%%%%%%%%%%%%%%%%%%%%%%%%%%%%%%%%%%%%%%%%%%%%%%%%%%
%%%%%%%%%%%%%%%%%%%%%%%%%%%%%%%%%%%%%%%%%%%%%%%%%%%%%%

\pagestyle{myheadings}

%%%%%%%%%%%%%%%%%%%%%%%%%%%%%%%%%%%%%%%%%%%%%%%%%%%%%%
%%%%%%%%%%%%%%%%%%%%%%%%%%%%%%%%%%%%%%%%%%%%%%%%%%%%%%


%%%%%%%%%%%%%%%%%%%%%%%%%%%%%%%%%%%%%%%%%%%%%%%%%%%%%%
%%%%%%%%%%%%%%%%%%%%%%%%%   Document Body   %%%%%%%%%%
%%%%%%%%%%%%%%%%%%%%%%%%%%%%%%%%%%%%%%%%%%%%%%%%%%%%%%

%Information from classinfo.tex file
%\def\classnum{2411}
%\def\classtitle{Calculus II}
%\def\classtitleshort{Calc 2}
%\def\classsec{001}
%\def\instructor{Rostermundt}
%\def\classterm{Fall 2024}
\def\topic{Divergence and Integral Tests}
\def\topicshort{Div. and Int. Tests}

	\title{\vspace{-1in}Math\classnum\;-\;\classtitle\\
	%Section\;\classsec\;-\;\classterm\\
	Guided Lecture Notes\\
	\topic}
	\author{University of Colorado Denver / College of Liberal Arts and Sciences}
	\date{Department of Mathematics}

	\markright{Math\classnum\;-\;\classtitleshort, University of Colorado Denver,\;\topicshort}


%%%%%%%%%%%%%%%%%%%%%%%%%%%%%%%%%%%%%%%%%%%%%%%%%%%%%%
\begin{document}\maketitle\thispagestyle{empty}
%%%%%%%%%%%%%%%%%%%%%%%%%%%%%%%%%%%%%%%%%%%%%%%%%%%%%%

\hrule

\section*{\topic\, Introduction:}

Our objective is to study some basic tests to determine whether an infinite series converges. So to get started, we recall geometric series
\[\ds\sum^{\infty}_{n=0}cr^n=c=cr+cr^2+cr^3+\cdots\]
We were able to find a formula for the $k^{th}$ partial sum $s_k$ and use this to determine when the series converged and to what value does it converge.
\[\ds\lim_{k\to\infty}s_k=\ds\lim_{k\to\infty}\,c\cdot\left(\ds\frac{1-r^{k-1}}{1-r}\right)=\ds\frac{c}{1-r}\quad\text{if }-1<r<r.\] 
For all other values of $r$ the geometric series diverges.
\vskip 5mm
Unfortunately, a formula for the $k^{th}$ partial sum $s_k$ is usually too difficult to find. So we need other methods to check for convergence or divergence.

\section*{Divergence Test:}

Can you think of a necessary condition on the terms of the series $a_n$ for convergence of $\sum^{\infty}_{n=1}a_n$? 
\vskip 1cm

Suppose that a sequence $\sum^{\infty}_{n=1}a_n$ converges. Then $\lim_{k\to\infty}s_k=L$ for some finite number $L$. Then we have
\[\ds\lim_{k\to\infty}a_k=\ds\lim_{k\to\infty}\Big(s_k-s_{k-1}\Big)=L-L=0.\]
So if a series converges we conclude that the terms in the series much converge to zero. This is logically equivalent to the statement of the following theorem.
\vskip 5mm

\begin{figure}[h!]
\begin{center}
\includegraphics[scale=0.7]{div_test_thm.jpg}
%\caption{}
\end{center}
\end{figure} 

\note The converse of the theorem is not true. That is, if $\lim_{n\to\infty}a_n=0$, we can conclude nothing about the convergence or divergence of the series. For example, $\lim_{n\to\infty}1/n=0$ while the series $\sum 1/n$ diverges. We will soon see that even though $\lim_{n\to\infty}1/n^2=0$ the series $\sum 1/n^2$ converges.
\vskip 5mm

\section*{Divergence Test Examples:}

\begin{example} Apply the divergence test to $\ds\sum^{\infty}_{n=0}\ds\frac{n}{3n-1}$.
\vskip 5mm
\noindent{\bf\emph{\underline{Workspace}:}}

\ifnum\longform=1
\vskip 1in
	\else
\vskip 1.5in

\fi

\ifnum\longform=1
	\begin{boxsolution}
\vspace*{5mm}
We see that 
\[\ds\lim_{n\to\infty}a_n=\ds\lim_{n\to\infty}\ds\frac{n}{3n-1}=\ds\frac{1}{3}\not=0.\]
\vskip 5mm
So the series diverges by the Divergence Test.
\vspace*{5mm}
	\end{boxsolution}
\vskip 5mm

\fi

\end{example}

\vskip 5mm


%%%%%%%%%%%%%%%%%%%%%%%%%%%%%%%%%%%%%%%%%%%%%%%%%%%%%%%%%
%%%%%%%%%%%%%%%%%%%%%%%%%%%%%%%%%%%%%%%%%%%%%%%%%%%%%%%%%
%\vskip 5mm
Here is another example.
\vskip 5mm

\begin{example} Apply the divergence test to $\ds\sum^{\infty}_{n=0}\ds\frac{1}{n^4}$.
\vskip 5mm
\noindent{\bf\emph{\underline{Workspace}:}}

\ifnum\longform=1
\vskip 1in
	\else
\vskip 1.5in

\fi

\ifnum\longform=1
	\begin{boxsolution}
\vspace*{5mm}
We see that 
\[\ds\lim_{n\to\infty}a_n=\ds\lim_{n\to\infty}\ds\frac{1}{n^4}=0.\]
\vskip 5mm
So the Divergence Test is inconclusive.
\vspace*{5mm}
	\end{boxsolution}
\vskip 5mm

\fi

\end{example}


%%%%%%%%%%%%%%%%%%%%%%%%%%%%%%%%%%%%%%%%%%%%%%%%%%%%%%%%%
%%%%%%%%%%%%%%%%%%%%%%%%%%%%%%%%%%%%%%%%%%%%%%%%%%%%%%%%%
\vskip 5mm
Here is another example.
\vskip 5mm

\begin{example} Apply the divergence test to $\ds\sum^{\infty}_{n=0}e^{1/n^2}$.
\vskip 5mm
\noindent{\bf\emph{\underline{Workspace}:}}

\vskip 2in

\ifnum\longform=1
	\begin{boxsolution}
\vspace*{5mm}
We see that 
\[\ds\lim_{n\to\infty}a_n=\ds\lim_{n\to\infty}e^{1/n^2}=e^0=1.\]
\vskip 5mm
So the series diverges by the Divergence Test.
\vspace*{5mm}
	\end{boxsolution}
\vskip 5mm

\fi

\end{example}


%%%%%%%%%%%%%%%%%%%%%%%%%%%%%%%%%%%%%%%%%%%%%%%%%%%%%%%%%
%%%%%%%%%%%%%%%%%%%%%%%%%%%%%%%%%%%%%%%%%%%%%%%%%%%%%%%%%

\section*{Integral Test:}

The Integral Test is an important and powerful test and is based on the logic of direct comparison. 
\vskip 5mm
Let's return to a familiar example $\ds\sum^{\infty}_{n=1}\ds\frac{1}{n}$ and use the improper integral $\ds\int^{\infty}_{x=1}\ds\frac{1}{x}\,dx$ for comparison.
\vskip 5mm

\begin{minipage}[]{6.5in}
\begin{center}
\includegraphics[scale=0.65]{int_test_gr1.jpg}
\captionof{figure}{Comparison Method for Integral Test}
\label{fig:}
\end{center}
\end{minipage}

\ifnum\longform=1
\vfill\eject
	\else
\vskip 5mm

\fi

We can see from the graphic that

\begin{minipage}[]{6.5in}
\begin{center}
\includegraphics[scale=0.75]{int_test_gr2.jpg}
%\captionof{figure}{}
\label{fig:}
\end{center}
\end{minipage}

Then since $\lim_{k\to\infty}\ln(1+k)=\infty$ we must have $\lim_{k\to\infty}S_k=\infty$ and the original infinite series diverges by the Integral Test.

\vskip 5mm

Let's consider another example with $\ds\sum^{\infty}_{n=1}\ds\frac{1}{n^2}$.

\vskip 5mm

\begin{minipage}[]{6.5in}
\begin{center}
\includegraphics[scale=0.5]{int_test_gr3.jpg}
\captionof{figure}{Comparison Method for Integral Test}
\label{fig:}
\end{center}
\end{minipage}

\ifnum\longform=1
\vskip 1cm
	\else
\vfill\eject

\fi

We can see from the graphic that

\vskip 5mm

\begin{minipage}[]{6.5in}
\begin{center}
\includegraphics[scale=0.75]{int_test_gr4.jpg}
%\captionof{figure}{}
\label{fig:}
\end{center}
\end{minipage}

We will use an important fact about infinite sequences. If a sequence $\{S_k\}$ is increasing and bounded then it must converge. In this example is is clear that
\[S_k=\ds\sum^{k}_{n=1}\ds\frac{1}{n^2}\]
is an increasing sequence. That is, $S_{k}\ge S_{k-1}$. We also see from above that the sequence of partial sums ${S_k}$ is bounded above by $2$. Therefore, the sequence of partial sums $\{S_k\}$ converges and we say the infinite series converges by the Integral Test.
\vskip 5mm
This process can be summed up by the following theorem.
\vskip 5mm
\begin{minipage}[]{6.5in}
\begin{center}
\includegraphics[scale=0.65]{int_test_thm.jpg}
%\captionof{figure}{}
\label{fig:}
\end{center}
\end{minipage} 

\vskip 1cm

Let's consider a few examples.

\section*{Integral Test Examples:}

%%%%%%%%%%%%%%%%%%%%%%%%%%%%%%%%%%%%%%%%%%%%%%%%%%%%%%
%%%%%%%%%%%%%%%%%%%%%%%%%%%%%%%%%%%%%%%%%%%%%%%%%%%%%%

\begin{example} Apply the Integral Test to $\ds\sum^{\infty}_{n=1}\ds\frac{1}{\sqrt{2n-1}}$.
\vskip 5mm
\noindent{\bf\emph{\underline{Workspace}:}}

\vfill\eject

\ifnum\longform=1
	\begin{boxsolution}
\vspace*{5mm}
We will compare
\vskip 5mm
\begin{minipage}[]{6.5in}
\begin{flushleft}
\includegraphics[trim= 0cm 0cm 4cm 0cm, clip=true, scale=0.75]{int_test_gr5.jpg}
%\captionof{figure}{}
\label{fig:}
\end{flushleft}
\end{minipage}

\vspace*{5mm}
	\end{boxsolution}
\vskip 5mm

\fi

\end{example}

We now look to an important type of series called a $p$-series.
\vskip 5mm
\begin{minipage}[]{6.5in}
\begin{center}
\includegraphics[scale=0.7]{pseries_def.jpg}
%\captionof{figure}{}
\label{fig:}
\end{center}
\end{minipage} 
\vskip 5mm

Fortunately we can come to a general solution about $p$-series.
\vskip 5mm
\begin{center}
A $p$-series $\sum^{\infty}_{n=1}\frac{1}{n^p}$ converges when $p>1$ and diverges when $p\le 1$.
\end{center}

Consider a few examples.

	\begin{enumerate}
			\item $\ds\sum^{\infty}_{n=1}\ds\frac{1}{n^{5/4}}$
\vskip 2mm
Since $p=5/4>1$ the infinite series converges as a $p$-series.
			\item $\ds\sum^{\infty}_{n=1}\ds\frac{1}{\sqrt{n}}$
\vskip 2mm
Since $p=1/2\le 1$ the infinite series diverges as a $p$-series.
			\item $\ds\sum^{\infty}_{n=1}\ds\frac{1}{n^3}$
\vskip 2mm
Since $p=3>1$ the infinite series converges as a $p$-series.
	\end{enumerate}

\ifnum\longform=1
\vskip 2mm
\note Observe that we do not necessarily know the the value of a convergent $p$-series. 
\vskip 2mm
Let's argue our result about $p$-series.

\fi

\vskip 5mm

\ifnum\longform=1
	\begin{discussion}
\vspace*{5mm}
\begin{minipage}[]{6.5in}
\begin{center}
\includegraphics[scale=0.7]{pseries_proof.jpg}
%\captionof{figure}{}
\label{fig:}
\end{center}
\end{minipage}

\vspace*{5mm}

	\end{discussion}
	
\fi

\section*{Estimation of Series Value:}

\vskip 5mm

\begin{minipage}[]{6.5in}
\begin{center}
\includegraphics[trim= 0cm 5.5cm 0cm 0cm, clip=true, scale=0.7]{series_value_estimate_gr1.jpg}
%\captionof{figure}{}
\label{fig:}
\end{center}
\end{minipage}

\vfill\eject

\begin{minipage}[]{6.5in}
\begin{center}
\includegraphics[trim= 0cm 0cm 2.3cm 3.3cm, clip=true, scale=0.7]{series_value_estimate_gr1.jpg}\\
\includegraphics[trim= 0cm 0cm 0cm 0cm, clip=true, scale=0.7]{series_value_estimate_gr2.jpg}
%\captionof{figure}{}
\label{fig:}
\end{center}
\end{minipage}
\vskip 1cm
\begin{minipage}[]{6.5in}
\begin{flushleft}
\qquad\qquad\includegraphics[trim= 0cm 0cm 2.3cm 3.3cm, clip=true, scale=0.8]{error_graphic.jpg}
\captionof{figure}{Visualization of Partial Sum Estimate}
\label{fig:}
\end{flushleft}
\end{minipage}

\vskip 1cm

We can see that 
\beq
R_N&=&a_{N+1}+a_{N+2}+a_{N+3}+\cdots\le\ds\int^{\infty}_{x=N}f(x)\,dx\\
R_N&=&a_{N+1}+a_{N+2}+a_{N+3}+\cdots\ge\ds\int^{\infty}_{x=N+1}f(x)\,dx\\
\eeq
So the integral is either an overestimate of $R_N$ or an underestimate of $R_n$.

\vskip 1cm

Let's work an example.

\vfill\eject


%%%%%%%%%%%%%%%%%%%%%%%%%%%%%%%%%%%%%%%%%%%%%%%%%%%%%%%%%
%%%%%%%%%%%%%%%%%%%%%%%%%%%%%%%%%%%%%%%%%%%%%%%%%%%%%%%%%

\begin{example} Consider the infinite series $\sum^{\infty}_{n=1}\frac{1}{n^3}$. It turns out that, even though we know the series converges as a $p$-series, to this day mathematicians have been unable to determine an explicit form the value of the series. So we must estimate the value. Let's use the partial sum $S_{10}$.
\vskip 5mm
\noindent{\bf\emph{\underline{Workspace}:}}

\vskip 4in

\ifnum\longform=1
	\begin{boxsolution}
\vspace*{5mm}
We can compute the following.
\vskip 5mm
\begin{minipage}[]{6.5in}
\begin{center}
\includegraphics[trim= 0cm 0cm 0.1cm 0cm, clip=true, scale=0.75]{series_estimate_example.jpg}
%\captionof{figure}{}
\label{fig:}
\end{center}
\end{minipage}
\[\vdots\]
\vspace*{5mm}
	\end{boxsolution}

\vfill\eject

	\begin{boxsolutioncont}
\vspace*{5mm}
\[\vdots\]
Then we have the error for our calculation as
\[R_{10}<\ds\frac{1}{2(10)^2}=0.005\]
So we conclude that
\[1.19253<\ds\sum^{\infty}_{n=1}\ds\frac{1}{n^3}<1.20253\]
If we want a more accurate estimation we simply need to add more terms. For example, if we want our error $R_N<0.00001$ we would could solve
\[R_N<\ds\frac{1}{2N^2}<\ds\frac{1}{100000}\quad\Longrightarrow\quad N>\sqrt{50000}\approx 223.6\]
Then the partial sum $S_{224}=\sum^{224}_{n=0}\frac{1}{n^3}$ would be within $0.00001$ of the true value of the series. After computing $S_{224}$ we conclude
\[1.20204<\ds\sum^{\infty}_{n=1}\ds\frac{1}{n^3}<1.20206\]
In fact, since the partial sums are increasing we can do better and conclude
\[S_{224}<\ds\sum^{\infty}_{n=1}\ds\frac{1}{n^3}<S_{224}+0.00001\quad\Longrightarrow\quad 1.202046983<\ds\sum^{\infty}_{n=1}\ds\frac{1}{n^3}<1.202056983\] 
Since the series converges we can continue to add more and more terms and become closer and closer to the true value of the series.
\vspace*{5mm}
	\end{boxsolutioncont}
	
\fi

\end{example}

\ifnum\longform=1
\vskip 5mm
This concludes our notes on the Divergence Test and the Integral Test.
\vskip 1cm

\fi


%%%%%%%%%%%%%%%%%%%%%%%%%%%%%%%%%%%%%%%%%%%%%%%%%%%%%%%%%
%%%%%%%%%%%%%%%%%%%%%%%%%%%%%%%%%%%%%%%%%%%%%%%%%%%%%%%%%


%%%%%%%%%%%%%%%%%%%%%%%%%%%%%%%%%%%%%%%%%%%%%%%%%%%%%%%%%
%%%%%%%%%%%%%%%%%%%%%%%%%%%%%%%%%%%%%%%%%%%%%%%%%%%%%%%%%
%%%%%%%%%%%%%%%%%%%%%%%%%%%%%%%%%%%%%%%%%%%%%%%%%%%%%%%%%
%%%%%%%%%%%%%%%%%%%%%%%%%%%%%%%%%%%%%%%%%%%%%%%%%%%%%%%%%

	
%%%%%%%%%%%%%%%%%%%%%%%%%%%%%%%%%%%%%%%%%%%%%%%%%%%%%%
%%%%%%%%%%%%%%%%%%%%%%%%%%%%%%%%%%%%%%%%%%%%%%%%%%%%%%

\ifnum\longform=1
\vskip 1cm
\hrule
\vskip 5mm
\begin{center}{\bf Please let me know if you have any questions, comments, or corrections!}
\end{center}	

\fi

%%%%%%%%%%%%%%%%%%%%%%%%%%%%%%%%%%%%%%%%%%%%%%%%%%%%%%
\end{document}
%%%%%%%%%%%%%%%%%%%%%%%%%%%%%%%%%%%%%%%%%%%%%%%%%%%%%%
\documentclass[11pt]{article}
\usepackage[suffix=Solutions]{teaching-header}

\def\classnum{2411}
\def\classtitle{Calculus II}
\def\classtitleshort{Calc 2}
\def\classsec{H01}
\def\instructor{Dr. Rostermundt}
\def\classterm{Spring 2025}


%%%%%%%%%%%%%%%%%%%%%%%%%%%%%%%%%%%%%%%%%%%%%%%%%%%%%%%%%%%%%%%%%%%%%%%%%%%%
%%%%%%%%%%%%%%%%%%%%%%%%%%%%%%%%%%%%%%%%%%%%%%%%%%%%%%%%%%%%%%%%%%%%%%%%%%%%

%This is defined in the teaching-header style file
%\ifnum\printsol=0 (when no solutions printed)
%Do something
%	\else  (when solutions are printed)
%Do something else
%\fi


% Package and setting included in teachin-header style file
%\RequirePackage{amsmath,amsfonts,amssymb,amsthm,graphicx, pgfplots, tcolorbox, xcolor,latexsym,color,verbatim,float,xcolor,setspace}
%%tikzsymbols
%
%\RequirePackage{enumerate}
%\RequirePackage{multicol}
%\RequirePackage{tikz}
%\RequirePackage{cancel}
%\usetikzlibrary{shapes.geometric}
%\usetikzlibrary{calc, positioning, arrows}
%\RequirePackage[margin=1in,letterpaper]{geometry}
%\RequirePackage[colorlinks=true,allcolors=blue]{hyperref}
%\usepackage[final]{pdfpages}
%%\usepackage{capt-of}
%
%
%\setlength{\textheight}{9in}
%\setlength{\textwidth}{6.5in}
%\addtolength{\topmargin}{0cm}
%%\addtolength{\oddsidemargin}{0cm}
%\parindent=0in
%\parskip=.35em
%\singlespacing
%%\pagestyle{empty}  % remove page numbers

%Add captions without being in figure environment
%\captioof{figure}{\text}\label[fig:]
\usepackage{capt-of}
\usepackage{mathtools}

\vfuzz2pt % Don't report over-full v-boxes if over-edge is small
\hfuzz2pt % Don't report over-full h-boxes if over-edge is small


%%%%%%%%%%%%%%%%%%%%%%%%%%%%%%%%%%%%%%%%%%%%%%%%%%%%%%
%%%%%%%%%%%%%%%%%%%%%%%%%%%%%%%%%%%%%%%%%%%%%%%%%%%%%%

\pagestyle{myheadings}

%%%%%%%%%%%%%%%%%%%%%%%%%%%%%%%%%%%%%%%%%%%%%%%%%%%%%%
%%%%%%%%%%%%%%%%%%%%%%%%%%%%%%%%%%%%%%%%%%%%%%%%%%%%%%


%%%%%%%%%%%%%%%%%%%%%%%%%%%%%%%%%%%%%%%%%%%%%%%%%%%%%%
%%%%%%%%%%%%%%%%%%%%%%%%%   Document Body   %%%%%%%%%%
%%%%%%%%%%%%%%%%%%%%%%%%%%%%%%%%%%%%%%%%%%%%%%%%%%%%%%

%Information from classinfo.tex file
%\def\classnum{2411}
%\def\classtitle{Calculus II}
%\def\classtitleshort{Calc 2}
%\def\classsec{001}
%\def\instructor{Rostermundt}
%\def\classterm{Fall 2024}
\def\topic{A Nice Limit Result}
\def\topicshort{A Nice Limit}

	\title{\vspace{-1in}Math\classnum\;-\;\classtitle\\
	%Section\;\classsec\;-\;\classterm\\
	Guided Lecture Notes\\
	\topic}
	\author{University of Colorado Denver / College of Liberal Arts and Sciences}
	\date{Department of Mathematics}

	\markright{Math\classnum\;-\;\classtitleshort, University of Colorado Denver,\;\topicshort}


%%%%%%%%%%%%%%%%%%%%%%%%%%%%%%%%%%%%%%%%%%%%%%%%%%%%%%
\begin{document}\maketitle\thispagestyle{empty}
%%%%%%%%%%%%%%%%%%%%%%%%%%%%%%%%%%%%%%%%%%%%%%%%%%%%%%

\hrule
\vskip 1cm

In our guided lecture notes we considered the infinite series
\[\ds\sum^{\infty}_{n=0}(-1)^{n+1}\ds\frac{\big(n!\big)^2}{(2n)!}.\]
We showed the series converges by the Ratio Test but we also wanted to try the Alternating Series Test. The question is how to evaluate the limit
\[\ds\lim_{n\to\infty}\ds\frac{\big(n!\big)^2}{(2n!)}\]
without using the heavy machinery of Stirling's approximation $n!\approx\sqrt{2\pi n}\left(\frac{e}{n}\right)^n$?
\vskip 5mm

Instead we will use Bernoulli's Inequality.

\begin{theorem}[Bernoulli] Let $x\ge -1$ be a real number and $r$ be any non-negative real number. Then we satisfy the following inequalities.
	\begin{enumerate}
		\item If $r\ge 1$, then $(1+x)^r\ge 1+rx$.
		\item If $0\le r\le 1$, then $(1+x)^r\le 1+rx$.
	\end{enumerate}

\end{theorem} 

To evaluate the limit we will consider the sequence $a_n=(2n)!/(n!)^2$ and first observe that
\[\ds\frac{(2n)!}{\big(n!\big)^2}=\ds\frac{(2n)(2n-1)(2n-2)!}{n^2\cdot\big[(n-1)!\big]^2}=4\left(1-\ds\frac{1}{2n}\right)\ds\frac{\big[2(n-1)\big]!}{\big[(n-1)!\big]\big)^2}\]
From this and an induction argument we see that
\beq
\ds\frac{(2n)!}{\big(n!\big)^2}&=&4^n\ds\prod^{n-1}_{k=0}\left(1-\ds\frac{1}{2k+2}\right)\\
&=&\ds\frac{4^n}{2}\ds\prod^{n-1}_{k=1}\left(1-\ds\frac{1}{2k+2}\,\right)\\
&=&\ds\frac{4^n}{2}\ds\prod^{n-1}_{k=1}\left(1-\ds\frac{1}{2}\left(\ds\frac{1}{k+1}\right)\,\right)\\
&\ge&\ds\frac{4^n}{2}\ds\prod^{n-1}_{k=1}\left(1-\ds\frac{1}{k+1}\,\right)^{1/2}\quad\text{By Bernoulli's Inequality}\\
&=&\ds\frac{4^n}{2\sqrt{n}}
\eeq
So we have the following inequalities.
\[\ds\frac{(2n)!}{\big(n!\big)^2}\ge \ds\frac{4^n}{2\sqrt{n}}\quad\Longrightarrow\quad 0\le\ds\frac{\big(n!\big)^2}{(2n)!}\le \ds\frac{2\sqrt{n}}{4^n}. \]
Now since $\ds\lim_{n\to\infty}\ds\frac{2\sqrt{n}}{4^n}=0$, we have by the Squeeze Theorem that $\ds\lim_{n\to\infty}\ds\frac{\big(n!\big)^2}{(2n)!}=0$.
%%%%%%%%%%%%%%%%%%%%%%%%%%%%%%%%%%%%%%%%%%%%%%%%%%%%%%%%%
%%%%%%%%%%%%%%%%%%%%%%%%%%%%%%%%%%%%%%%%%%%%%%%%%%%%%%%%%

 
%%%%%%%%%%%%%%%%%%%%%%%%%%%%%%%%%%%%%%%%%%%%%%%%%%%%%%%%%
%%%%%%%%%%%%%%%%%%%%%%%%%%%%%%%%%%%%%%%%%%%%%%%%%%%%%%%%%
%%%%%%%%%%%%%%%%%%%%%%%%%%%%%%%%%%%%%%%%%%%%%%%%%%%%%%%%%
%%%%%%%%%%%%%%%%%%%%%%%%%%%%%%%%%%%%%%%%%%%%%%%%%%%%%%%%%

	
%%%%%%%%%%%%%%%%%%%%%%%%%%%%%%%%%%%%%%%%%%%%%%%%%%%%%%
%%%%%%%%%%%%%%%%%%%%%%%%%%%%%%%%%%%%%%%%%%%%%%%%%%%%%%


\vskip 5mm
\hrule
\vskip 5mm
\begin{center}{\bf Please let me know if you have any questions, comments, or corrections!}
\end{center}	


%%%%%%%%%%%%%%%%%%%%%%%%%%%%%%%%%%%%%%%%%%%%%%%%%%%%%%
\end{document}
%%%%%%%%%%%%%%%%%%%%%%%%%%%%%%%%%%%%%%%%%%%%%%%%%%%%%%
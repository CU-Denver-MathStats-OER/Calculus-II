%%%%%%%%%%%%%%%%%%%%%%%%%%%%%%%%%%%%%%%%%%%%%%%%%%%%%%
%\documentclass[UTF8, a4, 11pt]{ctexart}
%%%%%%%%%%%%%%%%%%%%%%%%%%%%%%%%%%%%%%%%%%%%%%%%%%%%%%

%%%%%%%%%%%%%%%%%%%%%%%%%%%%%%%%%%%%%%%%%%%%%%%%%%%%%%
\documentclass[11pt]{article}
%%%%%%%%%%%%%%%%%%%%%%%%%%%%%%%%%%%%%%%%%%%%%%%%%%%%%%

\usepackage{amsmath}
\usepackage{amsthm}
\usepackage{amssymb}
\usepackage{mathtools}
\usepackage{latexsym}
\usepackage{graphicx}
\usepackage{color}
\usepackage{verbatim}
\usepackage{float}
\usepackage{multicol}
\usepackage[final]{pdfpages}


\usepackage[top=1in, bottom=1in, left=1in, right=2in]{geometry}
\setlength{\textheight}{9in}
\setlength{\textwidth}{6.5in}
%\addtolength{\topmargin}{-2cm}
%\addtolength{\oddsidemargin}{-1cm}
%\addtolength{\evensidemargin}{2cm}
\parindent=0in


%\def\classnum{2411}
\def\classtitle{Calculus II}
\def\classtitleshort{Calc 2}
\def\classsec{H01}
\def\instructor{Dr. Rostermundt}
\def\classterm{Spring 2025}

\input{latexmacros}

\vfuzz2pt % Don't report over-full v-boxes if over-edge is small
\hfuzz2pt % Don't report over-full h-boxes if over-edge is small

%%%%%%%%%%%%%%%%%%%%%%%%%%%%%%%%%%%%%%%%%


%%%%%%%%%%%%%%%%%%%%%%%%%%%%%%%%%%%%%%%%%%%%%%%%%%%%%%

%%%%%%%%%%%%%%%%%%%%%%%%%%%%%%%%%%%%%%%%%%%%%%%%%%%%%%

\pagestyle{myheadings}

%%%%%%%%%%%%%%%%%%%%%%%%%%%%%%%%%%%%%%%%%%%%%%%%%%%%%%

%%%%%%%%%%%%%%%%%%%%%%%%%%%%%%%%%%%%%%%%%%%%%%%%%%%%%%
%%%%%%%%%%%%%%%%%%%%%%%%%   Document Body   %%%%%%%%%%
%%%%%%%%%%%%%%%%%%%%%%%%%%%%%%%%%%%%%%%%%%%%%%%%%%%%%%

\def\printsol{0}
\def\classnum{2411}
\def\classtitle{Calculus II}
\def\classtitleshort{Calc 2}
\def\classsec{001}
\def\instructor{Rostermundt}
\def\classterm{Fall 2024}
\def\topic{Sequences and Series}
\def\topicshort{Separable ODEs}
%\def\hmwknum{\#2}

%\begin{figure}[h]
%\begin{center}
%\includegraphics[scale=0.6]{area_between_curves_gr1.jpg}
%\caption{Region between graphs $y=f(x)$ and $y=g(x)$ on the interval $[a,b]$}
%\end{center}
%\end{figure}

\ifnum\printsol=0
	\title{\vspace{-1in}Math\classnum\;-\;\classtitle\\
	Section\;\classsec\;\classterm\\
	Introduction to \topic}
	\author{University of Colorado Denver / College of Liberal Arts 	and Sciences}
	\date{Department of Mathematics - Dr. \instructor}

	\markright{Math\classnum\;-\;\classtitleshort,\;\topicshort, UCD, \classterm, Dr. \instructor}

		\else


%	\title{\vspace{-1in}Math\classnum\;-\;\classtitle\\
%	%Section\;\classsec\;-\;
%	\classterm\\
%	Calculus I Review Solutions}
%	\author{University of Colorado Denver / College of Liberal Arts 	and Sciences}
%	\date{Department of Mathematics - Dr. \instructor}
%
%	\markright{Math\classnum\;-\;\classtitleshort,\;Calc I Review Solutions, UCD, 		\classterm, Dr. \instructor}

\fi 


%%%%%%%%%%%%%%%%%%%%%%%%%%%%%%%%%%%%%%%%%%%%%%%%%%%%%%
\begin{document}\maketitle\thispagestyle{empty}
%%%%%%%%%%%%%%%%%%%%%%%%%%%%%%%%%%%%%%%%%%%%%%%%%%%%%%

\hrule

\section*{\topic\; Introduction:}

Our objective is to study the basics of {\bf\emph{infinite sequences}} and {\bf\emph{infinite series}}. So to get started, what is an infinite sequence?


\begin{figure}[h!]
\begin{center}
\includegraphics[scale=0.6]{sequence_def.jpg}
%\caption{}
\end{center}
\end{figure} 

\section*{Sequence Examples:}

\begin{ex}

Consider the sequence $\{a_n\}$ where $a_n=2^n$ for natural numbers $n$. 
\[\big\{2^n\big\}_{n\in\N}=\big\{2,4,8,16,32,\dots\big\}\]
Notice that we will use curly set brackets for sequence notation. And as above we can describe a sequence either by some explicit rule, or by listing enough of the sequence elements to understand the pattern (if there is indeed a pattern). Here is a graph of the sequence.
\vskip 5mm
\begin{figure}[h!]
\begin{center}
\includegraphics[scale=0.4]{sequence_graph_gr01.jpg}
\caption{Graph of the sequence where $a_n=2^n$.}
\end{center}
\end{figure}

\end{ex}

\vfill\eject


%%%%%%%%%%%%%%%%%%%%%%%%%%%%%%%%%%%%%%%%%%%%%%%%%%%%%%%%%
%%%%%%%%%%%%%%%%%%%%%%%%%%%%%%%%%%%%%%%%%%%%%%%%%%%%%%%%%

Here is another example.
\vskip 5mm

\begin{ex} The Fibonnaci sequence is one of the most famous sequences in all of mathematics. This is defined recursively as follows.
\[a_{n}=a_{n-1}+a_{n-2},\quad\text{where }a_0=a_1=1\]
Notice that the sequence is not defined as an explicit formula involving $n$. That is, we are not given a rule $a_n=f(n)$. Rather, the $n^{th}$ term in the sequence is determined by the previous two terms. Here is the sequence.
\[\big\{1,1,2,3,5,8,13,21,34,\dots\big\}\]

%\begin{figure}[h]
%\begin{center}
%\includegraphics[scale=0.7]{separable_ode_gr05.jpg}
%\caption{Examples of Differential Equations in the Form $y'=F(x,y)$.}
%\end{center}
%\end{figure}

\end{ex}

%%%%%%%%%%%%%%%%%%%%%%%%%%%%%%%%%%%%%%%%%%%%%%%%%%%%%%%%%
%%%%%%%%%%%%%%%%%%%%%%%%%%%%%%%%%%%%%%%%%%%%%%%%%%%%%%%%%
\vskip 5mm
Here is another simpler example.
\vskip 5mm

\begin{ex} Define a sequence as follows so that $a_n=\ds\frac{n}{n^2+1}$.
\[\big\{a_n\big\}_{n\in\N}=\left\{\ds\frac{1}{2},\ds\frac{2}{5},\ds\frac{3}{10},\ds\frac{4}{17},\dots\right\}\]

%\begin{figure}[h]
%\begin{center}
%\includegraphics[scale=0.7]{separable_ode_gr05.jpg}
%\caption{Examples of Differential Equations in the Form $y'=F(x,y)$.}
%\end{center}
%\end{figure}

\end{ex}


%%%%%%%%%%%%%%%%%%%%%%%%%%%%%%%%%%%%%%%%%%%%%%%%%%%%%%%%%
%%%%%%%%%%%%%%%%%%%%%%%%%%%%%%%%%%%%%%%%%%%%%%%%%%%%%%%%%
\vskip 5mm
Here is another example.
\vskip 5mm

\begin{ex} Here is is one of the most important sequences in all of mathematics. Let $k$ be any real number and define
\[a_n=\left(1+\ds\frac{k}{n}\right)^n\]
So we have
\[\big\{a_n\big\}^{\infty}_{n=1}=\left\{1+k, \left(1+\ds\frac{k}{2}\right)^2, \left(1+\ds\frac{k}{3}\right)^3,\left(1+\ds\frac{k}{4}\right)^4,\dots\right\}\]

\end{ex}

%%%%%%%%%%%%%%%%%%%%%%%%%%%%%%%%%%%%%%%%%%%%%%%%%%%%%%%%%
%%%%%%%%%%%%%%%%%%%%%%%%%%%%%%%%%%%%%%%%%%%%%%%%%%%%%%%%%
\vskip 5mm

\section*{Limiting Values of Sequences:}

The main question we will ask about sequences is whether of not the sequence converges to some value or diverges.
\vskip 5mm
\begin{figure}[h!]
\begin{center}
\includegraphics[scale=0.7]{sequence_converge_def1.jpg}
%\caption{}
\end{center}
\end{figure}
\vskip 5mm
There is a nice graphic that reflects this idea.
\vskip 5mm
\begin{figure}[h!]
\begin{center}
\includegraphics[scale=0.5]{sequence_converge_graphic.jpg}
\caption{Graphic Representation of a Sequence $\{a_n\}$ Converging to $L$.}
\end{center}
\end{figure}
\vskip 5mm
The above definition is a somewhat informal definition that we will use in this class. However, we should realize that this definition can be formalized to be perfectly precise as follows.
\vskip 5mm
\begin{figure}[h!]
\begin{center}
\includegraphics[scale=0.7]{sequence_converge_def2.jpg}
%\caption{}
\end{center}
\end{figure}
\vskip 5mm
We will not be working with this formal definition in this class. So how will we determine if a sequence converges or diverges? We will be exclusively evaluating explicit sequences and the following theorem will be our main tool for determining convergence or divergence.
\vskip 5mm
\begin{figure}[h!]
\begin{center}
\includegraphics[scale=0.7]{sequence_converge_thm1.jpg}
%\caption{}
\end{center}
\end{figure}

This theorem tells us that we can use all of our previous knowledge of function limits to evaluate sequence limits.

\vfill\eject

%%%%%%%%%%%%%%%%%%%%%%%%%%%%%%%%%%%%%%%%%%%%%%%%%%%%%%%%%
%%%%%%%%%%%%%%%%%%%%%%%%%%%%%%%%%%%%%%%%%%%%%%%%%%%%%%%%%

\begin{ex} Determine if the sequence $\{a_n\}$, where $a_n=n/(n^2+1)$, converges or diverges.
\vskip 5mm
\noindent{\bf\emph{\underline{Workspace}:}}

\vfill\eject

\ssol We see that $a_n=f(n)$ for the function $f(x)=x/(x^2+1)$. Then our knowledge of growth rates tells us that
\[\ds\lim_{x\to\infty}f(x)=\ds\lim_{x\to\infty}\ds\frac{x}{x^2+1}=0.\]
Therefore, the sequence $\{a_n\}$ converges to zero.
\vskip 5mm
Alternatively we could apply L'Hopital's rule.
\beq
\ds\lim_{x\to\infty}\ds\frac{x}{x^2+1}&=&\ds\frac{\infty}{\infty}\quad\text{form}\\
\\
&\downarrow&\text{by L'Hopital's Rule}\\
\\
\ds\lim_{x\to\infty}\ds\frac{x}{x^2+1}&=&\ds\lim_{x\to\infty}\ds\frac{1}{2x}=\ds\frac{1}{\infty}\quad\text{form}
\eeq
So $\ds\lim_{n\to\infty}a_n=\ds\lim_{n\to\infty}\ds\frac{n}{n^2+1}=0$.

\end{ex}

\vskip 5mm
Let's consider another example.
\vskip 5mm

%%%%%%%%%%%%%%%%%%%%%%%%%%%%%%%%%%%%%%%%%%%%%%%%%%%%%%%%%
%%%%%%%%%%%%%%%%%%%%%%%%%%%%%%%%%%%%%%%%%%%%%%%%%%%%%%%%%

\begin{ex} Determine if the sequence $\{a_n\}$, where $a_n=2n^3/(5n^3-n^2+1)$, converges or diverges.
\vskip 5mm
\noindent{\bf\emph{\underline{Workspace}:}}

\vfill\eject

\ssol We see that $a_n=f(n)$ for the function $f(x)=2x^3/(5x^3-x^2+1)$. Then our knowledge of growth rates tells us that
\[\ds\lim_{x\to\infty}f(x)=\ds\lim_{x\to\infty}\ds\frac{x}{x^2+1}=\ds\frac{2}{5}.\]
Therefore, the sequence $\{a_n\}$ converges to $2/5$.
\vskip 5mm
Alternatively we could apply algebraic techniques.
\beq
\ds\lim_{x\to\infty}\ds\frac{2x^3}{5x^3-x^2+1}&=&\ds\lim_{x\to\infty}\,\ds\frac{x^3}{x^3}\cdot\left(\ds\frac{2}{5-\ds\frac{1}{x}+\ds\frac{1}{x^3}}\right)\\
&=&\ds\lim_{x\to\infty}\ds\frac{2}{5-\ds\frac{1}{x}+\ds\frac{1}{x^3}}\\
&=&\ds\frac{2}{5-0+0}\\
\\
&=&\ds\frac{2}{5}
\eeq
So $\ds\lim_{n\to\infty}a_n=\ds\lim_{n\to\infty}\ds\frac{2n^3}{5n^3-n^2+1}=\ds\frac{2}{5}$.

\end{ex}

%%%%%%%%%%%%%%%%%%%%%%%%%%%%%%%%%%%%%%%%%%%%%%%%%%%%%%%%%
%%%%%%%%%%%%%%%%%%%%%%%%%%%%%%%%%%%%%%%%%%%%%%%%%%%%%%%%%


\vskip 5mm
Let's try another example.
\vskip 5mm

\begin{ex} Determine if the sequence $\{a_n\}^{\infty}_{n=1}$, where $a_n=7n/\ln(n+1)$, converges or diverges.
\vskip 5mm
\noindent{\bf\emph{\underline{Workspace}:}}

\vfill\eject

\ssol We see that $a_n=f(n)$ for the function $f(x)=7x/\ln(x+1)$. Then our knowledge of growth rates tells us that
\[\ds\lim_{x\to\infty}f(x)=\ds\lim_{x\to\infty}\ds\frac{7x}{\ln(x+1)}=\infty.\]
So the sequence $\{a_n\}$ diverges to $\infty$.
\vskip 5mm
Alternatively we could apply L'Hopital's rule.
\beq
\ds\lim_{x\to\infty}\ds\frac{7x}{\ln(x+1)}&=&\ds\frac{\infty}{\infty}\quad\text{form}\\
\\
&\downarrow&\text{by L'Hopital's Rule}\\
\\
\ds\lim_{x\to\infty}7(x+1)&=&\infty
\eeq
So $\ds\lim_{n\to\infty}a_n=\ds\lim_{n\to\infty}\ds\frac{7n}{\ln(n+1)}=\infty$ and the sequence diverges.

\end{ex}

\vskip 5mm
Let's try another extremely important example.
\vskip 5mm

%%%%%%%%%%%%%%%%%%%%%%%%%%%%%%%%%%%%%%%%%%%%%%%%%%%%%%%%%
%%%%%%%%%%%%%%%%%%%%%%%%%%%%%%%%%%%%%%%%%%%%%%%%%%%%%%%%%

\begin{ex} Determine if the sequence $\{a_n\}^{\infty}_{n=1}$, where $a_n=\left(1+\ds\frac{k}{n}\right)^n$, converges or diverges.
\vskip 5mm
\noindent{\bf\emph{\underline{Workspace}:}}

\vfill\eject

\ssol We see that $a_n=f(n)$ for the function $f(x)=\left(1+\ds\frac{k}{x}\right)^x$. Then our knowledge of limits from Calculus I tells us that
\[\ds\lim_{x\to\infty}f(x)=\ds\lim_{x\to\infty}\left(1+\ds\frac{k}{x}\right)^x=e^k.\]
So the sequence $\{a_n\}$ converges to the exponential $e^k$.
\vskip 5mm
Alternatively we could apply L'Hopital's rule to verify the above limit, but from this point on we will define
\[\ds\lim\left(1+\ds\frac{k}{n}\right)^n:=e^k.\]

\end{ex}
\vskip 5mm

And now one more example that will be important for us in this course.

%%%%%%%%%%%%%%%%%%%%%%%%%%%%%%%%%%%%%%%%%%%%%%%%%%%%%%%%%
%%%%%%%%%%%%%%%%%%%%%%%%%%%%%%%%%%%%%%%%%%%%%%%%%%%%%%%%%
\vskip 5mm

\begin{ex} Determine if the sequence $\{a_n\}^{\infty}_{n=1}$, where $a_n=cr^n$ for some real numbers $r$ and $c$, converges or diverges. This sequence is called a {\bf\emph{geometric sequence}}.
\vskip 5mm
\noindent{\bf\emph{\underline{Workspace}:}}

\vfill\eject

\ssol The sequence is listed as
\[\{a_n\}^{\infty}_{n=0}=\big\{c,cr,cr^2,cr^3,cr^4,\dots\big\}.\]
The defining property of this sequence is the common ration between consecutive terms.

\[\dots,cr^{k-1},\underbrace{cr^k,cr^{k+1}}_{a_{k+1}=ra_{k}},\dots\]
The limiting value of this sequence depends on the value of $r$.
	\begin{itemize}
		\item[$\bullet$] If $-1<r<1$, then $\ds\lim_{k\to\infty}a_k=\ds\lim_{k\to\infty}cr^k=0$.
		\item[$\bullet$] If $r=1$, then $\ds\lim_{k\to\infty}a_k=\ds\lim_{k\to\infty}c1^k=c$.
		\item[$\bullet$] If $r>1$, then $\ds\lim_{k\to\infty}a_k=\ds\lim_{k\to\infty}cr^k=\infty$.
		\item[$\bullet$] If $r\le -1$, then $\ds\lim_{k\to\infty}a_k=\ds\lim_{k\to\infty}cr^k$ does not exist.
	\end{itemize}




\end{ex}

%%%%%%%%%%%%%%%%%%%%%%%%%%%%%%%%%%%%%%%%%%%%%%%%%%%%%%%%%
%%%%%%%%%%%%%%%%%%%%%%%%%%%%%%%%%%%%%%%%%%%%%%%%%%%%%%%%%
\vskip 5mm

\section*{Infinite Series:}

Next we will be studying {\bf\emph{infinite series}}. So what is an infinite series? It is simply a non-terminating discrete sum. We can think of it as an infinite sum of the terms in an infinite sequence. 
\vskip 5mm

\begin{figure}[h!]
\begin{center}
\includegraphics[scale=0.7]{series_converge_def.jpg}
%\caption{}
\end{center}
\end{figure} 

\vskip 5mm

\section*{Infinite Series Examples:}



\vskip 5mm

%%%%%%%%%%%%%%%%%%%%%%%%%%%%%%%%%%%%%%%%%%%%%%%%%%%%%%%%%
%%%%%%%%%%%%%%%%%%%%%%%%%%%%%%%%%%%%%%%%%%%%%%%%%%%%%%%%%

\begin{ex} Determine if the following infinite series converges.
\[\ds\frac{3}{10}+\ds\frac{3}{100}+\ds\frac{3}{1000}+\ds\frac{3}{10000}+\cdots\] 
\vskip 5mm
\noindent{\bf\emph{\underline{Workspace}:}}

\vfill\eject

\ssol Recognizing this series as corresponding to the decimal expansion of $1/3$ we can see that the series will converge to $1/3$. But let's be more formal and use the given definition.
\[s_k=\ds\frac{3}{10}+\ds\frac{3}{10^2}+\ds\frac{3}{10^3}+\cdots+\ds\frac{3}{10^k}.\]
Multiplying both sides by $1/10$ we have
\[s_k/10=\ds\frac{3}{10^2}+\ds\frac{3}{10^3}+\ds\frac{3}{10^4}+\cdots+\ds\frac{3}{10^{k+1}}.\]
It follows that
\[s_k-\ds\frac{s_k}{10}=\ds\frac{9s_k}{10}=\ds\frac{3}{10}-\ds\frac{3}{10^{k+1}}\quad\Longrightarrow\quad s_k=\ds\frac{10}{9}\left(\ds\frac{3}{10}-\ds\frac{3}{10^{k+1}}\right)=\ds\frac{1}{3}-\ds\frac{1}{3\cdot 10^k}.\]
To keep things as concrete as possible let's see what this means. If we have added $k=5$ terms then the $5^{th}$ partial sum is
\[s_5=\ds\frac{1}{3}-\ds\frac{1}{3\cdot 10^5}=0.33333.\]
If we have added $k=10$ terms then the $10^{th}$ partial sum is
\[s_{10}=\ds\frac{1}{3}-\ds\frac{1}{3\cdot 10^{10}}=0.3333333333.\]
If we have added $k=100$ terms then the $100^{th}$ partial sum is
\[s_{100}=\ds\frac{1}{3}-\ds\frac{1}{3\cdot 10^{100}}=0.\underbrace{33333\dots 3333}_{\text{One Hundred 3's}}.\]
It should be clear that as we add more and more terms we are getting arbitrarily close to the value $1/3$.
\vskip 5mm
More formally,
\[\ds\lim_{k\to\infty}s_k=\ds\lim_{k\to\infty}\ds\frac{1}{3}-\ds\frac{1}{3\cdot 10^k}=\ds\frac{1}{3}-0=\ds\frac{1}{3}.\] 
So the infinite series converges to $s=1/3$.
\end{ex}

\vskip 5mm
This example is a specific case of what's known as a geometric series. Let's consider the general case for a geometric series (one of the most important sequences).
\vskip 5mm


%%%%%%%%%%%%%%%%%%%%%%%%%%%%%%%%%%%%%%%%%%%%%%%%%%%%%%%%%
%%%%%%%%%%%%%%%%%%%%%%%%%%%%%%%%%%%%%%%%%%%%%%%%%%%%%%%%%


\begin{ex} Determine if the following infinite series converges.
\[\ds\sum^{\infty}_{n=0}cr^n=c+cr+cr^2+cr^3+cr^4+\cdots\] 
\vskip 5mm
\noindent{\bf\emph{\underline{Workspace}:}}

\vfill\eject

\noindent{\bf\emph{\underline{Workspace Cont.}:}}

\vfill\eject

\ssol Let's be very careful and use the formal definition.
\[s_k=c+cr+cr^2+cr^3+cr^4+\cdots+ cr^k.\]
Multiplying both sides by $1/10$ we have
\[rs_k=cr+cr^2+cr^3+cr^4+\cdots+cr^{k+1}.\]
It follows that
\[s_k-\ds\frac{s_k}{10}=(1-r)s_k=c-cr^{k+1}\quad\Longrightarrow\quad s_k=c\left(\ds\frac{1-r^{k+1}}{1-r}\right).\]
Now formally,
\[\ds\lim_{k\to\infty}s_k=\ds\lim_{k\to\infty}c\left(\ds\frac{1-r^{k+1}}{1-r}\right)=c\left(\ds\frac{1}{1-r}\right) \quad\text{if}\quad -1<r<1.\] 
Otherwise the limit does not exist. So the geometric series converges to $s=\ds\frac{c}{1-r}$ only when $|r|<1$. In this case, we write
\[\ds\sum^{\infty}_{n=0}cr^k=\ds\frac{c}{1-r}.\] 
Otherwise, the series diverges. This is a fact that should be memorized from this point onwards.

\end{ex}

\vskip 5mm
Let's try another important example.
\vskip 5mm


%%%%%%%%%%%%%%%%%%%%%%%%%%%%%%%%%%%%%%%%%%%%%%%%%%%%%%%%%
%%%%%%%%%%%%%%%%%%%%%%%%%%%%%%%%%%%%%%%%%%%%%%%%%%%%%%%%%

\begin{ex}  Determine if the infinite series $\ds\sum^{\infty}_{n=1}\ds\frac{1}{n}$ converges or diverges.
\vskip 3mm
\note This is an important infinite series known as the {\bf\emph{Harmonic Series}}.
%\vskip 5mm
%\noindent{\bf\emph{\underline{Workspace}:}}
\vskip 5mm
\noindent{\bf\emph{\underline{Workspace}:}}

\vfill\eject

\ssol Let's group terms of the sum as follows.
\[\ds\sum^{\infty}_{n=1}\ds\frac{1}{n}=1+\ds\frac{1}{2}+\underbrace{\ds\frac{1}{3}+\ds\frac{1}{4}}_{\ge 1/2}+\underbrace{\ds\frac{1}{5}+\cdots+\ds\frac{1}{8}}_{\ge 1/2}+\underbrace{\ds\frac{1}{9}+\cdots+\ds\frac{1}{16}}_{\ge 1/2}+\underbrace{\ds\frac{1}{17}+\cdots+\ds\frac{1}{32}}_{\ge 1/2}+\cdots\]
It is easy to see that the sequence of partial sums $\big\{s_k\big\}$ is increasing and not bounded above. Therefore we have
\[\ds\lim_{k\to\infty}s_k=\infty\]
and the infinite series diverges to $\infty$.

\end{ex}

\vskip 5mm

We will discover that for most infinite series it may be practically impossible to discover a formula for the $k^{th}$ partial sum. So we will need to spend a good deal of time developing tests for convergence and divergence of infinite series that don't require knowledge of any formula for the partial sums. This content will occupy a good deal of time in the next part of the course. See you there.
\vskip 5mm


%%%%%%%%%%%%%%%%%%%%%%%%%%%%%%%%%%%%%%%%%%%%%%%%%%%%%%%%%
%%%%%%%%%%%%%%%%%%%%%%%%%%%%%%%%%%%%%%%%%%%%%%%%%%%%%%%%%


%\begin{ex} Determine if the following infinite series converges. 
%\vskip 5mm
%\noindent{\bf\emph{\underline{Workspace}:}}
%
%\vfill\eject
%
%\ssol 
%
%
%\end{ex}
%
%\vskip 5mm
%Let's try another example.
%\vskip 5mm


%%%%%%%%%%%%%%%%%%%%%%%%%%%%%%%%%%%%%%%%%%%%%%%%%%%%%%%%%
%%%%%%%%%%%%%%%%%%%%%%%%%%%%%%%%%%%%%%%%%%%%%%%%%%%%%%%%%




%%%%%%%%%%%%%%%%%%%%%%%%%%%%%%%%%%%%%%%%%%%%%%%%%%%%%%%%%
%%%%%%%%%%%%%%%%%%%%%%%%%%%%%%%%%%%%%%%%%%%%%%%%%%%%%%%%%
%%%%%%%%%%%%%%%%%%%%%%%%%%%%%%%%%%%%%%%%%%%%%%%%%%%%%%%%%
%%%%%%%%%%%%%%%%%%%%%%%%%%%%%%%%%%%%%%%%%%%%%%%%%%%%%%%%%

	
%%%%%%%%%%%%%%%%%%%%%%%%%%%%%%%%%%%%%%%%%%%%%%%%%%%%%%
%%%%%%%%%%%%%%%%%%%%%%%%%%%%%%%%%%%%%%%%%%%%%%%%%%%%%%


\vskip 5mm
\hrule
\vskip 5mm
\begin{center}{\bf Please let me know if you have any questions, comments, or corrections!}
\end{center}	


%%%%%%%%%%%%%%%%%%%%%%%%%%%%%%%%%%%%%%%%%%%%%%%%%%%%%%
\end{document}
%%%%%%%%%%%%%%%%%%%%%%%%%%%%%%%%%%%%%%%%%%%%%%%%%%%%%%
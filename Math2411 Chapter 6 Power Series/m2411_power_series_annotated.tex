\documentclass[11pt]{article}
\usepackage[suffix=Solutions]{teaching-header}

\def\classnum{2411}
\def\classtitle{Calculus II}
\def\classtitleshort{Calc 2}
\def\classsec{H01}
\def\instructor{Dr. Rostermundt}
\def\classterm{Spring 2025}


%%%%%%%%%%%%%%%%%%%%%%%%%%%%%%%%%%%%%%%%%%%%%%%%%%%%%%%%%%%%%%%%%%%%%%%%%%%%
%%%%%%%%%%%%%%%%%%%%%%%%%%%%%%%%%%%%%%%%%%%%%%%%%%%%%%%%%%%%%%%%%%%%%%%%%%%%

%This is defined in the teaching-header style file
%\ifnum\printsol=0 (when no solutions printed)
%Do something
%	\else  (when solutions are printed)
%Do something else
%\fi


% Package and setting included in teachin-header style file
%\RequirePackage{amsmath,amsfonts,amssymb,amsthm,graphicx, pgfplots, tcolorbox, xcolor,latexsym,color,verbatim,float,xcolor,setspace}
%%tikzsymbols
%
%\RequirePackage{enumerate}
%\RequirePackage{multicol}
%\RequirePackage{tikz}
%\RequirePackage{cancel}
%\usetikzlibrary{shapes.geometric}
%\usetikzlibrary{calc, positioning, arrows}
%\RequirePackage[margin=1in,letterpaper]{geometry}
%\RequirePackage[colorlinks=true,allcolors=blue]{hyperref}
%\usepackage[final]{pdfpages}
%%\usepackage{capt-of}
%
%
%\setlength{\textheight}{9in}
%\setlength{\textwidth}{6.5in}
%\addtolength{\topmargin}{0cm}
%%\addtolength{\oddsidemargin}{0cm}
%\parindent=0in
%\parskip=.35em
%\singlespacing
%%\pagestyle{empty}  % remove page numbers

%Add captions without being in figure environment
%\captioof{figure}{\text}\label[fig:]
\usepackage{capt-of}
\usepackage{mathtools}

\vfuzz2pt % Don't report over-full v-boxes if over-edge is small
\hfuzz2pt % Don't report over-full h-boxes if over-edge is small


%%%%%%%%%%%%%%%%%%%%%%%%%%%%%%%%%%%%%%%%%%%%%%%%%%%%%%
%%%%%%%%%%%%%%%%%%%%%%%%%%%%%%%%%%%%%%%%%%%%%%%%%%%%%%

\pagestyle{myheadings}

%%%%%%%%%%%%%%%%%%%%%%%%%%%%%%%%%%%%%%%%%%%%%%%%%%%%%%
%%%%%%%%%%%%%%%%%%%%%%%%%%%%%%%%%%%%%%%%%%%%%%%%%%%%%%


%%%%%%%%%%%%%%%%%%%%%%%%%%%%%%%%%%%%%%%%%%%%%%%%%%%%%%
%%%%%%%%%%%%%%%%%%%%%%%%%   Document Body   %%%%%%%%%%
%%%%%%%%%%%%%%%%%%%%%%%%%%%%%%%%%%%%%%%%%%%%%%%%%%%%%%

%Information from classinfo.tex file
%\def\classnum{2411}
%\def\classtitle{Calculus II}
%\def\classtitleshort{Calc 2}
%\def\classsec{001}
%\def\instructor{Rostermundt}
%\def\classterm{Fall 2024}
\def\topic{Power Series}
\def\topicshort{Power Series}

	\title{\vspace{-1in}Math\classnum\;-\;\classtitle\\
	%Section\;\classsec\;-\;\classterm\\
	Guided Lecture Notes\\
	\topic}
	\author{University of Colorado Denver / College of Liberal Arts and Sciences}
	\date{Department of Mathematics}

	\markright{Math\classnum\;-\;\classtitleshort, University of Colorado Denver,\;\topicshort}


%%%%%%%%%%%%%%%%%%%%%%%%%%%%%%%%%%%%%%%%%%%%%%%%%%%%%%
\begin{document}\maketitle\thispagestyle{empty}
%%%%%%%%%%%%%%%%%%%%%%%%%%%%%%%%%%%%%%%%%%%%%%%%%%%%%%

\hrule

\section*{\topic\, Introduction:}

Our objective is to study a class of infinite series called {\bf\emph{power series}}.
\vskip 5mm

\begin{minipage}[]{6.5in}
\begin{center}
\includegraphics[scale=0.7]{power_series_def.jpg}
%\captionof{figure}{}
\label{fig:powers_series_def}
\end{center}
\end{minipage}


\noindent{\bf\emph{\underline{Question}:}} What do you notice that is different about a power series compared to the series we have previously studied?
\vskip 2in

\noindent{\bf\emph{\underline{Answer}:}} The terms of a power series contain a variable quantity $x$ and so can be thought of as functions $f(x)$. One of the most important questions we ask about a function $f$ is the following: ``What is the domain of $f$?" In other words, what are the $x$-values where our function $f(x)$ is defined (or makes sense)? Since we are dealing with an infinite series we could pose this question as follows: For which $x$-values does the power series converge? We can summarize the answer to this question with the following theorem.

\vfill\eject

\begin{minipage}[]{6.5in}
\begin{center}
\includegraphics[scale=0.75]{power_series_thm.jpg}
%\captionof{figure}{}
\label{fig:}
\end{center}
\end{minipage}

\vskip 5mm

Look at the following diagram to get a better sense of the statement of the theorem.

\vskip 5mm

\begin{minipage}[]{6.5in}
\begin{center}
\includegraphics[scale=0.7]{interval_convergence_graphic.jpg}
\captionof{figure}{Three Convergence Possibilities for a Power Series}
\label{fig:interval_converge_graphic}
\end{center}
\end{minipage}

\vskip 1cm

Here is some useful terminology that we will be using.

\vskip 1cm

\begin{minipage}[]{6.5in}
\begin{center}
\includegraphics[scale=0.75]{interval_convergence_def.jpg}
%\captionof{figure}{}
\label{fig:interval_converge_def}
\end{center}
\end{minipage}

\vskip 1cm

Let's work some examples and discover the process.


%%%%%%%%%%%%%%%%%%%%%%%%%%%%%%%%%%%%%%%%%%%%%%%%%%%%%%%%%
%%%%%%%%%%%%%%%%%%%%%%%%%%%%%%%%%%%%%%%%%%%%%%%%%%%%%%%%%

\section*{Power Series Examples:}

\vskip 5mm
\begin{example} Find the interval of convergence and the radius of convergence for the power series
\[\ds\sum^{\infty}_{n=0}\ds\frac{x^n}{n!}.\]
\vskip 5mm
\noindent{\bf\emph{\underline{Workspace}:}}

\vfill\eject

\ifnum\longform=1
	\begin{boxsolution}
\vspace*{5mm}
Observe that the power series is centered at $a=0$. We will use the Ratio Test and so compute
\beq
r&=&\ds\lim_{n\to\infty}\left|\ds\frac{a_{n+1}}{a_n}\right|\\
\\
&=&\ds\lim_{n\to\infty}\left|\ds\frac{x^{n+1}}{(n+1)!}\cdot\ds\frac{n!}{x^n}\right|\\
\\
&=&\ds\lim_{n\to\infty}\left|\ds\frac{x}{n+1}\right|\\
\\
&=&|x|\ds\lim_{n\to\infty}\ds\frac{1}{n+1}\\
\\
&=&0<1\quad\text{for all }x\text{-values}.
\eeq
So the power series converges for all $x\in\R$ and the interval of convergence is $(-\infty,\infty)$ and radius of convergence is $R=\infty$.
\vspace*{5mm} 
	\end{boxsolution}

\vskip 5mm

\fi
	
\end{example}


Let's try another example.
\vskip 5mm

%%%%%%%%%%%%%%%%%%%%%%%%%%%%%%%%%%%%%%%%%%%%%%%%%%%%%%%%%
%%%%%%%%%%%%%%%%%%%%%%%%%%%%%%%%%%%%%%%%%%%%%%%%%%%%%%%%%


\vskip 5mm
\begin{example} Find the interval of convergence and the radius of convergence for the power series
\[\ds\sum^{\infty}_{n=0}\ds\frac{(x-2)^n}{(n+1)3^n}.\]
\vskip 5mm
\noindent{\bf\emph{\underline{Workspace}:}}

\vfill\eject

\ifnum\longform=1
\noindent{\bf\emph{\underline{Workspace Continued}:}}

\vfill\eject

\fi

\ifnum\longform=1
	\begin{boxsolution}
Observe that the power series is centered at $a=2$. We will use the Ratio Test and so compute
\beq
r&=&\ds\lim_{n\to\infty}\left|\ds\frac{a_{n+1}}{a_n}\right|\\
\\
&=&\ds\lim_{n\to\infty}\left|\ds\frac{(x-2)^{n+1}}{(n+2)3^{n+1}}\cdot\ds\frac{(n+1)3^n}{(x-2)^n}\right|\\
\\
&=&\ds\lim_{n\to\infty}\left|\ds\frac{(x-2)(n+1)}{3(n+2)}\right|\\
\\
&=&|x-2|\ds\lim_{n\to\infty}\ds\frac{n+1}{3(n+2)}\\
\\
&=&\ds\frac{1}{3}\cdot|x-2|
\eeq
So we solve $r=\frac{1}{3}|x-2|<1$ for $x$.
\[\frac{1}{3}\cdot|x-2|<1\quad\iff\quad\underbrace{|x-2|<3}_{\textcolor{red}{R=3}}\quad\iff\quad-3<x-2<3\quad\iff\quad -1<x<5\]
So we see that the radius of convergence is $R=3$ and by the Ratio Test the power series converges absolutely when $-1<x<5$ and diverges when $x<-1$ and when $x>5$. However, the Ratio Test fails at $x=-1$ and $x=5$ because at those points we have $r=1$. So we must check for convergence at those endpoint $x$-values using some other convergence test.
\vskip 5mm

	\begin{enumerate}
		\item[$\bullet$] $x=-1$: We have the series $\ds\sum^{\infty}_{n=0}\ds\frac{(-3)^n}{(n+1)3^n}=\ds\sum^{\infty}_{n=1}(-1)^{n+1}\ds\frac{1}{n}$ which is the convergent alternating Harmonic Series. The original power series converges at $x=-1$.
\vskip 5mm
		\item[$\bullet$] $x=5$: We have the series $\ds\sum^{\infty}_{n=0}\ds\frac{3^n}{(n+1)3^n}=\ds\sum^{\infty}_{n=1}\ds\frac{1}{n}$ which is the divergent Harmonic Series. The original power series diverges at $x=5$.
		
	\end{enumerate}

So we have the interval of convergence is $[-1,5)$. Observe that the center of the interval is $a=2$.
\vspace*{5mm}
	\end{boxsolution}

\vskip 5mm

\fi

\end{example}


Let's try another example.
\vskip 5mm


%%%%%%%%%%%%%%%%%%%%%%%%%%%%%%%%%%%%%%%%%%%%%%%%%%%%%%%%%
%%%%%%%%%%%%%%%%%%%%%%%%%%%%%%%%%%%%%%%%%%%%%%%%%%%%%%%%%


\vskip 5mm
\begin{example} Find the interval of convergence and the radius of convergence for the power series
\[\ds\sum^{\infty}_{n=0}\ds\frac{x^{2n+2}}{4^{n+1}}.\]

\ifnum\longform=1
\vfill\eject

\fi

\noindent{\bf\emph{\underline{Workspace}:}}

\vfill\eject

\ifnum\longform=1
	\begin{boxsolution}
Observe that the power series is centered at $a=0$. We will use the Ratio Test and so compute
\beq
r&=&\ds\lim_{n\to\infty}\left|\ds\frac{a_{n+1}}{a_n}\right|\\
\\
&=&\ds\lim_{n\to\infty}\left|\ds\frac{x^{2n+4}}{4^{n+2}}\cdot\ds\frac{4^{n+1}}{x^{2n+2}}\right|\\
\\
&=&\ds\lim_{n\to\infty}\left|\ds\frac{x^2}{4}\right|\\
\\
&=&\ds\frac{x^2}{4}\\
\eeq
So we solve $r=x^2/4<1$ for $x$.
\[\frac{x^2}{4}<1\quad\iff\quad x^2<4\quad\iff\quad\underbrace{|x|<2}_{\textcolor{red}{R=2}}\quad\iff\quad -2<x<2\]
So we see that the radius of convergence is $R=2$ and by the Ratio Test the power series converges absolutely when $-2<x<2$ and diverges when $x<-2$ and when $x>2$. However, the Ratio Test fails at $x=-2$ and $x=2$ because at those points we have $r=1$. So we must check for convergence at those endpoint $x$-values using some other convergence test.
\vskip 5mm

	\begin{enumerate}
		\item[$\bullet$] $x=-2$: We have the series $\ds\sum^{\infty}_{n=0}\ds\frac{(-2)^{2n+2}}{4^{n+1}}=\ds\sum^{\infty}_{n=0}1$ which is a divergent series. The original power series diverges at $x=-2$.
\vskip 5mm
		\item[$\bullet$] $x=2$: We have the series $\ds\sum^{\infty}_{n=0}\ds\frac{2^{2n+2}}{4^{n+1}}=\ds\sum^{\infty}_{n=0}1$ which is a divergent series. The original power series diverges at $x=2$.
		
	\end{enumerate}

So we have the interval of convergence is $(-2,2)$. Observe that the center of the interval is $a=0$.
\vspace*{5mm}
	\end{boxsolution}

\vskip 5mm

\fi

\end{example}


Let's try another example.


%%%%%%%%%%%%%%%%%%%%%%%%%%%%%%%%%%%%%%%%%%%%%%%%%%%%%%%%%
%%%%%%%%%%%%%%%%%%%%%%%%%%%%%%%%%%%%%%%%%%%%%%%%%%%%%%%%%


\vskip 5mm
\begin{example} Find the interval of convergence and the radius of convergence for the power series
\[\ds\sum^{\infty}_{n=0}\ds\frac{n(x+1)^{n}}{e^n}.\]
\ifnum\longform=0
\vskip 5mm
	\else
\vfill\eject

\fi

\noindent{\bf\emph{\underline{Workspace}:}}

\vfill\eject

%\ifnum\longform=1
%\noindent{\bf\emph{\underline{Workspace Continued}:}}
%
%\vfill\eject
%
%\fi

\ifnum\longform=1
	\begin{boxsolution}
Observe that the power series is centered at $a=0$. We will use the Ratio Test and so compute
\beq
r&=&\ds\lim_{n\to\infty}\left|\ds\frac{a_{n+1}}{a_n}\right|\\
\\
&=&\ds\lim_{n\to\infty}\left|\ds\frac{(n+1)(x+1)^{n+1}}{e^{n+1}}\cdot\ds\frac{e^{n}}{n(x+1)^{n}}\right|\\
\\
&=&\ds\frac{|x+1|}{e}\ds\lim_{n\to\infty}\ds\frac{n+1}{n}\\
\\
&=&\ds\frac{|x+1|}{e}\\
\eeq
So we solve $r=|x+1|/e<1$ for $x$.
\[\frac{|x+1|}{e}<1\quad\iff\quad\underbrace{|x+1|<e}_{\textcolor{red}{R=e}}\quad\iff\quad -e<x+1<e\quad\iff\quad -1-e<x<-1+e\]
So we see that the radius of convergence is $R=e$ and by the Ratio Test the power series converges absolutely when $-1-e<x<-1+e$ and diverges when $x<-1-e$ and when $x>-1+e$. However, the Ratio Test fails at $x=-1-e$ and $x=-1+e$ because at those points we have $r=1$. So we must check for convergence at those endpoint $x$-values using some other convergence test.
\vskip 5mm

	\begin{enumerate}
		\item[$\bullet$] $x=-1-e$: We have the series $\ds\sum^{\infty}_{n=0}\ds\frac{n\big(-e\,\big)^{n}}{e^n}=\ds\sum^{\infty}_{n=0}(-1)^n\cdot n$ which is a divergent series. The original power series diverges at $x=-1-e$.
\vskip 5mm
		\item[$\bullet$] $x=-1+e$: We have the series $\ds\sum^{\infty}_{n=0}\ds\frac{ne^n}{e^n}=\ds\sum^{\infty}_{n=0}n$ which is a divergent series. The original power series diverges at $x=-1+e$.
		
	\end{enumerate}
\vskip 5mm
So we have the interval of convergence is $(-1-e,-1+e)$. Observe that the center of the interval is $a=-1$.
\vspace*{5mm}
	\end{boxsolution}

\vskip 5mm

\fi

\end{example}


%%%%%%%%%%%%%%%%%%%%%%%%%%%%%%%%%%%%%%%%%%%%%%%%%%%%%%%%%
%%%%%%%%%%%%%%%%%%%%%%%%%%%%%%%%%%%%%%%%%%%%%%%%%%%%%%%%%

 
%%%%%%%%%%%%%%%%%%%%%%%%%%%%%%%%%%%%%%%%%%%%%%%%%%%%%%%%%
%%%%%%%%%%%%%%%%%%%%%%%%%%%%%%%%%%%%%%%%%%%%%%%%%%%%%%%%%
%%%%%%%%%%%%%%%%%%%%%%%%%%%%%%%%%%%%%%%%%%%%%%%%%%%%%%%%%
%%%%%%%%%%%%%%%%%%%%%%%%%%%%%%%%%%%%%%%%%%%%%%%%%%%%%%%%%

	
%%%%%%%%%%%%%%%%%%%%%%%%%%%%%%%%%%%%%%%%%%%%%%%%%%%%%%
%%%%%%%%%%%%%%%%%%%%%%%%%%%%%%%%%%%%%%%%%%%%%%%%%%%%%%

\ifnum\longform=1

\vskip 1cm
\hrule
\vskip 5mm
\begin{center}{\bf Please let me know if you have any questions, comments, or corrections!}
\end{center}	

\fi

%%%%%%%%%%%%%%%%%%%%%%%%%%%%%%%%%%%%%%%%%%%%%%%%%%%%%%
\end{document}
%%%%%%%%%%%%%%%%%%%%%%%%%%%%%%%%%%%%%%%%%%%%%%%%%%%%%%
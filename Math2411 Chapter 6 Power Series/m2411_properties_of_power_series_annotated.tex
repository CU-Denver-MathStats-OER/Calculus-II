\documentclass[11pt]{article}
\usepackage[suffix=Solutions]{teaching-header}

\def\classnum{2411}
\def\classtitle{Calculus II}
\def\classtitleshort{Calc 2}
\def\classsec{H01}
\def\instructor{Dr. Rostermundt}
\def\classterm{Spring 2025}


%%%%%%%%%%%%%%%%%%%%%%%%%%%%%%%%%%%%%%%%%%%%%%%%%%%%%%%%%%%%%%%%%%%%%%%%%%%%
%%%%%%%%%%%%%%%%%%%%%%%%%%%%%%%%%%%%%%%%%%%%%%%%%%%%%%%%%%%%%%%%%%%%%%%%%%%%

%This is defined in the teaching-header style file
%\ifnum\printsol=0 (when no solutions printed)
%Do something
%	\else  (when solutions are printed)
%Do something else
%\fi


% Package and setting included in teachin-header style file
%\RequirePackage{amsmath,amsfonts,amssymb,amsthm,graphicx, pgfplots, tcolorbox, xcolor,latexsym,color,verbatim,float,xcolor,setspace}
%%tikzsymbols
%
%\RequirePackage{enumerate}
%\RequirePackage{multicol}
%\RequirePackage{tikz}
%\RequirePackage{cancel}
%\usetikzlibrary{shapes.geometric}
%\usetikzlibrary{calc, positioning, arrows}
%\RequirePackage[margin=1in,letterpaper]{geometry}
%\RequirePackage[colorlinks=true,allcolors=blue]{hyperref}
%\usepackage[final]{pdfpages}
%%\usepackage{capt-of}
%
%
%\setlength{\textheight}{9in}
%\setlength{\textwidth}{6.5in}
%\addtolength{\topmargin}{0cm}
%%\addtolength{\oddsidemargin}{0cm}
%\parindent=0in
%\parskip=.35em
%\singlespacing
%%\pagestyle{empty}  % remove page numbers

%Add captions without being in figure environment
%\captioof{figure}{\text}\label[fig:]
\usepackage{capt-of}
\usepackage{mathtools}

\vfuzz2pt % Don't report over-full v-boxes if over-edge is small
\hfuzz2pt % Don't report over-full h-boxes if over-edge is small


%%%%%%%%%%%%%%%%%%%%%%%%%%%%%%%%%%%%%%%%%%%%%%%%%%%%%%
%%%%%%%%%%%%%%%%%%%%%%%%%%%%%%%%%%%%%%%%%%%%%%%%%%%%%%

\pagestyle{myheadings}

%%%%%%%%%%%%%%%%%%%%%%%%%%%%%%%%%%%%%%%%%%%%%%%%%%%%%%
%%%%%%%%%%%%%%%%%%%%%%%%%%%%%%%%%%%%%%%%%%%%%%%%%%%%%%


%%%%%%%%%%%%%%%%%%%%%%%%%%%%%%%%%%%%%%%%%%%%%%%%%%%%%%
%%%%%%%%%%%%%%%%%%%%%%%%%   Document Body   %%%%%%%%%%
%%%%%%%%%%%%%%%%%%%%%%%%%%%%%%%%%%%%%%%%%%%%%%%%%%%%%%

%Information from classinfo.tex file
%\def\classnum{2411}
%\def\classtitle{Calculus II}
%\def\classtitleshort{Calc 2}
%\def\classsec{001}
%\def\instructor{Rostermundt}
%\def\classterm{Fall 2024}
\def\topic{Properties of Power Series}
\def\topicshort{Properties of Power Series}

	\title{\vspace{-1in}Math\classnum\;-\;\classtitle\\
	%Section\;\classsec\;-\;\classterm\\
	Guided Lecture Notes\\
	\topic}
	\author{University of Colorado Denver / College of Liberal Arts and Sciences}
	\date{Department of Mathematics}

	\markright{Math\classnum\;-\;\classtitleshort, University of Colorado Denver,\;\topicshort}


%%%%%%%%%%%%%%%%%%%%%%%%%%%%%%%%%%%%%%%%%%%%%%%%%%%%%%
\begin{document}\maketitle\thispagestyle{empty}
%%%%%%%%%%%%%%%%%%%%%%%%%%%%%%%%%%%%%%%%%%%%%%%%%%%%%%

\hrule

\section*{\topic\, Introduction:}

One of the best things about working with power series is they are very easy to manipulate to create other series. For example, we can add, subtract, scale, and multiply power series in the most natural ways.

\vskip 5mm

\begin{minipage}[]{6.5in}
\begin{center}
\includegraphics[scale=0.6]{combining_power_series.jpg}
%\captionof{figure}{}
\label{fig:powers_series_def}
\end{center}
\end{minipage}

\vskip 5mm

\begin{minipage}[]{6.5in}
\begin{center}
\includegraphics[scale=0.6]{multiplying_power_series.jpg}
%\captionof{figure}{}
\label{fig:powers_series_def}
\end{center}
\end{minipage}


\vfill\eject

We can also make substitutions. Lets consider a few examples with substitutions.
\vskip 5mm


%%%%%%%%%%%%%%%%%%%%%%%%%%%%%%%%%%%%%%%%%%%%%%%%%%%%%%%%%
%%%%%%%%%%%%%%%%%%%%%%%%%%%%%%%%%%%%%%%%%%%%%%%%%%%%%%%%%

\section*{Power Series Examples:}

\vskip 5mm
\begin{example} We can show that the function $f(x)=\frac{1}{1-x}$ can be represented as the following power series with interval of convergence $(-1,1)$.
\[\ds\frac{1}{1-x}=\ds\sum^{\infty}_{n=0}x^n=1+x+x^2+x^3+x^4+\cdots\]
Can we use this series to find a power series representation of the function $\frac{1}{1+2x}$?
\vskip 5mm
\noindent{\bf\emph{\underline{Workspace}:}}

\vfill\eject

\ifnum\longform=1
	\begin{boxsolution}
We see that $\frac{1}{1+2x}=\frac{1}{1-(-2x)}=f(-2x)$. So we have the power series representation
\[\ds\frac{1}{1+2x}=\ds\sum^{\infty}_{n=0}(-2x)^n=\ds\sum^{\infty}_{n=0}(-1)^n2^nx^n=1-2x+4x^2-8x^3+16x^4+\cdots\]
when $-1<-2x<1$, which is equivalent to $-1/2<x<1/2$. The radius of convergence for the new series is $(-1/2,1/2)$ and the radius of convergence as $R=1/2$. Notice that with this substitution we do not need a Ratio Test to determine the radius of convergence and interval of convergence. We simply use the known radius of convergence and interval of convergence for the original series.
\vspace*{5mm}
	\end{boxsolution}
\vskip 5mm

\fi

\end{example}

Let's try another example.
\vskip 5mm

%%%%%%%%%%%%%%%%%%%%%%%%%%%%%%%%%%%%%%%%%%%%%%%%%%%%%%%%%
%%%%%%%%%%%%%%%%%%%%%%%%%%%%%%%%%%%%%%%%%%%%%%%%%%%%%%%%%


\vskip 5mm
\begin{example} We can show that the function $f(x)=\frac{1}{1-x}$ can be represented as the following power series with interval of convergence $(-1,1)$.
\[\ds\frac{1}{1-x}=\ds\sum^{\infty}_{n=0}x^n=1+x+x^2+x^3+x^4+\cdots\]
Can we use this series to find a power series representation of the function $\frac{1}{1+x^2}$?
\vskip 5mm
\noindent{\bf\emph{\underline{Workspace}:}}

\vfill\eject

\ifnum\longform=1
	\begin{boxsolution}
We see that $\frac{1}{1+x^2}=\frac{1}{1-\left(-x^2\right)}=f\left(-x^2\right)$. So we have the power series representation
\[\ds\frac{1}{1+x^2}=\ds\sum^{\infty}_{n=0}(-x^2)^n=\ds\sum^{\infty}_{n=0}(-1)^nx^{2n}=1-x^2+x^4-x^6+x^8+x^{10}+\cdots\]
when $-1<-x^2<1$, which is equivalent to $-1<x<1$. The radius of convergence for the new series is $(-1,1)$ and the radius of convergence as $R=1$. Notice that with this substitution we do not need a Ratio Test to determine the radius of convergence and interval of convergence. We simply use the known radius of convergence and interval of convergence for the original series.
\vspace*{5mm}
	\end{boxsolution}
\vskip 5mm

\fi

\end{example}


Let's try another example.
\vskip 5mm

%%%%%%%%%%%%%%%%%%%%%%%%%%%%%%%%%%%%%%%%%%%%%%%%%%%%%%%%%
%%%%%%%%%%%%%%%%%%%%%%%%%%%%%%%%%%%%%%%%%%%%%%%%%%%%%%%%%


\vskip 5mm
\begin{example} We can show that the function $f(x)=\frac{1}{1+x^2}$ can be represented as the following power series with interval of convergence $(-1,1)$.
\[\ds\frac{1}{1+x^2}=\ds\sum^{\infty}_{n=0}(-1)^nx^{2n}=1-x^2+x^4-x^6+x^8-x^{10}+\cdots\]
Can we use this series to find a power series representation of the function $\tan^{-1}(x)$? Yes.

\vskip 5mm

\begin{minipage}[]{6.5in}
\begin{center}
\includegraphics[scale=0.6]{calc_of_power_series.jpg}
%\captionof{figure}{}
\label{fig:powers_series_def}
\end{center}
\end{minipage}

\ifnum\longform=1
\vfill\eject

\noindent{\bf\emph{\underline{Workspace}:}}

\vfill\eject

	\else
\vskip 5mm
\noindent{\bf\emph{\underline{Workspace}:}}

\vfill\eject

\noindent{\bf\emph{\underline{Workspace Cont.}:}}

\vfill\eject

\fi

\ifnum\longform=1
	\begin{boxsolution}
We know that $\tan^{-1}(x)=\int^{t=x}_{t=0}\frac{1}{1+t^2}\,dt$. It follows that the power series for $\tan^{-1}(x)$ can be found by integrating the series for $\frac{1}{1+x^2}$.
\beq
\ds\int^{t=x}_{t=0}(-1)^nt^{2n}\,dt&=&\ds\int^{t=x}_{t=0}1-t^2+t^4-t^6+t^8-t^{10}+\cdots\,dt\\
\\
&=&t-\ds\frac{t^3}{3}+\ds\frac{t^5}{5}-\ds\frac{t^7}{7}+\ds\frac{t^9}{9}-\ds\frac{t^{11}}{11}+\cdots\,\Bigg|^{t=x}_{t=0}\\
\\
&=&\left(x-\ds\frac{x^3}{3}+\ds\frac{x^5}{5}-\ds\frac{x^7}{7}+\ds\frac{x^9}{9}-\ds\frac{x^{11}}{11}+\cdots\right)-\Big(0-0+0-0+0-0+\cdots\Big)\\
\\
&=&x-\ds\frac{x^3}{3}+\ds\frac{x^5}{5}-\ds\frac{x^7}{7}+\ds\frac{x^9}{9}-\ds\frac{x^{11}}{11}+\cdots\\
\\
&=&\ds\sum^{\infty}_{n=0}(-1)^n\ds\frac{x^{2n+1}}{2n+1}
\eeq
\vskip 5mm
The previous theorem tells us that the new series will have the same radius of convergence, and so the series will converge on the interval $(-1,1)$. However, integrating or differentiating a power series can change convergence or divergence at the endpoints of the interval of convergence. So we check the endpoints.
\vskip 5mm
	\begin{enumerate}
		\item[$\bullet$]\; $x=-1$:\quad\, $\ds\sum^{\infty}_{n=0}(-1)^n\ds\frac{(-1)^{2n+1}}{2n+1}=-\ds\sum^{\infty}_{n=0}(-1)^n\ds\frac{1}{2n+1}$ is a convergent alternating series.
\vskip 5mm 
		\item[$\bullet$]\; $x=1$:\qquad $\ds\sum^{\infty}_{n=0}(-1)^n\ds\frac{1^{2n+1}}{2n+1}=\ds\sum^{\infty}_{n=0}(-1)^n\ds\frac{1}{2n+1}$ is a convergent alternating series.
	\end{enumerate}
\vskip 5mm
So the interval	of convergence is $[-1,1]$ and we have for $-1\le x\le 1$ that
\[\tan^{-1}(x)=\ds\sum^{\infty}_{n=0}(-1)^n\ds\frac{x^{2n+1}}{2n+1}\]
\vspace*{5mm}
	\end{boxsolution}
\vskip 5mm

\fi
	
\end{example}

\vskip 5mm
Let's try another example.
\vskip 5mm

%%%%%%%%%%%%%%%%%%%%%%%%%%%%%%%%%%%%%%%%%%%%%%%%%%%%%%%%%
%%%%%%%%%%%%%%%%%%%%%%%%%%%%%%%%%%%%%%%%%%%%%%%%%%%%%%%%%

\ifnum\longform=1
\vfill\eject

\fi

\begin{example} We can show that the function $f(x)=\sin(x)$ can be represented as the following power series with interval of convergence $(-\infty,\infty)$.
\[\sin(x)=\ds\sum^{\infty}_{n=0}(-1)^n\ds\frac{x^{2n+1}}{(2n+1)!}=x-\ds\frac{x^3}{3!}+\ds\frac{x^5}{5!}-\ds\frac{x^7}{7!}+\ds\frac{x^9}{9!}-\ds\frac{x^{11}}{11!}+\cdots\]
Can we use this series to find a power series representation of the function $\cos(x)$? Yes.
\vskip 5mm
\noindent{\bf\emph{\underline{Workspace}:}}

\vfill\eject

\ifnum\longform=1
	\begin{boxsolution}
We know that $\cos(x)=\frac{d}{dx}\Big[\sin(x)\Big]$. It follows that the power series for $\cos(x)$ can be found by differentiating the power series for $\sin(x)$.
\beq
\frac{d}{dx}\Big[x-\ds\frac{x^3}{3!}+\ds\frac{x^5}{5!}-\ds\frac{x^7}{7!}+\ds\frac{x^9}{9!}-\ds\frac{x^{11}}{11!}+\cdots\Big]&=&1-\ds\frac{x^2}{2!}+\ds\frac{x^4}{4!}-\ds\frac{x^6}{6!}+\ds\frac{x^8}{8!}-\ds\frac{x^{10}}{10!}+\cdots
\\
&=&\ds\sum^{\infty}_{n=0}(-1)^n\ds\frac{x^{2n}}{(2n)!}
\eeq
\vskip 5mm
The previous theorem tells us that the new series will have the same radius of convergence, and so the series will converge on the interval $(-\infty,\infty)$. We have for all $x$-values that
\[\cos(x)=\ds\sum^{\infty}_{n=0}(-1)^n\ds\frac{x^{2n}}{(2n)!}=1-\ds\frac{x^2}{2!}+\ds\frac{x^4}{4!}-\ds\frac{x^6}{6!}+\ds\frac{x^8}{8!}-\ds\frac{x^{10}}{10!}+\cdots\]
\vspace*{5mm}
	\end{boxsolution}
\vskip 5mm

\fi

\end{example}

\ifnum\longform=1
\vskip 2mm
Hopefully we can see that power series are fairly easy to manipulate power series to discover power series for certain desired functions. Let's consider one more example along with an application.
\vskip 5mm

\fi

%%%%%%%%%%%%%%%%%%%%%%%%%%%%%%%%%%%%%%%%%%%%%%%%%%%%%%%%%
%%%%%%%%%%%%%%%%%%%%%%%%%%%%%%%%%%%%%%%%%%%%%%%%%%%%%%%%%


\vskip 5mm
\begin{example} Use the power series for $\sin(x)$ centered at $a=0$ to evaluate the integral $\ds\int^{x=1}_{x=0}\ds\frac{\sin(x)}{x}\,dx$.
%We can show that the function $f(x)=e^x$ can be represented as the following power series with interval of convergence $(-\infty,\infty)$.
%\[e^x=\ds\sum^{\infty}_{n=0}\ds\frac{x^n}{n!}=1+x+\ds\frac{x^2}{2!}+\ds\frac{x^3}{3!}+\ds\frac{x^4}{4!}+\ds\frac{x^5}{5!}+\cdots\]
%Can we use this series to find a power series representation of the function $e^{-x^2}$? Yes.
\vskip 5mm
\noindent{\bf\emph{\underline{Workspace}:}}

\vfill\eject

\ifnum\longform=1
\noindent{\bf\emph{\underline{Workspace Continued}:}}

\vfill\eject

\fi

\ifnum\longform=1
	\begin{boxsolution}
\vspace*{5mm}
We have the power series representation
\[\ds\frac{\sin(x)}{x}=\ds\frac{1}{x}\ds\sum^{\infty}_{n=0}(-1)^n\ds\frac{x^{2n+1}}{(2n+1)!}=\ds\sum^{\infty}_{n=0}(-1)^n\ds\frac{x^{2n}}{(2n+1)!}\]
%=1-\ds\frac{x^2}{3!}+\ds\frac{x^4}{5!}-\ds\frac{x^6}{7!}+\ds\frac{x^8}{9!}-\ds\frac{x^{10}}{11!}+\cdots\]
We can also check that this series converges for all $x$-values. We can use this to solve the difficult problem of evaluating the following integral.
\[\ds\int^{x=1}_{x=0}\ds\frac{\sin(x)}{x}\,dx\]
The difficulty is that there is no elementary antiderivative for the function $\sin(x)/x$ Fortunately, we can integrate power series (all power series correspond to continuous functions) and so we have
\beq
\ds\int^{x=1}_{x=0}\ds\frac{\sin(x)}{x}\,dx&=&\ds\int^{x=1}_{x=0}\,\left(\,\ds\sum^{\infty}_{n=0}(-1)^n\ds\frac{x^{2n}}{(2n+1)!}\,\right)\,dx\\
\\
&=&\ds\int^{x=1}_{x=0}1-\ds\frac{x^2}{3!}+\ds\frac{x^4}{5!}-\ds\frac{x^6}{7!}+\ds\frac{x^8}{9!}-\ds\frac{x^{10}}{11!}+\cdots\,dx\\
\\
&=&x-\ds\frac{x^3}{3\cdot 3!}+\ds\frac{x^5}{5\cdot 5!}-\ds\frac{x^7}{7\cdot 7!}+\ds\frac{x^9}{9\cdot 9!}-\cdots\ds\Bigg|^{x=1}_{x=0}\\
\\
&=&\left(1-\ds\frac{1}{3\cdot 3!}+\ds\frac{1}{5\cdot 5!}-\ds\frac{1}{7\cdot 7!}+\ds\frac{1}{9\cdot 9!}-\cdots\right)-\Big(0-0+0-0+0-0+\cdots\Big)\\
\\
&=&\ds\sum^{\infty}_{n=0}(-1)^n\ds\frac{1}{(2n+1)(2n+1)!}
\eeq

\vskip 5mm

This is an alternating series and so we can easily approximate the value of the integral and have an easy analysis of the error in our approximation. For example, consider the $3^{th}$ partial sum.
\[S_6=\ds\sum^{3}_{n=0}(-1)^n\ds\frac{1}{(2n+1)(2n+1)!}=1-\ds\frac{1}{3\cdot 3!}+\ds\frac{1}{5\cdot 5!}-\ds\frac{1}{7\cdot 7!}=0.9460827664\]
We know that the error can be analyzed as follows.
\[|R_3|=\big|S-S_3\big|<\ds\frac{1}{9\cdot 9!}=0.000000306.\]
\[\vdots\]
\vskip 5mm
	\end{boxsolution}
	\begin{boxsolutioncont}
\vskip 2mm
\[\vdots\]
So we have the following range of values:
\vskip 2mm
\[0.9460824602<\ds\int^{x=1}_{x=0}e^{-x^2}\,dx<0.9460830726\]
\vskip 2mm
and our estimate $S_3$ is accurate to at least five decimal places. The computer generated result is 
\[\ds\int^{x=1}_{x=0}\ds\frac{\sin(x)}{x}\,dx=0.9460830704.\]
\vskip 2mm
	\end{boxsolutioncont}

\vskip 5mm

\fi

\end{example}

\ifnum\longform=1

We can make a few more comments about this power series expansion for $f(x)=\sin(x)/x$.
\vskip 5mm


	\begin{discussion}	
\vskip 2mm
Notice that we would describe the domain of $f(x)=\sin(x)/x$ as $(-\infty,0)\cup(0,\infty)$ since we can not divide by zero. But the power series representation of $f(x)$ can be evaluated at $x=0$.
\[\ds\sum^{\infty}_{n=0}(-1)^n\ds\frac{x^{2n}}{(2n+1)!}\,\ds\Bigg|_{x=0}=1-0+0-0+0-0+\cdots=1.\]
So the power series representation for $f(x)=\sin(x)/x$ has domain $\R=(-\infty,\infty)$ and so we have ``extended" the domain of the original function $f(x)$. This is interesting! Moreover, recall from Calculus I that
\[\ds\lim_{x\to 0}\ds\frac{\sin(x)}{x}=1\]
and since power series represent continuous functions we would need to have the function value at $x=0$ equal to $1$. The value of the power series at $x=0$ matches this requirement exactly.
\vskip 5mm
\begin{minipage}[]{6.5in}
\begin{center}
\includegraphics[scale=0.5]{sinx_over_x.jpg}
\captionof{figure}{Graph of $y=\sin(x)/x$.}
\label{fig:sinx_over_x}
\end{center}
\end{minipage}

\vskip 5mm

We can see that our knowledge of Taylor series makes us pretty powerful when it comes to solving certain difficult problems.
\[\vdots\]
\vskip 5mm

	\end{discussion}

\vfill\eject

	\begin{discussion}
\vskip 2mm
\[\vdots\]
Note that the same process would not work for the function $g(x)=\cos(x)/x$. If we use the power series expansion of $g(x)$ we would have
\[\ds\frac{\cos(x)}{x}=\ds\frac{1}{x}\ds\sum^{\infty}_{n=0}(-1)^n\ds\frac{x^{2n}}{(2n)!}=\ds\frac{1}{x}-\ds\frac{x}{2!}+\ds\frac{x^3}{4!}-\ds\frac{x^5}{6!}+\cdots.\]
This is not a power series (because of the first term) and so results about power series do not apply. Moreover, the series can not be evaluated at $x=0$ and the domain of $g(x)$ can not be extended using this power series. This is not surprising since $f(x)=\cos(x)/x$ has a vertical asymptote at $x=0$.
\vskip 5mm
\begin{minipage}[]{6.5in}
\begin{center}
\includegraphics[scale=0.5]{cosx_over_x.jpg}
\captionof{figure}{Graph of $y=\cos(x)/x$.}
\label{fig:cosx_over_x}
\end{center}
\end{minipage} 
\vspace*{1cm}	

We can freely multiply a power series by a quantity $bx^m$ where $m\ge 0$. But when multiplying by other quantities we must be much more careful about the resulting series.
\vspace*{1cm}

	\end{discussion}
	

\fi



%%%%%%%%%%%%%%%%%%%%%%%%%%%%%%%%%%%%%%%%%%%%%%%%%%%%%%%%%
%%%%%%%%%%%%%%%%%%%%%%%%%%%%%%%%%%%%%%%%%%%%%%%%%%%%%%%%%


%%%%%%%%%%%%%%%%%%%%%%%%%%%%%%%%%%%%%%%%%%%%%%%%%%%%%%%%%
%%%%%%%%%%%%%%%%%%%%%%%%%%%%%%%%%%%%%%%%%%%%%%%%%%%%%%%%%
%%%%%%%%%%%%%%%%%%%%%%%%%%%%%%%%%%%%%%%%%%%%%%%%%%%%%%%%%
%%%%%%%%%%%%%%%%%%%%%%%%%%%%%%%%%%%%%%%%%%%%%%%%%%%%%%%%%

	
%%%%%%%%%%%%%%%%%%%%%%%%%%%%%%%%%%%%%%%%%%%%%%%%%%%%%%
%%%%%%%%%%%%%%%%%%%%%%%%%%%%%%%%%%%%%%%%%%%%%%%%%%%%%%

\ifnum\longform=1
\vskip 1cm
\hrule
\vskip 5mm
\begin{center}{\bf Please let me know if you have any questions, comments, or corrections!}
\end{center}	

\fi

%%%%%%%%%%%%%%%%%%%%%%%%%%%%%%%%%%%%%%%%%%%%%%%%%%%%%%
\end{document}
%%%%%%%%%%%%%%%%%%%%%%%%%%%%%%%%%%%%%%%%%%%%%%%%%%%%%%